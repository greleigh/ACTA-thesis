% Emacs, this is -*-latex-*-

\chapter{Cyclic Arithmetics}
\label{chap:ca}

\section{Simple Cyclic Heyting Arithmetic}
\label{sec:cha}

We begin by introducing a cyclic proof system for Heyting arithmetic, using
said system to illustrate some wider aspects of cyclic proof theory. We call
this system \emph{simple} cyclic Heyting arithmetic because some of its
combinatorial aspects are simpler than those of the cyclic proof systems for
Heyting arithmetic we introduce in \Cref{sec:more-cas}.

Heyting arithmetic employs the language of first-order arithmetic,
which is given below. We fix a countable set of variables $\var$ and write
$\FV(\varphi)$ for the set of \emph{free variables} which occur unbound in a
formula $\varphi$.
\begin{align*}
  s,t \in \term \langeq~& x ~|~ 0 ~|~ S s ~|~ s + t ~|~ s \cdot t  & x \in \var \\
  \varphi, \psi \in \form \langeq~& s = t ~|~ \bot ~|~ \varphi \wedge \psi ~|~ \varphi \vee \psi ~|~ \varphi \to \psi ~|~ \forall x.\varphi ~|~ \exists x. \varphi
\end{align*}
Denote by $[t / x]$ the usual \emph{substitution operation}, replacing all free
occurrences of the variable $x$ by the term $t$. On formulas, this is a partial
operation, $\varphi[t / x]$ being undefined if the free variables in $t$ are not
distinct from the bound variables in $\varphi$. Henceforth, writing $\varphi[t /
x]$ will double as an assertion of the resulting formula being defined.

We begin by recalling an inductive proof system for Heyting arithmetic, i.e. one
presented in the common manner of finite derivation trees.

\begin{definition}[Heyting arithmetic]
  The \emph{sequents} of \emph{Heyting arithmetic} are expressions $\Gamma \sdash \delta$ where
  $\Gamma$ is a finite set of formulas and $\delta$ a single formula of first-order arithmetic.
  Write $\Gamma, \varphi$ for
  $\Gamma \cup \{\varphi\}$ and $\Gamma, \Gamma'$ for $\Gamma \cup \Gamma'$.
  The \emph{derivation rules} of $\HA$ comprise of the following choice of standard rules for
  classical first-order logic with equality,
  \begin{mathpar}
    \inference[\RAx]{}{\Gamma{}, \delta{} \sdash{} \delta}

    \inference[$\to$L]{\Gamma{}, \varphi{} \sdash{} \delta{} \quad \Gamma{} \sdash{}
      \psi{}}{\Gamma{}, \varphi{} \to \psi{} \sdash{} \delta}

    \inference[$\to$R]{\Gamma{}, \varphi{} \sdash{} \psi{}}{\Gamma{} \sdash{} \varphi{} \to \psi{}}

    \inference[$\wedge$L]{\Gamma{}, \varphi{}, \psi{} \sdash{} \delta{}}{\Gamma{},
      \varphi{} \wedge{} \psi{} \sdash \delta{}}

    \inference[$\wedge$R]{\Gamma \sdash \varphi \quad \Gamma \sdash \psi}{\Gamma \sdash \varphi \wedge
      \psi}

    \inference[$\vee$L]{\Gamma, \varphi \sdash \delta \quad \Gamma, \psi \sdash \delta}{\Gamma, \varphi \vee{}
      \psi \sdash \delta}

    \inference[$\vee$R]{\Gamma{} \sdash \varphi_i}{\Gamma{} \sdash
      \varphi_1 \vee{} \varphi_2}

    \inference[$\forall$L]{\Gamma, \varphi[t / x] \sdash \delta}{\Gamma, \forall x.\varphi \sdash \delta}

    \inference[$\forall$R]{\Gamma \sdash \varphi \quad x \not\in
      \FV(\Gamma, \forall x. \varphi)}{\Gamma
      \sdash \forall x.\varphi}

    \inference[$\exists$L]{\Gamma, \varphi \sdash \delta \quad x \not\in
      \FV(\Gamma, \exists x. \varphi, \delta)}{\Gamma, \exists x. \varphi \sdash \delta}

    \inference[$\exists$R]{\Gamma \sdash \varphi[t / x]}{\Gamma \sdash \exists x. \varphi}

    \inference[$\bot$L]{}{\Gamma, \bot \sdash \delta}

    \inference[$=$L]{\Gamma[x / y] \sdash \delta[x / y] \quad x,
      y \not\in \FV(s, t)}{\Gamma[s / x, t / y], s = t \sdash \delta[s / x, t
      / y]}

    \inference[$=$R]{}{\Gamma \sdash t = t}
  \end{mathpar}
  with the following two structural rules,
  \begin{mathpar}
    \inference[\RWk]{\Gamma \sdash \delta}{\Gamma, \Gamma' \sdash \delta}
  
    \inference[\RCut]{\Gamma \sdash \varphi \quad \Gamma, \varphi \sdash \delta}{\Gamma \sdash \delta}
  \end{mathpar}
  the following axiomatic sequents which characterise the function symbols of
  first-order arithmetic,
  \begin{align*}
    & \sdash 0 \neq S t & S s = S t & \sdash s = t && \sdash s + 0 = 0 \\
    &\sdash s + St = S(s + t) & & \sdash s \cdot 0 = 0 && \sdash s \cdot St = (s \cdot t) + s
  \end{align*}
  and the axiomatic sequents of \emph{induction}, for any formula $\varphi$,
  \[ \varphi(0), \forall x. \varphi(x) \to \varphi(Sx) \sdash \varphi(s) . \]
  If a
  proof with endsequent $\Gamma \sdash \delta$
  exists then $\Gamma \sdash \delta$ is \emph{provable in Heyting arithmetic},
  denoted by $\HA \vdash \Gamma \sdash \delta$.
\end{definition}

To illustrate the contrast with cyclic proof systems, it is helpful to consider
the representation of proofs in Heyting arithmetic.
Formally, a proof in $\HA$ is
a pair $\Pi = (T, \lambda)$ consisting of a finite tree $T$ and a function
$\lambda : \Node(T) \to \Seq$ labelling the nodes in accordance with the derivation
rules given above.
Considering the conclusions of rules without premises as axiomatic,
if $t \in \Leaf(T)$ then $\lambda(t)$ is
an axiomatic sequent.

The underlying structure of cyclic proofs is somewhat more complicated.
Often, this structure is described simply as `finite graphs'. However,
in practice, a stricter notion consisting of a finite tree with `added cycles'
often proves more useful, retaining common notions, such as `root',
`leaf' and `subtree'.
A \emph{cyclic tree} is a pair $C = (T, \beta)$ consisting of a finite tree $T$ and a
partial function $\beta \colon \Leaf(T) \to \Inner(T)$ mapping some leaves of
$T$ onto inner nodes of $T$. A leaf $b \in \dom(\beta)$ is called a \emph{bud}
and $\beta(b)$ its \emph{companion}.

Commonly, a further restriction is imposed on the cyclic trees underlying cyclic proofs.
A cyclic tree is said to be in \emph{cycle normal form}
if for every bud $b \in \dom(\beta)$ its companion $\beta(b)$ occurs along the
path from the tree's root to $b$. This restriction has various useful consequences.
For example, all buds of an inner node must be located in
the subtree above it, easing recursive proof translations. Furthermore, trees
in cycle normal form possess a well-defined notion of the `inside' of a cycle:
the path from $\beta(t)$ to $t$ for $t \in \dom(\beta)$.
Every cyclic tree can be unfolded into a
cyclic tree in cycle normal form, although this may cause a super-exponential
increase in the number of nodes~\parencite[see][Theorem
6.3.6]{brotherstonSequentCalculusProof2006}.

We proceed by defining a cyclic proof system $\CHA$ for Heyting arithmetic. It eschews
the induction axioms in favour of \emph{cycles} which allow leaves to
be considered closed if they are labelled by a sequent that has previously
appeared on the branch between the root and said node.

\begin{definition}[$\CHA$ pre-proofs]\label{def:pre-proof}
  The sequents of \emph{cyclic Heyting arithmetic} are the same as in regular
  Heyting arithmetic.
  A \emph{pre-proof} of $\CHA$ is a pair $\Pi = (C, \lambda)$
  consisting of a cyclic tree $C = (T, \beta)$
  and a labelling function $\lambda : \Node(T) \to \Seq$.
  The labelling function $\lambda$ labels the nodes of $T$ with sequents such that
  \begin{itemize}
  \item
    inner nodes are labelled according to the derivation rules of $\HA$ or the $\CHA$-specific derivation rules
    \[
      \inference[$\RCase_x$]{\Gamma[0 / x] \sdash
        \delta[0 / x] \quad \Gamma[Sx / x] \sdash \delta[Sx / x]}{\Gamma \sdash
        \delta}
    \]
    \[
      \inference[$\RSubst$]{\Gamma \sdash \delta}{\Gamma[t / x] \sdash \delta[t / x]}
    \]
    in which $x \not\in \FV(\Gamma, \delta)$ for $\RCase_x$ and $x \not\in
    \FV(t)$ for $\RSubst$.
  \item
    leaves, with the exception of buds, must be labelled with $\HA$ axioms,
    excluding the induction axioms
  \item each bud and companion share labels, i.e. $\lambda(s) = \lambda(\beta(s))$ for $s \in \dom(\beta)$.
  \end{itemize}
\end{definition}

For an example of a $\CHA$ pre-proof, consider \Cref{fig:ind}. It concludes the
induction scheme of $\HA$. We employ the shorthand $\ol{\varphi} \coloneq
\varphi(0), \forall x. \varphi(x) \to \varphi(Sx)$ for the left-hand side of the
induction scheme. The two sequents marked with a $\star$ form a cycle.

\begin{figure}[h]
  \centering

  \begin{comfproof}
    \AXC{}
    \LSC{\textsc{Ax}}
    \UIC{$\ol{\varphi}, \varphi(0) \sdash \varphi(0)$}
    \AXC{$\ol{\varphi} \sdash \varphi(x) ~~\star$}
    \LSC{\textsc{Wk}}
    \UIC{$\ol{\varphi} \sdash \varphi(x), \varphi(S\,x)$}
    \AXC{}
    \RSC{\textsc{Ax}}
    \UIC{$\ol{\varphi}, \varphi(S\,x) \sdash \varphi(S\,x)$}
    \RSC{$\forall$L,$\to$L}
    \BIC{$\ol{\varphi}, \forall x. \varphi(x) \to \varphi(Sx) \sdash \varphi(S\,x)$}
    \insertBetweenHyps{\hskip 4pt}
    \LSC{$\textsc{Case}_x$}
    \BIC{$\ol{\varphi} \sdash \varphi(x) ~~\star$}
    \LSC{$\RSubst$}
    \UIC{$\ol{\varphi} \sdash \varphi(s)$}
  \end{comfproof}

  \caption{A $\CHA$ pre-proof of the induction axiom, $\star$ marking a cycle}
  \label{fig:ind}
\end{figure}

However, not every $\CHA$ pre-proof is a sound argument. For a counterexample,
consider \Cref{fig:unsound}, in which an `argument' concluding the inconsistency
$\sdash \bot$ is made.

\begin{figure}[h]
  \centering

  \begin{comfproof}
    \AXC{}
    \LSC{$\bot$L}
    \UIC{$\bot \sdash \bot$}
    \AXC{$\sdash \bot ~~\star$}
    \LSC{$\RCut$}
    \BIC{$\sdash \bot~~\star$}
  \end{comfproof}

  \caption{An unsound $\CHA$ pre-proof, $\star$ marking a cycle}
  \label{fig:unsound}
\end{figure}

To ensure soundness of $\CHA$ proofs, an additional condition beyond
well-formedness, called a \emph{soundness condition}, must be imposed. The
soundness condition given in \Cref{def:cpa-gtc} is called a \emph{global
trace condition}. We consider alternative soundness conditions for cyclic Heyting
arithmetic in \Cref{chap:representation}.

\begin{definition}[$\CHA$ proofs]\label{def:cha-gtc}
  Let $\Pi = (C, \lambda)$ be a pre-proof. An \emph{infinite branch} through $\Pi$
  is an infinite sequence $t \in \Node(T)^\omega$ with $t_0$ being the root node
  of $T$ and such
  that for each $i \in \omega$ either $t_{i + 1} \in \Chld(t_i)$ or $t_i \in
  \dom(\beta)$ and $t_{i + 1} = \beta(t_i)$. This induces an infinite sequence
  $(\Gamma_i \sdash \delta_i)_{i \in \omega}$ of sequents with $\lambda(t_i) = \Gamma_i \sdash
  \Delta_i $, which we use interchangeably with $(t_i)_{i
    \in \omega}$ to denote an infinite branch. A variable $x$ is said to have \emph{a
    trace along} $(\Gamma_i \sdash \delta_i)_{i \in \omega}$ if there exists an
   $n \in \omega$ such that $x \in \FV(\Gamma_i, \delta_i)$ for all $i \geq n$.
   Such a trace on $x$ is said to be \emph{progressing} if it passes through
   instances of the $\RCase_x$-rule (for the same variable \( x \)) infinitely often.

  A pre-proof $\Pi$ is a \emph{proof} in cyclic Heyting arithmetic if for every
  infinite branch through $\Pi$ there exists a variable which has a progressing trace along
  it. $\CHA \vdash \Gamma \sdash \delta$ denotes provability in $\CHA$.
\end{definition}

Observe that the pre-proof of $\sdash \bot$ in \Cref{fig:unsound} is no $\CHA$
proof because it does not contain any free variables. The pre-proof of the
induction scheme in \Cref{fig:ind} is a proper proof: It only has one infinite
branch, obtained by following the $\star$-cycle indefinitely. This branch has an
$x$-trace, starting above the $\RSubst$-application, which passes through the
$\RCase_x$-rule infinitely often.

% Proofs as programs
From the perspective of proof-as-programs, cyclic proofs correspond to
recursively defined functions, the soundness conditions corresponding to
termination conditions. This connection has been used to bring notions and
results from the field of program
termination~\parencite{leeSizechangePrincipleProgram2001} to the setting of
cyclic proof
theory~\parencite{nolletPSPACECompletenessThreadCriterion2019,afshariAbstractCyclicProofs2022}.

% Similarity of GTCs/TCs
The global trace condition is the most common soundness condition in cyclic
proof theory. It usually follows a common template: Along every infinite branch,
a syntactic feature (such as a term, fixed-point quantifier, formula, ...) can
be traced which \emph{progresses} (e.g.~decreases, is unfolded, is principal to
certain derivation rules, ...) infinitely often. Based on this observation,
abstract notions of trace, progress and cyclic proof have been formulated which
allow the study of cyclic proofs from a general
perspective~\parencite{brotherstonSequentCalculusProof2006,afshariAbstractCyclicProofs2022}.
We discuss the latter framework in \Cref{sec:acds}.
It is relied on in \papOne{} to carry out a construction for
many cyclic proof systems at once.

% Other PA GTC-systems
The system $\CHA$ is merely one possible cyclic proof system for
Heyting arithmetic. Even when restricting one's attention to cyclic proof systems
with global trace conditions, there are two other renditions of Heyting arithmetic
as a cyclic proof system in the literature, both of which we present in
\Cref{sec:more-cas}.

The cyclic proof system $\CHA$ can be transformed into a cyclic proof system
$\CPA$ of Heyting arithmetic via the usual broadening of the right-hand side of sequents to a
finite set $\Delta$. The global trace condition, including the employed
notions of branch, trace and progress, remain unchanged.
% The same transformation
% can be carried out for the cyclic proof systems for Peano arithmetic given in
% \textcite{simpsonCyclicArithmeticEquivalent2017} and
% \textcite{berardiEquivalenceInductiveDefinitions2017} with the same result.

The purpose of the global trace condition in \Cref{def:cpa-gtc} is most easily
illustrated by proving the soundness of $\CHA$ proofs over the standard model of
the natural numbers, i.e. the structure $\omega$ of first-order arithmetic whose
domain are the natural numbers and which interprets the function symbols in the
usual manner.

\begin{theorem}[Soundness]\label{lem:cha-sound}
  If $\CHA \vdash \Gamma \sdash \delta$ then $\omega \vDash \Gamma \sdash \delta$.
\end{theorem}
\begin{proof}
  Let $\Pi$ be a $\CHA$ proof of $\Gamma \sdash \delta$ and suppose, towards
  contradiction, that there is an assignment $\rho : \var \to \omega$ such that
  $\rho \not\vDash \Gamma \sdash \delta$. Consider the last $\CHA$-rule applied
  to derive $\Gamma \sdash \delta$ in $\Pi$. There must exist a premise
  $\Gamma_1 \sdash \delta_1$ and an assignment $\rho_1$ such that $\rho_1
  \not\vDash \Gamma_1 \sdash \delta_1$. We treat a few illustrative cases:
  \begin{description}
  \item[$\wedge$R:] Then $\delta = \varphi_0 \wedge \varphi_1$ and the
    premises of the rule are $\Gamma \sdash \varphi$ and $\Gamma \sdash
    \psi$. Fix $\rho_1 \coloneq \rho$. If $\rho \not\vDash \Gamma \sdash \delta$ then $\rho \vDash
    \Gamma$ and $\rho \not\vDash \delta$.
    Thus there must be $i \in \{0, 1\}$ such that $\rho \not\vDash
    \varphi_i$ and thus $\rho \not\vDash \varphi_i$.
  \item[$\RSubst$:] Then $\Gamma = \Gamma'[t / x]$ and $\delta = \delta'[t / x]$
    for $x \not\in \FV(t)$. Furthermore, the premise of the rule is
    $\Gamma' \sdash \delta'$. Fix $\rho_1 \coloneq \rho[x \mapsto
    t^\rho]$, i.e. the environment which assigning the same values to $y \neq x$
    as $\rho$ and the evaluation of $t$ under $\rho$ to $x$. Observe that for
    every term $s$, $(s[t / x])^\rho = s^{\rho_1}$. Thus, $\rho \not\vDash
    \Gamma[t / x] \sdash \delta[t / x]$ entails $\rho_1 \not\vDash \Gamma \sdash
    \delta$.
  \item[$\forall$R:] Then $\delta = \forall x. \varphi$ with $x \not\in
    \FV(\Gamma, \delta)$ and the premise is $\Gamma \sdash \varphi$.
    By $\rho \not\vDash \Gamma \sdash \forall x. \varphi$
    there must exist $n \in
    \omega$ such that $\rho[x \mapsto n] \not\vDash \varphi$. Hence, fixing
    $\rho_1 \coloneq \rho[x \mapsto n]$, obtains $\rho_1 \not\vDash \Gamma
    \sdash \varphi$ because $x \not\in \FV(\Gamma, \delta)$.
  \item[$\forall$L:] Then $\Gamma = \Gamma', \forall x. \varphi$ and the premise
    is $\Gamma, \varphi[t / x] \sdash \delta$. From $\rho \not\vDash \Gamma,
    \forall x. \varphi$ it follows that $\rho \vDash \forall x. \varphi$ and
    thus $\rho[x \mapsto t^\rho] \vDash \varphi$, or equivalently, $\rho \vDash
    \varphi[t / x]$. Hence, fixing $\rho_1 \coloneq \rho$, it follows that
    $\rho_1 \not\vDash \Gamma', \varphi[t / x] \sdash \delta$.
  \item[$=$L:] Then $\Gamma = \Gamma'[s / x, t / y], s = t$ and $\delta =
    \delta'[s / x, t / y]$ with $x, y \not\in \FV(s, t)$ and the premise is $\Gamma'[t / x, s / y]
    \sdash \delta'[t / x, s / y]$. From $\rho \not\vDash \Gamma \sdash \delta$
    it follows that $\rho \vDash s = t$ and hence $s^\rho = t^\rho$. Hence, for
    any term $u$ and $v \in \{x, y\}$, $(u[s / v])^\rho = (u[t / x])^\rho$ and
    thus $\rho \not\vDash \Gamma'[s / x, t / y], s = t \sdash \delta'[s / x, t /
    y]$ entails $\rho \not\vDash \Gamma'[t / x, s / y] \sdash \delta'[t / x, s /
    y]$.
  \item[$\RCase_x$:] Then $x \in \FV(\Gamma, \delta)$. Observe that either
    $\rho(x) = 0$ or $\rho(x) = n + 1$. In the former case, $\rho \not\vDash
    \Gamma[0 / x] \sdash \delta[0 / x]$, in the latter case $\rho[x \mapsto n]
    \not\vDash \Gamma[S x / x] \sdash \delta[S x / x]$.
  \end{description}
  The remaining cases are analogous to those considered above.
  This argument can
  be applied to any non-axiomatic node of $\Pi$.
  The contradicted premises arrived at
  cannot be axioms as they cannot be contradicted in the
  standard model.
  Iterating this
  argument, `following' buds to their companions, yields an infinite branch $(\Gamma_i \sdash \delta_i)_{i \in \omega}$
  through $\Pi$ and a sequence $(\rho_i)_{i \in \omega}$ of contradicting
  assignments such that $\rho_i \not\vDash \Gamma_i \sdash \delta_i$. As $\Pi$ is a proof, there exists a progressing $x$-trace along $(\Gamma_i \sdash
  \delta_i)_{i \in \omega}$ (w.l.o.g.\ we assume it starts at $\Gamma_0 \sdash \delta_0$). Now consider the sequence
  $(\rho_i(x))_{i \in \omega}$ in $\omega$. From the cases above it follows
  that an increase, i.e. $\rho_i(x) < \rho_{i + 1}(x)$, can only occur in the
  cases of $\forall$R and $\exists$L with $x$ being the bound variable. However,
  as $x$ is free in all $\Gamma_i, \delta_i$, this cannot occur along the
  branch. On the
  other hand, decreases, i.e. $\rho_i(x) > \rho_{i + 1}(x)$, take place whenever
  the right-hand side of a $\RCase_{x}$-rule is passed. As this takes place
  infinitely often, $(\rho_i(x))_{i \in \omega}$ is a never increasing,
  infinitely decreasing sequence of natural numbers, which cannot exist.
  Hence, the conclusion of $\Pi$ could not have been invalid to begin with.
\end{proof}

% Illustrates well-foundedness in logic
We have stated that the common strategies to incorporate the
infinite branches of ill-founded proofs into sound arguments was to
introduce well-foundedness or coinduction at the level of the logic. $\CHA$
takes the former approach, ensuring that all infinite branches of a $\CHA$ proof
produce an infinitely decreasing sequence and can thus be disregarded when
considering soundness. Other systems, such as $\mu$-calculi, allow for
coinductive arguments, ensuring that greatest fixed points are unfolded
infinitely along infinite branches

% Other soundness proofs
The proof strategy in \Cref{lem:cpa-sound} is employed
throughout the literature to prove soundness of cyclic proof systems with a
global trace condition. Often, the demonstrated decreasing sequence is a
sequence of ordinals, rather than just natural numbers. \textcite[Section
4.2.1]{brotherstonSequentCalculusProof2006} gives a general account of this
style of soundness proof. \TODO{We give a better proof!} It should be noted the soundness proof
is highly classical: deducing the existence of a contradicted premise requires
variants of the law of the excluded middle and `constructing' the contradicted
branch by iterating the argument requires a variant of the axiom of choice.
A corollary of the equivalences between $\CHA$ and $\HA$ proven in \papTwo{} is
a fully constructive soundness proof of $\CHA$.

\section{Relating Proof Systems}
\label{sec:relating}

This section discusses how cyclic proof systems may be related to other kinds of
proof systems. We continue to work with $\CHA$ as the example. The soundness
result in \Cref{lem:cpa-sound} does not contribute towards proving the
equivalence between the `inductive' proof system $\HA$ and the cyclic proof
system $\CHA$. Instead, this equivalence is witnessed by proof theoretic
translations back and forth. One of the two directions can be proven using the
notions introduced so far.

\begin{theorem}\label{lem:ha-into-cha}
  If $\HA \vdash \Gamma \sdash \delta$ then $\CHA \vdash \Gamma \sdash \delta$.
\end{theorem}
\begin{proof}
  Recall that \Cref{fig:ind} gives $\CHA$ proofs for every instance of the
  induction scheme of $\HA$.
  Now, suppose $\HA \vdash \Gamma \sdash \delta$. A
  $\HA$ proof $\Pi$ differs from a $\CHA$ pre-proof only by the
  fact that some of its leaves may be labelled with instances of the induction
  schemes. Thus, a $\CHA$ pre-proof $\Pi'$ of $\Gamma \sdash \delta$ may be obtained by
  `grafting' copies of the $\CHA$ proof in \Cref{fig:ind} onto each such leaf of
  $\Pi$. As there are no cycles in $\Pi$, every infinite branch of $\Pi'$
  must follow the $\star$-cycle of one of the `grafted on' induction scheme
  proofs and thus has a successful trace. Hence $\Pi'$ is a $\CHA$ proof
  witnessing $\CHA \vdash \Gamma \sdash \delta$.
\end{proof}

Whereas the translation given in \Cref{lem:pa-into-cpa} is rather simple, the
converse direction of the equivalence is much more complicated. It was
simultaneously obtained by
\textcite{simpsonCyclicArithmeticEquivalent2017} and \textcite{berardiEquivalenceInductiveDefinitions2017}. It is also the main
concern of \TODO{\papTwo{}}. We thus contend ourselves with simply stating the result,
referring the reader to \papTwo{} for its proof.

\begin{fact}\label{lem:cha-into-ha}
  If $\CHA \vdash \Gamma \sdash \delta$ then $\HA \vdash \Gamma \sdash \delta$.
\end{fact}

In cyclic proof systems for logics, rather than first-order theories like Heyting
or Peano arithmetic, the equivalence to inductive proof systems is usually
established by proving the soundness and completeness of the cyclic proof system
directly in terms of a canonical semantics of the logic. When comparing the
proofs of soundness and completeness between the inductive and cyclic proof
systems, cyclic proof systems often require more intricate proofs of
soundness (as witnessed by \Cref{lem:cha-sound}) but allow for simpler
completeness proofs. One intuition behind this phenomenon is that ill-founded
proofs (and hence cyclic proofs) are \emph{coinductive} objects whereas
inductive proofs are \emph{inductive} objects. Framed in the language of
category theory, algebras ease
the construction of morphisms `out of them' via recursion
(corresponding to the `direction' of soundness) whereas coalgebras ease the
construction of morphisms `into them' via corecursion (corresponding to the
`direction' of completeness).

For some cyclic proof systems, it is not yet known whether they are equivalent
to the `corresponding' inductive proof system.
\textcite{brotherstonSequentCalculusProof2006} puts forward a family of
inductive ($\mathrm{LKID}$) and cyclic ($\mathrm{CLKID}^\omega$) proof systems
of first-order logic extended with concrete inductive definitions. He
conjectures but does not prove the equivalence of $\mathrm{LKID}$ and
$\mathrm{CLKID}^\omega$, a claim which is now referred to as the
\emph{Brotherston-Simpson conjecture}.
\textcite{berardiEquivalenceInductiveDefinitions2017} give a partial solution to
the conjecture: If the systems feature an inductive natural number predicate
$N\,t$ and the functions of first-order arithmetic, i.e. if the systems contain
Peano arithmetic, then $\mathrm{LKID}$ and $\mathrm{CLKID}^\omega$ prove the
same theorem. This result has been extended to intuitionistic variants of
$\mathrm{LKID}$ and $\mathrm{CLKID}^\omega$ containing Heyting arithmetic in
\parencite{berardiEquivalenceIntuitionisticInductive2017}. Further,
\textcite{tatsutaClassicalSystemMartinLof2019} have shown that for a `weaker'
inductive predicate, the two systems are not equivalent. However, a full
classification of which inductive predicates induce equivalence has not yet been
given.

Cyclic proofs can be viewed as finite representations of ill-founded proofs. In
the setting of first-order arithmetic, such ill-founded proofs are interesting
in their own right.

\begin{definition}[$\CHA$ $\infty$-proofs]\label{def:infty-proofs}
  The \emph{sequents} of $\CHA$ $\infty$-pre-proofs are the same as those of
  Peano arithmetic.
  An \emph{$\infty$-pre-proof} is a pair $\Pi = (T, \lambda)$
  consisting of a (possibly) ill-founded tree $T$
  and a labelling function $\lambda : \Node(T) \to \Seq$.
  The labelling function $\lambda$ labels the nodes of $T$ with sequents such that
  \begin{itemize}
  \item
    inner nodes are labelled according to the derivation rules of $\HA$ or the $\CHA$-specific derivation rule
    \[
      \inference[$\RCase_x$]{\Gamma[0 / x] \sdash
        \delta[0 / x] \quad \Gamma[Sx / x] \sdash \delta[Sx / x]}{\Gamma \sdash \delta}
      \qquad
      \inference[$\RSubst$]{\Gamma \sdash \delta}{\Gamma[t / x] \sdash \delta[t / x]}
    \]
    in which $x \not\in \FV(\Gamma, \delta)$ for $\RCase_x$ and $x \not\in
    \FV(t)$ for $\RSubst$.
  \item
    leaves labelled with $\HA$ axioms,
    excluding the induction axioms.
  \end{itemize}

  A $\CHA$ $\infty$-pre-proof $\Pi$ is a \emph{$\CHA$ $\infty$-proof} if for every infinite branch
  through $\Pi$ there exists a variable which has a progressing trace along it.
  Provability via $\CHA$ $\infty$-proofs is denoted $\CHA \vdash^\infty \Gamma
  \sdash \delta$.
\end{definition}

Every $\CHA$ pre-proof can be `unfolded' into an $\CHA$ $\infty$-pre-proof by
continuously replacing each bud with the subtree above, and including, its companion. Because
the soundness condition imposed on $\CHA$ proofs and $\CHA$ $\infty$-proofs is
`the same', this method transforms $\CHA$ proofs into $\infty$-proofs. The
$\CHA$ proofs corresponds to a class of $\infty$-proofs: the \emph{regular}
$\CHA$ $\infty$-proofs, i.e.\ those with finitely many distinct subproofs. The
notion of $\infty$-proof is rather general and can be considered in the context
of any cyclic proof system. Conversely, the regular proofs of an $\infty$-proof
system always induce a cyclic proof systems.

From Gödel's incompleteness theorems, it follows that $\HA$ (and
thus $\CHA$) is incomplete, i.e.\ that there are statements satisfied by the
standard model $\omega$ of arithmetic which are not provable in $\HA$.
Completeness can be achieved by extending $\HA$ with an infinitary $\omega$-rule which allows
universal statements to be deduced from derivations of the statement for every
numeral $S^n 0$ for $n \in \omega$.
\[
  \inference[$\omega$]{\Gamma \sdash \varphi[0 / x] \quad \Gamma \sdash
    \varphi[S0 / x] \quad \ldots \quad \Gamma \sdash \varphi[S^n 0 / x] \quad \ldots}{\Gamma \sdash \forall x. \varphi}
\]
\textcite[][Theorem 4]{simpsonCyclicArithmeticEquivalent2017} observes that the $\omega$-rule is admissible in
$\infty$-proofs.

\begin{proposition}
  The $\omega$-rule is admissible in $\infty$-proofs.
\end{proposition}
\begin{proof}
  Suppose there were $\infty$-proofs $\Pi_n$ of $\Gamma \sdash \varphi[S^n 0 /
  x]$ for every $n \in \omega$. Consider the $\infty$-pre-proof $\Pi$ below,
  in which w.l.o.g.\ $x \not\in \FV(\Gamma, \forall x.\varphi)$.
  \begin{comfproof}
    \AXC{$\Pi_0$}
    \UIC{$\Gamma \sdash \varphi[0 / x]$}
    \AXC{$\Pi_1$}
    \UIC{$\Gamma \sdash \varphi[S0 / x]$}
    \AXC{$\Pi_n$}
    \UIC{$\Gamma \sdash \varphi[S^n0 / x]$}
    \AXC{}
    \DOC{}
    \UIC{$\Gamma \sdash \varphi[S^{n + 1}x / x]$}
    \RSC{$\RCase_x$}
    \BIC{$\Gamma \sdash \varphi[S^nx / x]$}
    \DOC{}
    \UIC{$\Gamma \sdash \varphi[S^2x / x]$}
    \RSC{$\RCase_x$}
    \BIC{$\Gamma \sdash \varphi[Sx / x]$}
    \LSC{$\RCase_x$}
    \BIC{$\Gamma \sdash \varphi$}
    \LSC{$\forall$R}
    \UIC{$\Gamma \sdash \forall x. \varphi$}
  \end{comfproof}
  It remains to argue that $\Pi$ is a proof. Thus, consider an
  infinite path through it. Such a path must either infinitely pass through the
  right-hand sides of the $\RCase_{x}$-applications or enter one of the $\Pi_n$.
  Per assumption, every branch through the $\Pi_n$ has a progressing trace. As
  the notion of progressing trace is closed under taking prefixes, all
  the branches through $\Pi$ entering one of the $\Pi_n$ exhibit one. The branch passing
  infinitely many right-hand sides of $\RCase_{x}$ clearly has a progressing
  $x$-trace. Hence $\Pi$ is a proof.
\end{proof}
\begin{corollary}[Completeness]
  If $\omega \vDash \Gamma \sdash \delta$ then $\CHA \vdash^\infty \Gamma \sdash
  \delta$.
\end{corollary}

For the cyclic proofs for first-order logic with inductive definitions considered by
\textcite{brotherstonSequentCalculusProof2006}, a similar mismatch is exhibited,
the $\infty$-proofs being complete for standard second-order models.
However, this disagreement between cyclic and $\infty$-proofs is not universal.
For example, in modal logics like the modal
$\mu$-calculus~\parencite{niwinskiGamesMcalculus1996}, Gödel-Löb
logic~\parencite{shamkanovCircularProofsGodelLob2014} and Grzegorczyk
logic~\parencite{savateevNonWellFoundedProofsGrzegorczyk2018}, cyclic proofs and
$\infty$-proofs prove exactly the same theorems. More generally, cyclic and
$\infty$-proof systems in which only finitely many distinct
sequents can occur in a proof are always equivalent \parencite[see][Theorem
4.6]{wehrAbstractFrameworkAnalysis2021}.
This result applies to many proof systems which exhibit a subformula property,
such as the modal logics listed above.

Returning to the setting of first-order arithmetic, Heyting and true arithmetic
corresponding to regular and unrestricted $\infty$-proofs, respectively,
naturally raises the question whether correspondences between other classes of
$\infty$-proofs and fragments of first-order arithmetic can be found. An
interesting extension of regular trees are the classes of hyper-regular
trees generated by higher-order recursion schemes as considered
in~\textcite{ongModelCheckingTreesGenerated2006}. The article proves that
acceptance of $\omega$-regular conditions, such as the global trace condition,
on such hyper-regular trees is decidable, extending the property of
decidable proof checking to hyper-regular $\infty$-proofs. However, the trees
considered may only employ a finite supply of labels, corresponding to the case
of proof systems with finitely many sequents discussed in the previous
paragraph. Hence, higher-order recursion system generated $\infty$-proofs (for
virtually every cyclic proof system in the literature) prove the same theorems
as cyclic proofs. This is the only other such result we are aware of.

\section{More Cyclic Arithmetics}
\label{sec:more-cas}

\textcite{simpsonCyclicArithmeticEquivalent2017} presents a proof system over
the language of first-order arithmetic extended by a relation $s < t$. Traces in
his system are sequences of terms appearing along the branch which are, at each
`step', either equal or ordered by a formula $s < t$ to the left of $\sdash$.
Each $<$-step is considered a progression.
\textcite{berardiEquivalenceInductiveDefinitions2017} consider a system of
first-order arithmetic extended by inductively defined predicates, including a
one-place predicate $N\,t$ asserting that $t$ denotes a natural number.
Traces are sequences of inductively defined predicates appearing
to the left of $\sdash$. Progress takes place whenever a case-distinction rule
is applied to the traced predicate, the effect of a case-distinction on $N\,t$
being analogous to that of the $\RCase_{x}$-rule of \Cref{def:pre-proof}.
An important difference between our notion of trace and those employed in the two
aforementioned articles is that it does not `respect' the $\RSubst$ rule, i.e.\
if an $x$-trace passing through a $\RSubst$-instance which replaces $x$ by a
fresh $y$ does not `switch focus' to $y$ but is simply terminated.


\subsection{Simpson-style Cyclic Heyting Arithmetic}
\label{sec:simpson-cha}

The term and formula languages of $\CHAl$ are given below. The formula language is
non-standard, treating inequality $s < t$ as a primitive, rather than a defined
notion. As will become clear below, this eases the definition of the global
trace condition of $\CHAl$.
\begin{align*}
  s,t \in \term \langeq~& x ~|~ 0 ~|~ S s ~|~ s + t ~|~ s \cdot t \\
  \varphi, \psi \in \form \langeq~& s = t ~|~ s < t ~|~ \bot ~|~ \varphi \wedge \psi ~|~ \varphi \vee \psi ~|~ \varphi \to \psi ~|~ \forall x.\varphi ~|~ \exists x. \varphi
\end{align*}
We still denote by $[t / x]$ the \emph{substitution operation} described in
\Cref{sec:cha}.

\begin{definition}\label{def:chal-rules}
  The \emph{sequents} of $\CHAl$ are expressions $\Gamma \sdash \delta$ where
  $\Gamma$ is a finite sets of formulas and $\delta$ a single formula.
  Write $\Gamma, \varphi$ for
  $\Gamma \cup \{\varphi\}$ and $\Gamma, \Gamma'$ for $\Gamma \cup \Gamma'$.
  The \emph{derivation rules} of $\CHAl$ comprise of the following choice of standard rules for
  intuitionistic first-order logic,
  \begin{mathpar}
    \inference[\RAx]{}{\Gamma{}, \delta{} \sdash{} \delta}

    \inference[$\to$L]{\Gamma{}, \varphi{} \sdash{} \delta{} \quad \Gamma{} \sdash{}
      \psi{}}{\Gamma{}, \varphi{} \to \psi{} \sdash{} \delta}

    \inference[$\to$R]{\Gamma{}, \varphi{} \sdash{} \psi{}}{\Gamma{} \sdash{} \varphi{} \to \psi{}}

    \inference[$\wedge$L]{\Gamma{}, \varphi{}, \psi{} \sdash{} \delta{}}{\Gamma{},
      \varphi{} \wedge{} \psi{} \sdash \delta{}}

    \inference[$\wedge$R]{\Gamma \sdash \varphi \quad \Gamma \sdash \psi}{\Gamma \sdash \varphi \wedge
      \psi}

    \inference[$\vee$L]{\Gamma, \varphi \sdash \delta \quad \Gamma, \psi \sdash \delta}{\Gamma, \varphi \vee{}
      \psi \sdash \delta}

    \inference[$\vee$R]{\Gamma{} \sdash \varphi_i}{\Gamma{} \sdash
      \varphi_1 \vee{} \varphi_2}

    \inference[$\forall$L]{\Gamma, \varphi[t / x] \sdash \delta}{\Gamma, \forall x.\varphi \sdash \delta}

    \inference[$\forall$R]{\Gamma \sdash \varphi \quad x \not\in
      \FV(\Gamma, \forall x. \varphi)}{\Gamma
      \sdash \forall x.\varphi}

    \inference[$\exists$L]{\Gamma, \varphi \sdash \delta \quad x \not\in
      \FV(\Gamma, \exists x. \varphi, \delta)}{\Gamma, \exists x. \varphi \sdash \delta}

    \inference[$\exists$R]{\Gamma \sdash \varphi[t / x]}{\Gamma \sdash \exists x. \varphi}

    \inference[$\bot$L]{}{\Gamma, \bot \sdash \delta}

    \inference[$=$L]{\Gamma[t / x, s / y] \sdash
      \delta[t / x, s / y] \quad x, y \not\in \FV(s, t)}{\Gamma[s / x, t / y], s = t \sdash \delta[s / x, t
      / y]}

    \inference[$=$R]{}{\Gamma \sdash t = t}
  \end{mathpar}
  with the following structural rules,
  \begin{mathpar}
    \inference[\RWk]{\Gamma \sdash \delta}{\Gamma, \Gamma' \sdash \delta}

    \inference[\RCut]{\Gamma \sdash \varphi \quad \Gamma, \varphi \sdash \delta}{\Gamma \sdash \delta}
    
    \inference[\RSub]{\Gamma \sdash \delta}{\Gamma[s / x] \sdash \delta[s / x]}
  \end{mathpar}
  with $x \not\in
    \FV(t)$ for $\RSubst$ and
  % \begin{mathpar}
  %   \inference[\RWk]{\Gamma \sdash \Delta}{\Gamma, \Gamma' \sdash \Delta, \Delta}
  %
  %   \inference[\RCut]{\Gamma, \varphi \sdash \Delta \quad \Gamma \sdash \varphi,
  %     \Delta}{\Gamma \sdash \Delta}
  %
  %   \inference[\RSub]{\Gamma \sdash \Delta}{\Gamma[s / x] \sdash \Delta[s / x]}
  %
  %   \inference[Repeat]{[\Gamma \sdash \Delta] \\ \vdots \\ \Gamma \sdash \Delta}{\Gamma \sdash \Delta}
  % \end{mathpar}
  the following arithmetic-specific axioms%
  \begin{align*}
    s < t, t < u & \sdash s < u &
    s < t & \sdash Ss < St \\
    & \sdash s + St = S(s + t) &
    s < t, t < s & \sdash \\
    & \sdash s < t \vee s = t \vee t < s &
    & \sdash t \cdot 0 = 0 \\
    s < t, t < Ss & \sdash &
    & \sdash t < St \\
    & \sdash s \cdot St = (s \cdot t) + s &
    t < 0 & \sdash \\
    & \sdash t + 0 = t
  \end{align*}
  and the arithmetic-specific derivation rule%
  \begin{mathpar}
    \inference[S]{\Gamma, t = Sx \sdash \delta \quad x \text{ fresh}}{\Gamma, 0 < t \sdash \delta}
  \end{mathpar}
  $ $
\end{definition}


\begin{definition}[Trace]
  A term $t$ \emph{occurs} in a sequent $\Gamma \sdash
  \delta$ if it appears, possibly as a subterm of another term, in a formula in
  $\Gamma$ or $\delta$. Write $\term(\Gamma \sdash \delta)$ for the set of
  terms occurring in $\Gamma \sdash \delta$.

  Let $R$ be a rule of $\CHAl$ of the form
  \[
    \inference[R]{\Gamma_1 \sdash \delta_1 \qquad \ldots \qquad \Gamma_n \sdash \delta_n}{\Gamma \sdash \delta}
  \]
  Fix $t \in \term(\Gamma \sdash \delta)$ and $t' \in \term(\Gamma_i \sdash
  \delta_i)$. The term $t'$ is called a
  \emph{precursor} of $t$, denoted $t' \leftarrow^i_R t$ if one of the following three
  conditions holds:
  \begin{itemize}
  \item $R$ is an instance of $(\textsc{Sub})$ and $\Gamma = \Gamma'[s / x], \delta =
    \delta'[s / x]$ and $t = t'[s / x]$;
  \item or $R$ is an instance of a rule \emph{other than} $(\RSub)$ and $t = t'$;
  \item or \( R \) is an instance of \( (=\hspace{-0.25em}\text{L})\) and $\Gamma = \Gamma''[s / x,
    t / y], \Gamma' = \Gamma''[t / x, s / y]$ and analogously for the $\delta$
    and there exists a term $t''$ such that $t = t''[s / x, t / y]$ and $t' =
    t''[t / x, s / y]$.
  \end{itemize}

  Now let $\Pi$ be $\CHAl$ pre-proof and
  let $(\Gamma_i \sdash \delta_i)_{i \in \omega}$ be an infinite branch of
  $\Pi$. A trace along said branch is a sequence of terms $(t_i)_{i \in \omega}$
  and an offset $N$ such that for every $i \in \omega$ we have $t_i \in \term(\Gamma_{N + i} \sdash \delta_{N +
    i})$ and furthermore, if $\Gamma_{N + i + 1} \sdash \delta_{N + i + 1}$ is
  the $j$th premise of $\CHAl$-rule $R$:
  \begin{enumerate}[label=(\roman*)]
  \item either $t_{i + 1} \leftarrow^j_R t_i$
  \item or $s < t_i \in \Gamma$ and $t_{i + 1} \leftarrow^j_R t_i$.
  \end{enumerate}
  A trace is said to be progressing if case (ii) applies infinitely often along it.
\end{definition}

\begin{definition}[$\CHAl$-proof]
  A $\CHAl$ pre-proof $\Pi$ is a proof if every infinite branch of $\Pi$ has a
  progressing trace. We write $\CHAl \vdash \Gamma \sdash \delta$ to mean that
  there exists a $\CHAl$-proof $\Pi$ with endsequent $\Gamma \sdash \delta$.
\end{definition}

\subsection{Brotherston-Simpson Cyclic Arithmetic}
\label{sec:brotherston-simpson}

\subsection{Translating between Cyclic Arithmetics}
\label{sec:translating-cas}

\section{Complexity results}
\label{sec:cha-complexity}
