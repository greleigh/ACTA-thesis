% Emacs, this is -*-latex-*-

\chapter{Introduction}
\label{chap:intro}

In cyclic proof systems, derivations are finite graphs annotated with sequents,
rather than the common annotated finite trees. Cyclic proofs may thus present
circular arguments.
This calls to mind the logical fallacy of \emph{circular reasoning}.
Indeed, unfolding a cyclic proof yields an ill-founded proof tree, corresponding
to infinite arguments.
Soundness along infinite branches of such ill-founded proofs,
and by extension cyclic proofs, cannot be reduced to truth of the system's
axioms and truth-preservation of its deduction rules.
The common solutions to this conundrum are to introduce well-foundedness at the
level of the system's logic, rather than the level of derivation structure, or
to allow infinite branches to represent coinductive arguments.
As long as the proof system reasons about (co-)inductive notions, ill-founded
derivations may thus still represent sound arguments. There are various features
of logical systems, both logics and theories, on which
cyclic proof systems may be based:
\begin{itemize}
\item \emph{a domain of (co-)inductive objects:} Such systems may, for example,
  be arithmetical
  systems~\parencite{berardiEquivalenceIntuitionisticInductive2017}, type systems
  featuring a natural number type~\parencite{dasCircularVersionGodel2021} or
  theories considering inductively defined predicates~\parencite{brotherstonSequentCalculusProof2006}.
\item \emph{fixed-point constructions:} The primary type of such
  systems are $\mu$-calculi with explicit fixed-point quantifiers for least and
  greatest fixed points, such as the modal
  $\mu$-calculus~\parencite{niwinskiGamesMcalculus1996}, first-order
  $\mu$-calculus~\parencite{sprengerGlobalInductionMechanisms2003}, higher-order
  fixed point logic~\parencite{koriCyclicProofSystem2021}, linear time
  $\mu$-calculus~\parencite{daxProofSystemLinear2006} or linear logic with
  fixed points~\parencite{baeldeInfinitaryProofTheory2016}. The suitability often
  extends to subsystems of
  the aforementioned $\mu$-calculi, such as
  the alternation-free modal
  $\mu$-calculus~\parencite{martiFocusSystemAlternationFree2021}, propositional
  dynamic~\parencite{dochertyNonwellfoundedLabelledProof2019} or
  computational tree logic~\parencite{afshariCyclicProofSystem2023}.
  A further case are systems reasoning about the inclusion relations on
  regular~\parencite{dasCutFreeCyclicProof2017} and
  $\omega$-regular~\parencite{hazardCyclicProofsTransfinite2022} languages,
  the expressions $e^*$ and $e^\omega$ expressing a least and greatest
  fixed point, respectively.
\item \emph{well-founded relations:} The primary examples of
  such systems take a well-founded relation as a primitive,
  such as the $<$-ordering on natural
  numbers~\parencite{simpsonCyclicArithmeticEquivalent2017} or on
  ordinals~\parencite{sprengerGlobalInductionMechanisms2003}. This category also includes modal
  logics, such as Gödel-Löb
  logic~\parencite{shamkanovCircularProofsGodelLob2014} and Grzegorczyk
  logic~\parencite{savateevNonWellFoundedProofsGrzegorczyk2018}, defined over
  well-founded frames.
\end{itemize}

Even in cyclic proof systems for suitable logics or theories, as those described
above, well-formedness according to the derivation rules is not sufficient to
guarantee soundness. Instead, soundness is ensured by imposing further
conditions on cyclic derivations. Such soundness conditions are discussed in
great detail in \Cref{chap:representation}.

Proof systems for the logics and theories characterised above whose proofs are
finite trees always feature induction rules or axioms. We shall refer to such
proof systems as \emph{inductive}. The derivation rules of cyclic proof systems
are often identical to those of an inductive proof system for the same logic,
except for the induction rules, which the cyclic proof system either completely
omits or at least greatly simplifies. There are two applications of proof
systems which can benefit from such simplifications: \emph{automatic theorem
proving} and \emph{proof theoretic investigations}.

The field of automatic theorem proving aims to develop algorithms which can
prove theorems without human interaction. A major difficulty in devising such
algorithms is determining at which point in a proof an induction rule should be
applied. Even for human provers, finding suitable induction invariants to prove
a claim can be difficult. Encoding the required `human creativity' into an
algorithm naturally poses a difficult problem. This problem can be circumvented
completely by developing automated theorem provers in terms of a cyclic proof
system without induction rules. Indeed, multiple experimental theorem provers following
this approach have already been developed
\parencite[see][]{brotherstonAutomatedCyclicEntailment2011,brotherstonGenericCyclicTheorem2012,tellezAutomaticallyVerifyingTemporal2017}.

Another application of cyclic proof systems is carrying out proof theoretic investigations
of logics and theories with features as described above. Induction axioms are
often non-logical rules, not directly tied to logical connectives, and can thus
`disturb' the `homogeneity' of a proof system. This, for example, is exhibited
by the fact that $\RCut$s on induction axioms can not be eliminated in some theories. Cyclic proof systems, which
forgo induction axioms in favour of cycles, can be better suited to certain kinds
of proof theoretic applications. So far, cyclic proof systems have proven especially
useful to deriving interpolation
results~\parencite[see][]{shamkanovCircularProofsGodelLob2014,savateevNonWellFoundedProofsGrzegorczyk2018,martiFocusSystemAlternationFree2021,afshariUniformInterpolationCyclic2021,afshariLyndonInterpolationModal2022}.
They have also been used to derive results about non-cyclic proof systems, such
as the $\RCut$-free completeness of an axiomatisation of the modal
$\mu$-calculus~\parencite{afshariFinitaryProofSystems2016}.

Lastly, cyclic proof systems are of interest as objects of proof theoretic study
in their own right. This is the perspective assumed in this thesis. Cyclic proof
theory is a rather young field, arguably originating in the works of
\textcite{niwinskiGamesMcalculus1996} and
\textcite{sprengerGlobalInductionMechanisms2003}. Some of the most basic
questions, such as general strategies for $\RCut$-elimination in cyclic proofs,
remain open.


% LocalWords:  axiomatisation sequents foundedness coinductive provers Peano
% LocalWords:  invariants endsequent subtree pre formedness Heyting GTC Gödel
% LocalWords:  morphisms intuitionistic subproofs infinitary Löb Grzegorczyk
% LocalWords:  decidability uncountably Ackermann PSPACE coNP derivedness iff
% LocalWords:  subgraph subgraphs subsequence subsequences combinatorial
% LocalWords:  interpolants interpolant subproof homomorphism functors subpath
% LocalWords:  semilattice arithmetics Sprenger formalisations
