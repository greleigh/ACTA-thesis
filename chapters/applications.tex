% Emacs, this is -*-latex-*-

\chapter{Applications}
\label{chap:applications}

\section{Consistency of Cyclic Peano Arithmetic}
\label{sec:cons-of-cha}

\subsection{Ordinal-bounded stack-controlled Peano arithmetic}
\label{sec:ord-spa}

In this section, we consider a (conservative) extension of Peano Arithmetic with
bounded quantification. That is, the syntax we employ is given below.
\begin{align*}
  s,t \in \term \langeq &~ x ~|~ 0 ~|~ S s ~|~ s + t ~|~ s \cdot t  & x \in \var \\
  \varphi, \psi \in \form \langeq &~ s = t ~|~ \bot ~|~ \varphi \to \psi ~|~ \forall x.\varphi ~|~
                \forall x < s. \varphi ~|~ \forall x \leq s. \varphi & n \in \omega
\end{align*}

The sequents we consider are of the form $\Lambda \vdash^\alpha_k \Gamma \sdash
\Delta$ where $\Gamma \sdash \Delta$ is a $\PA$-sequent in the language above
with $\Gamma, \Delta$ \emph{lists of formulas},
 $\alpha$ is a \emph{height bound} in the
usual ordinal notation system for $\varepsilon_0$ and $k < \omega$ is a
\emph{$\RCut$-rank}. $\Lambda$ is a list of jump-labels $x^\sigma \mapsto \Gamma
\sdash \Delta$, with the same restrictions as in \needcite{}, except $\Gamma,
\Delta$ additionally may not contain $\forall x < s. \varphi$ or $\forall x \leq
s. \varphi$ as subformulas.

We begin by specifying the rules of $\APA$ \TODO{Fix system name} which are
essentially taken from \parencite{mancosuIntroductionProofTheory2021}.
Deviating from them, we introduce an explicit $\RRaise$-rule
for raising $\RCut$-ranks. This does not impact the arguments but simplifies
their presentation.
Below, $\alpha + \beta$ denotes the usual commutative addition operator defined on the
ordinal notation system for $\varepsilon_0$ and $s(\varphi)$ is the usual
measure of formula size.

  \begin{description}
	\item [Logical rules] The following rules which do not interact with companion labels:
  \begin{mathpar}
    \inference[$\to$L]{\Lambda \vdash^{\alpha}_{k} \Gamma{}, \varphi \sdash \Delta \quad \Lambda \vdash^{\beta}_{k} \Pi \sdash \psi, \Sigma}{\Lambda \vdash^{\alpha + \beta}_{k} \Gamma, \Pi, \varphi \to \psi \sdash \Delta, \Sigma}

    \inference[$\to$R]{\Lambda \vdash^{\alpha}_{k} \Gamma, \varphi \sdash \psi, \Delta}{\Lambda \vdash^{\alpha + 1}_{k} \Gamma \sdash \varphi \to \psi, \Delta}


    \inference[$\forall$L]{\Lambda \vdash^{\alpha}_{k} \Gamma, \varphi[t / x] \sdash \Delta}{\Lambda \vdash^{\alpha + 1}_{k} \Gamma, \forall x.\varphi \sdash \Delta}

    \inference[$\forall$R]{\Lambda \vdash^{\alpha}_{k} \Gamma \sdash \varphi[y/x], \Delta \quad y \not\in \FV(\Gamma, \forall x. \varphi, \Delta)}{\Lambda \vdash^{\alpha + 1}_{k} \Gamma \sdash \forall x.\varphi, \Delta}

    \inference[$=$L]{\Lambda \vdash^{\alpha}_{k} \Gamma[t / x, s / y] \sdash \Delta[t / x, s / y] \quad x, y \not\in \FV(s, t)}{\Lambda \vdash^{\alpha + 1}_{k} \Gamma[s / x, t / y], s = t \sdash \Delta[s / x, t / y]}

    \inference[$=$R]{}{\Lambda \vdash^{1}_{k} \; \sdash t = t}

	\inference[$\bot$L]{}{\Lambda \vdash^{1}_{k} \bot \sdash }

  \inference[\RWk]{\Lambda \vdash^{\alpha}_{k} \Gamma \sdash \Delta}{\Lambda \vdash^{\alpha}_{k} \Gamma, \Gamma' \sdash \Delta, \Delta'}

  \inference[\RLEx]{\Lambda \vdash^{\alpha}_{k} \Gamma, \psi, \varphi, \Gamma'
    \sdash \Delta}{\Lambda \vdash^{\alpha}_{k} \Gamma, \varphi, \psi, \Gamma' \sdash \Delta}

  \inference[\RREx]{\Lambda \vdash^{\alpha}_{k} \Gamma \sdash \Delta, \psi,
    \varphi, \Delta'}{\Lambda \vdash^{\alpha}_{k} \Gamma \sdash \Delta, \varphi, \psi, \Delta'}

  \inference[\RLCtr]{\Lambda \vdash^{\alpha}_{k} \Gamma, \varphi, \varphi \sdash \Delta}{\Lambda \vdash^{\alpha}_{k} \Gamma, \varphi \sdash \Delta}

  \inference[\RRCtr]{\Lambda \vdash^{\alpha}_{k} \Gamma \sdash \varphi, \varphi, \Delta}{\Lambda \vdash^{\alpha}_{k} \Gamma \sdash \varphi, \Delta}

  \inference[\RCut]{s(\varphi) < k \\ \Lambda \vdash^{\alpha}_{k} \Gamma \sdash \varphi, \Delta \quad
    \Lambda \vdash^{\beta}_{k} \Pi, \varphi \sdash \Sigma}{\Lambda \vdash^{\alpha + \beta}_{k} \Gamma, \Pi \sdash \Delta, \Sigma}

  \inference[$\RRaise$]{\Lambda \vdash^{\alpha}_{k + 1} \Gamma \sdash \Delta}{\Lambda \vdash^{\omega^\alpha}_{k} \Gamma \sdash \Delta}
  \end{mathpar}
  where we impose another restriction on $=$L: $\Gamma, \Delta$ may not contain
  a subformula in which $x$ or $y$ occur in the upper bound of bounded
  quantification. That is, if $\forall z < u.~\varphi$ or $\forall z \leq
  u.~\varphi$ is a subformula of $\Gamma, \Delta$ then $\FV(u) \cap \{x, y\} = \emptyset$.
  \item [Arithmetical axioms] These are of the form: 
  \begin{align*}
    \varepsilon \vdash^{1}_{k} \;  0 = S t \sdash ~ & & \varepsilon \vdash^{1}_{k} S s = S t \sdash s = t && \varepsilon \vdash^{1}_{k} \; \sdash s + 0 = 0 \\
    \varepsilon \vdash^{1}_{k} \; \sdash s + St = S(s + t) && \varepsilon \vdash^{1}_{k} \; \sdash s \cdot 0 = 0 &&\varepsilon \vdash^{1}_{k} \; \sdash s \cdot St = (s \cdot t) + s
  \end{align*}
\end{description}

Let $<^{-} \coloneq <$ and $<^{+} \coloneq \leq$. We write $\forall \vec{x}
<^{\vec{\sigma}} \vec{t}$ as a shorthand for a sequence of bounded quantifications $\forall x_0 <^{\sigma_0} t_0
\ldots \forall x_i <^{\sigma_i} t_i \ldots$. Consider two such sequences
$\forall \vec{x} <^{\vec{\sigma}} \vec{t}$ and $\forall \vec{x} <^{\vec{\sigma'}}
\vec{s}$ such that the $\vec{t}$ and $\vec{s}$ are numerals, i.e. $t_i = \ol{n_i}$
and $s_i = \ol{m_i}$ with $n_i, m_i \in \omega$. We say that $\vec{\sigma'} ;
\vec{s}$ is a shifting of $\vec{\sigma} ; \vec{t}$, writing $\vec{\sigma'} ;
\vec{s} \preceq \vec{\sigma} ; \vec{t}$ if for ever $i < \abs{x}$:
\begin{itemize}
\item either $\sigma'_i = +$ and $\sigma_i = -$ and $m_i < n_i$,
\item or $\sigma'_i = -$ or $\sigma_i = +$ and $m_i \leq n_i$.
\end{itemize}
Observe that these are exactly the sound `shifts' of bounds.
We now add the following rules for the bounded quantifiers. Note that these are
chosen to provide a complete axiomatization of bounded quantification but rather
as exactly those auxiliary rules which we need for our argument.
\begin{mathpar}
  \inference[$\Ale$R]{\varepsilon \vdash^{\alpha_0}_k \Gamma \sdash A(0) \quad
    \ldots \quad \varepsilon \vdash^{\alpha_{n - 1}}_k \Gamma \sdash
    A(\ol{n - 1})}{\varepsilon \vdash^{\Sigma \alpha_i}_k\Gamma \sdash \forall x < \ol{n}.~A(x)}

  \inference[$\Aleq$R]{\varepsilon \vdash^{\alpha_0}_k \Gamma \sdash A(0) \quad
    \ldots \quad \varepsilon \vdash^{\alpha_{n}}_k \Gamma \sdash
    A(\ol{n})}{\varepsilon \vdash^{\Sigma \alpha_i}_k\Gamma \sdash \forall x \leq \ol{n}.~A(x)}

  \inference[$\Aleq$L]{\varepsilon \vdash^\alpha_k \Gamma,
    A(\ol{n}) \sdash B}{\varepsilon \vdash^{\alpha + 1}_k \Gamma, \forall x \leq \ol{n}.~A(x) \sdash B}

  \inference[$\RShift$]{\varepsilon \vdash^{\alpha}_k \Gamma, \forall \vec{x}
    <{\vec{\sigma'}} \vec{s}.~\varphi \sdash \Delta ~~\text{ where } \vec{\sigma'};\vec{s}
    \preceq \vec{\sigma}; \vec{t}}{\varepsilon \vdash^{\alpha}_k \Gamma, \forall \vec{x} <{\vec{\sigma}} \vec{t}.~\varphi \sdash \Delta}
\end{mathpar}

\TODO{What about $\RBin$ and $\Lambda$?}

We call a substitution $\theta : V \to \term$ with $V \subseteq \var$
\emph{closed} if $\im(\theta)$ are closed terms. If $\im(\theta) \subseteq
\{\ol{n} \mid n \in \omega \}$ we call $\theta$ an assignment.
We call $\theta$ a \emph{closing} for a sequent $\Gamma \sdash
\delta$ if it is closed and $\dom(\theta) = \FV(\Gamma, \delta)$.

As before, jump-labels are of the form $x^\sigma \mapsto \Gamma \sdash \Delta$
with the usual side-conditions. For a jump-label stack $\Lambda$, we define $\HH_\Lambda \coloneq
H(\emptyset, \emptyset, \Lambda)$ with
\begin{align*}
  H(V, I, \varepsilon) &\coloneq I \\
  H(V, I, (x^\sigma \mapsto \Gamma \sdash \Delta); \Lambda') &\coloneq H(V \cup \{x\}, I \cup \{\varphi\}, \Lambda')
\end{align*}
where $\vec{v} = V \setminus \{x\}$, $\vec{y} = \FV(\Gamma, \delta) \setminus (V
\cup \{x\})$ and $x', \vec{u}$ fresh and $\Phi \to \Psi$ is a shorthand for $\bigwedge
\Phi \to \bigvee \Psi$:
\[
  \varphi \coloneq \forall x' <^\sigma x.\forall \vec{u} \leq \vec{v}. \forall
  \vec{y}. (I, \Gamma \to \Delta)[x' / x, \vec{u} / \vec{v}].
\]
We shall write $\varphi(\Lambda)$ for the last formula $\varphi$ added to
$\HH_\Lambda$.

\begin{lemma}\label{lem:cons-close-buds}
  Let $\Lambda = \Lambda'; x^+ \mapsto \Gamma \sdash \Delta$. There exists a
  $h(\Lambda)$ such that for every assignment $\theta$ with $\dom(\theta) =
  \var(\Lambda)$ there is a proof of $\varepsilon
  \vdash^{h(\Lambda)}_0 \HH_\Lambda[\theta], \Gamma[\theta] \sdash
  \Delta[\theta]$ using the rules specified so far.
\end{lemma}
\begin{proof}
  The derivation can be obtained via
  \begin{comfproof}
    \AXC{}
    \UIC{$\varepsilon \vdash^n_0 (\HH_{\Lambda'}, \Gamma \to \Delta)[\theta], \HH_\Lambda[\theta], \Gamma[\theta] \sdash \Delta[\theta]$}
    \LSC{$\Aleq\text{L}^*$}
    \UIC{$\varepsilon \vdash^{n + \abs{\var(\Lambda)}}_0 \varphi(\Lambda)[\theta], \HH_{\Lambda'}[\theta], \Gamma[\theta] \sdash \Delta[\theta]$}
  \end{comfproof}
  where we note that $\varphi(\Lambda)$ is of the form $\forall \vec{x'} \leq
  \theta(\vec{x}). \forall \vec{y}. \ldots$. Observe that he premise of the
  $\Aleq$L-steps is an instance of the tautological sequent $\Phi \to
  \Psi, \Phi \sdash \Psi$ and thus has a proof of some finite height $n$. Then fix
  $h(\Lambda) \coloneq n + \abs{\Lambda}$.
\end{proof}

We can now specify the height bounds and $\RCut$-ranks of the rules which manipulate
companion labels:
\begin{mathpar}
  \inference[$\RComp$]{s(\varphi(\Lambda; (x^- \mapsto \Gamma \sdash \Delta))) <
    k\\ \Lambda; (x^- \mapsto \Gamma \sdash \Delta) \vdash^{\alpha}_{k} \Gamma \sdash \Delta}{\Lambda \vdash^{\alpha^+}_{k} \Gamma \sdash \Delta}

  \inference[$\RBud$]{}{\Lambda; (x^+ \mapsto \Gamma \sdash \Delta) \vdash^{h(\Lambda; (x^+ \mapsto \Gamma \sdash \Delta))}_{k} \Gamma \sdash \Delta}

  \inference[$\RDrop$]{\Lambda \vdash^{\alpha}_{k} \Gamma \sdash
    \Delta}{\Lambda; \Lambda' \vdash^{\alpha + 1}_{k} \Gamma \sdash \Delta}

  \inference[$\RCase_{x}$]{\Lambda \vdash^{\alpha}_{k} \Gamma(0) \sdash \Delta(0)
    \qquad
    \Lambda^{+x} \vdash^{\beta}_{k} \Gamma(S\,x) \sdash \Delta(S\,x)}{\Lambda \vdash^{\alpha + \beta}_{k} \Gamma(x) \sdash \Delta(x)}
\end{mathpar}

Lastly, the system contains some `bookkeeping' rules, which represent suspended
states of the closing procedure, which is part of the consistency argument. The
final, $\RCut$-free proof which is produced by the argument will contain no
instances of these rules.
\begin{mathpar}
  \inference[$\RSub$]{\varepsilon \vdash_k^{\alpha} \Gamma \sdash \Delta}{\varepsilon \vdash_k^{\alpha + 1} \Gamma[\theta] \sdash \Delta[\theta]}

  \inference[$\RComp'$]{\Lambda; x^- \mapsto \Gamma' \sdash \delta' \vdash^\beta_k \Gamma' \sdash \delta'}{\varepsilon \vdash^{\beta^+}_k \HH_\Lambda[\theta], \Gamma'[\theta] \sdash \delta'[\theta]}

  \inference[$\forall$R']{\varepsilon \vdash_k^{\alpha} \Gamma
    \sdash \varphi[y / x], \Delta \quad y \not\in \FV(\Gamma, \forall x. \varphi, \Delta)}{\varepsilon \vdash_k^{\alpha + 1} \Gamma[\theta]
    \sdash (\forall x.\varphi, \Delta)[\theta]}
\end{mathpar}
Each of the rules above imposes some restrictions on the substitution $\theta$
used in its definition. For $\RSub$, $\theta$ may be any closed substitution.
For $\RComp'$, $\theta$ must be an assignment with $\dom(\theta) =
\var(\Lambda)$ and $x \not\in \dom(\theta)$. For $\forall$R',
$\theta$ must be a closing substitution.

Using the results from~\needcite{}, it is easy to see that $\CPA$-proofs embed
into \TODO{the system}. Observe also that all transformations involved in this
argument can be implemented primitive recursively.
\begin{proposition}\label{lem:cons-embed-cpa}
  If $\CPA \vdash \Gamma \sdash \Delta$ then there exists $\alpha$ such that
  $\varepsilon \vdash^\alpha_0 \Gamma \sdash \Delta$.
\end{proposition}
\begin{proof}
  \TODO{Refer to the earlier chapter.}
\end{proof}

\begin{fact}
  If $\Lambda \vdash^{\alpha}_k \Gamma \sdash \Delta$ then $\Lambda
  \vdash^\alpha_{k'} \Gamma \sdash \Delta$ for any $k' \geq k$.
\end{fact}

\subsection{$\RCut$-elimination procedure in ...}
\label{sec:spa-cut-elm}

We present the steps in the same order as they are carried out in the procedure.

\begin{lemma}[Companion elimination]\label{lem:cons-comp-sub}
  Let $\theta$ be a closed substitution with $\dom(\theta) = \var(\Lambda) \cup
  \{x\}$ such that $\theta \uh \var(\Lambda)$ is an assignment. If a derivation ends in
    \begin{comfproof}
      \AXC{$\Pi$}
      \UIC{$\Lambda; x^- \mapsto \Gamma \sdash \Delta \vdash^\alpha_k \Gamma \sdash \Delta$}
      \LSC{$\RComp$}
      \UIC{$\Lambda \vdash^{\alpha^+}_k \Gamma \sdash \Delta$}
    \end{comfproof}
    then $\varepsilon \vdash_k^{< \alpha^+} (\HH_\Lambda, \Gamma)[\theta] \sdash \Delta[\theta]$.
\end{lemma}
\begin{proof}
  For this, we prove the following two statements by joint induction on
  $\alpha$. Clearly, 1.\ closes the claim.
  \begin{enumerate}
  \item If $\theta$ is as described above and $\alpha \coloneq \beta^+$ with
    \begin{comfproof}
      \AXC{$\Pi$}
      \UIC{$\Lambda; x^- \mapsto \Gamma \sdash \Delta \vdash^\beta_k \Gamma \sdash \Delta$}
      \LSC{$\RComp$}
      \UIC{$\Lambda \vdash^\alpha_k \Gamma \sdash \Delta$}
    \end{comfproof}
    then $\varepsilon \vdash_k^{< \alpha} (\HH_\Lambda, \Gamma)[\theta] \sdash \Delta[\theta]$.
  \item Let $\Lambda \vdash^\alpha_k \Gamma \sdash \Delta$ and $\theta$ an assignment
    with $\dom(\theta) = \var(\Lambda)$. Then
  $\varepsilon \vdash^{\leq \alpha}_k \HH_\Lambda[\theta], \Gamma[\theta] \sdash
  \Delta[\theta]$.
  \end{enumerate}
  \begin{enumerate}
  \item
    Denote by $\varphi^*$ the formula $\varphi(\Lambda; x^- \mapsto \Gamma
    \sdash \Delta)$, i.e.
    \[
      \varphi^* = \forall x' < x.\forall \vec{u} \leq \vec{v}. \forall
      \vec{y}. (\HH_\Lambda, \Gamma \to \Delta)[x' / x, \vec{u} / \vec{v}]
    \]
    where $\vec{v} = \var(\Lambda) \setminus \{x\}$, $\vec{y} = \FV(\Gamma,
    \Delta) \setminus \{\vec{v}, x\}$ and $x', \vec{u}$ fresh.

    We prove a slightly simpler claim:
    Let $\theta'$ be an assignment with $\dom(\theta') = \var(\Lambda) \cup
    \{x\}$. Then $\varepsilon \vdash^{\leq \beta \cdot n + l}_k \HH_\Lambda[\theta'],
    \Gamma[\theta'] \sdash \Delta[\theta']$ for some $n, l \in \omega$.
    We prove this claim per induction on $\theta'(x)$. Suppose the
    claim held for all $\theta^*$ with $\theta^*(x) < \theta'(x)$.
    Then we derive the following:
    \begin{scprooftree}{0.8}
      \AXC{$\ldots$}
      \AXC{IH: $\theta^*(x) < \theta'(x)$}
      \UIC{$\varepsilon \vdash_k^{\leq \beta \cdot n^* + l^*} \HH_\Lambda[\theta^*], \Gamma[\theta^*] \sdash \Delta[\theta^*]$}
      \UIC{$\varepsilon \vdash_k^{\leq \beta \cdot n^* + l_2^*}~\sdash \forall \vec{y}.~(\HH_\Lambda, \Gamma \to \Delta)[\theta^*]$}
      \AXC{$\ldots$}
      \LSC{$\Ale$R, $\Aleq$R}
      \TIC{$\varepsilon \vdash_k^{\leq \beta \cdot n + l} ~\sdash \varphi^*[\theta']$}
      \AXC{IH 2.}
      \UIC{$\varepsilon \vdash_k^{\leq \beta} \HH_\Lambda[\theta'], \varphi^*[\theta'], \Gamma[\theta'] \sdash \Delta[\theta']$}
      \LSC{$\RCut$}
      \BIC{$\varepsilon \vdash_k^{\leq \beta \cdot (n + 1) + l} \HH_\Lambda[\theta'], \Gamma[\theta'] \sdash \Delta[\theta']$}
    \end{scprooftree}
    We first explain the derivation, then how the bounds are arrived at.
    The $\RCut$-application is allowed as the $\RComp$-rule ensures
    $s(\varphi^*) < k$. On the right-hand side of the $\RCut$, we apply the
    inductive hypothesis 2. of the outer induction, noting that $\HH_{\Lambda ; x^- \mapsto \Gamma
      \sdash \Delta} = \HH_\Lambda, \varphi^*$. On the left-hand side, we apply
    the $\Ale$R and $\Aleq$R rules until all bounded quantifiers have been
    removed from $\varphi^*$, replacing $x'$ and the $\vec{u}$ with numerals.
    This results in a finite number of open premises, each of the form $\sdash
    \forall \vec{y}.(\HH_\Lambda, \Gamma \to
    \Delta)[\theta^*]$, where $\theta^*$ is the assignment with $\theta^*(z)$
    being the numeral inserted by the right-rule of the bounded quantification
    on $z \in \var(\Lambda) \cup \{x\}$. With applications of $\forall$R,
    $\to$R, $\wedge$L and $\vee$R, such sequents can be transformed back into
    the shape $\HH_\Lambda[\theta^*], \Gamma[\theta^*] \sdash \Delta[\theta^*]$,
    to which we may apply the inner inductive hypothesis, observing that
    $\theta^*(x) < \theta'(x)$ by the structure of the $\Ale$R-rule.

    To understand the ordinal bound, observe that the inner inductive hypothesis
    yields derivations of height $\leq \beta \cdot n^* + l^*$ with $n^*, l^* \in
    \omega$ for every assignment $\theta^*$ as described above which, at the
    cost of increasing $l^*$ to some $l^*_2 \in \omega$, can be transformed into
    derivations of $\sdash \forall \vec{y}.~(\HH_\Lambda, \Gamma \to
    \Delta)[\theta^*]$. As the height of $\Ale$R, $\Aleq$R is simply the finite sum of the height
    of their premises, these combine into a derivation of height $\leq \beta
    \cdot n + l$ for some $n, l \in \omega$. The rest of the bound calculation
    is self-explanatory.

    Now, if $\theta$ already is an assignment, we are done
    because $\beta \cdot n + l < \beta^+$.
    Otherwise, $x \not\in \var(\Lambda)$ and $\theta(x)$ is a closed term which is not a numeral.
    Consider the assignment $\theta' \coloneq \theta[x \mapsto
    \ol{\theta(x)^{\NN}}]$. By the previous
    argument, we have $\varepsilon \vdash^{\leq \beta \cdot n + l}_k
    \HH_\Lambda[\theta'], \Gamma[\theta'] \sdash \Delta[\theta']$.
    Because $x \not\in \var(\Lambda)$ and $\FV(\HH_\Lambda) = \var(\Lambda)$,
    this means $\HH_\Lambda[\theta] = \HH_\Lambda[\theta']$ and the
    aforementioned derivation already derives $\varepsilon \vdash^{\leq \beta \cdot n + l}_k
    \HH_\Lambda[\theta], \Gamma[\theta'] \sdash \Delta[\theta']$.
    Further, there exists $l^* \in \omega$ such that
    $\varepsilon \vdash_k^{l} \theta(x) = \theta'(x)$. Then derive
    \begin{scprooftree}{1}
      \AXC{$\vdash^{l^*}_k \theta(x) = \theta'(x)$}
      \AXC{$\varepsilon \vdash^{\leq \beta \cdot n + l}_k \HH_\Lambda[\theta], \Gamma[\theta'] \sdash \Delta[\theta']$}
      \RSC{$=$L}
      \UIC{$\varepsilon \vdash^{\leq \beta \cdot n + l + 1}_k \theta(x) =
        \theta'(x), \HH_\Lambda[\theta], \Gamma[\theta] \sdash \Delta[\theta]$}
      \LSC{$\RCut$}
      \BIC{$\varepsilon \vdash^{\leq \beta \cdot n + l + l^* + 1}_k \HH_\Lambda[\theta], \Gamma[\theta] \sdash \Delta[\theta]$}
    \end{scprooftree}
    observing that still $\beta \cdot n + l + l^* + 1 < \beta^+$ as desired and
    furthermore, $=$L may be applied to change $(\Gamma, \Delta)[\theta]$ to
    $(\Gamma, \Delta)[\theta']$ because $\Gamma, \Delta$ do not contain any
    bounded quantifications.
  \item We proceed by case-distinction on the last rule applied.
    \begin{description}
    \item[$\RComp$:] Then the situation is as follows:
      \begin{comfproof}
        \AXC{$\Pi$}
        \UIC{$\Lambda; x^- \mapsto \Gamma \sdash \Delta \vdash^\beta_k \Gamma \sdash \Delta$}
        \LSC{$\RComp$}
        \UIC{$\Lambda \vdash^{\beta^+}_k \Gamma \sdash \Delta$}
      \end{comfproof}
      There are two cases to consider. First, if $x \in \dom(\theta)$ proceed
      like in 1.\ Otherwise, derive
      \begin{comfproof}
        \AXC{$\Pi$}
        \UIC{$\Lambda; x^- \mapsto \Gamma \sdash \Delta \vdash^\beta_k \Gamma \sdash \Delta$}
        \LSC{$\RComp'$}
        \UIC{$\varepsilon \vdash^{\beta^+}_k \HH_\Lambda[\theta], \Gamma[\theta] \sdash \Delta[\theta]$}
      \end{comfproof}
    \item[$\RBud$:] This case is treated by \Cref{lem:cons-close-buds}.
    \item[$\RDrop$:] Then the situation is as follows:
      \begin{comfproof}
        \AXC{$\Lambda \vdash^{\beta}_k \Gamma \sdash \Delta$}
        \LSC{$\RDrop$}
        \UIC{$\Lambda; \Lambda' \vdash^{\beta + 1}_k \Gamma \sdash \Delta$}
      \end{comfproof}
      and we derive
      \begin{comfproof}
        \AXC{IH 2.}
        \UIC{$\varepsilon \vdash_k^{\leq \beta} \HH_{\Lambda}[\theta'], \Gamma[\theta'] \sdash \Delta[\theta']$}
        \LSC{$\RSub$}
        \UIC{$\varepsilon \vdash_k^{\leq \beta + 1} \HH_{\Lambda}[\theta], \Gamma[\theta] \sdash \Delta[\theta]$}
        \LSC{$\RWk$}
        \UIC{$\varepsilon \vdash_k^{\leq \beta + 1} \HH_{\Lambda; \Lambda'}[\theta], \Gamma[\theta] \sdash \Delta[\theta]$}
      \end{comfproof}
      where $\theta' \coloneq \theta \uh \var(\Lambda)$ and $\RSub$ is not
      inserted if $\var(\Lambda \uh x) = \var(\Lambda)$.
    \item[$\RCase_x$:] Only the case of $x \in \var(\Lambda)$ as below is interesting:
      \begin{comfproof}
        \AXC{$\Pi_0$}
        \UIC{$\Lambda \uh x \vdash^{\beta}_k \Gamma[0 / x] \sdash \Delta[0 / x]$}
        \AXC{$\Pi_S$}
        \UIC{$\Lambda^{+ x} \vdash^{\beta'}_k \Gamma[S\,x / x] \sdash
          \Delta[S\,x / x]$}
        \LSC{$\RCase_x$}
        \BIC{$\Lambda \vdash^{\beta + \beta' + 1}_k \Gamma \sdash \Delta$}
      \end{comfproof}
      Here, we must perform a case-distinction on $\theta(x)$:
      \begin{description}
      \item[$\theta(x) = 0$:] In this case, simply derive
        \begin{comfproof}
          \AXC{IH 2. for $\Pi_0$}
          \UIC{$\varepsilon \vdash^{\leq \beta}_k \HH_{\Lambda \uh x}[\theta', 0 / x],
            \Gamma[\theta, 0 / x] \sdash \Delta[\theta, 0 / x]$}
          \LSC{$\RSub$}
          \UIC{$\varepsilon \vdash^{\leq\beta + 1}_k \HH_{\Lambda \uh x}[\theta], \Gamma[\theta] \sdash \Delta[\theta]$}
          \LSC{$\RWk$}
          \UIC{$\varepsilon \vdash^{\leq\beta + 1}_k \HH_\Lambda[\theta], \Gamma[\theta] \sdash \Delta[\theta]$}
        \end{comfproof}
        where $\theta' \coloneq \theta \uh \var(\Lambda)$ and $\RSub$ is not
        inserted if $\var(\Lambda) = \var(\Lambda; \Lambda')$.
      \item[$\theta(x) = m + 1$:] In this case, we define $\theta' \coloneq
        \theta[x \mapsto m]$ and derive
        \begin{comfproof}
          \AXC{IH 2. for $\Pi_s$}
          \UIC{$\varepsilon \vdash_k \HH_{\Lambda^{+x}}[\theta'], \Gamma[\theta] \sdash \Delta[\theta]$}
          \LSC{$\RShift$}
          \UIC{$\varepsilon \vdash_k \HH_\Lambda[\theta], \Gamma[\theta] \sdash \Delta[\theta]$}
        \end{comfproof}
        where we note that $\bullet [S x / x][\theta'] = \bullet[\theta]$.
        Observe that the difference between $\HH_\Lambda[\theta]$ and
        $\HH_{\Lambda^{+x}}[\theta']$ is that all $x$-bounds are reduced from $m
        + 1$ to $m$ and that some bounds of the form $\forall x' < x$ are
        replaced by $\forall x' \leq x$. Both of these changes can be
        accomplished by $\RShift$.
      \end{description}
    \item[Other rules:] Usual substitution + weakening argument.
    \end{description}
  \end{enumerate}
\end{proof}

\begin{definition}[End-part]
  Consider a proof in \TODO{system name}. A node of the derivation is said
  to be \emph{in the end-part} if between it and the root of the proof, there is
  no instance of a logical connective rule. That is, no instance of $\to$L,
  $\to$R, $\forall$L, $\forall$R, $\forall$R', $\Aleq$L, $\Ale$R or $\Aleq$R.

  An instance of the aforementioned logical rules whose conclusion is in the
  end-part is called a \emph{boundary rule application}.

  We call the end-part of a proof \emph{closed} if all sequents of nodes in the
  end-part are closed and further the premise of any $\forall$L border
  application is closed.
\end{definition}

\begin{remark}
  Observe that if $\varepsilon \vdash^\alpha_k \Gamma \sdash \Delta$ has a
  closed end-part, the end-part may not contain instances of $\RComp$,
  $\RComp$', $\RSub$, $\RBud$ or $\RCase_x$ as each of these rules requires the
  availability of free variables, either in its premises or consequence.
\end{remark}

\begin{lemma}[End-part closing]\label{lem:cons-close-end}
  Let $\varepsilon \vdash^\alpha_k \Gamma \sdash \Delta$ and $\theta$ be a
  closing substitution for $\Gamma \sdash \Delta$. Then
  $\varepsilon \vdash^{\leq \alpha}_k \Gamma[\theta] \sdash \Delta[\theta]$
  with a closed end-part.
\end{lemma}
\begin{proof}
  Per induction on $\alpha$. We proceed per case-distinction on the last rule
  applied.
  \begin{description}
  \item[$\RComp'$:] Then the situation is as follows:
    \begin{comfproof}
      \AXC{$\Pi$}
      \UIC{$\Lambda; x^- \mapsto \Gamma' \sdash \Delta' \vdash^\beta_k \Gamma' \sdash \Delta'$}
      \LSC{$\RComp'$}
      \UIC{$\varepsilon \vdash^{\beta^+}_k \HH_\Lambda[\theta'], \Gamma'[\theta'] \sdash \Delta'[\theta']$}
    \end{comfproof}
    where $\theta'$ is an assignment, $\dom(\theta) \cap \dom(\theta') = \emptyset$ and $x \in
    \dom(\theta)$. First, apply \Cref{lem:cons-comp-sub} to obtain
    $\varepsilon \vdash^{< \beta^+}_k \HH_\Lambda[\theta^*], \Gamma[\theta^*]
    \sdash \Delta[\theta^*]$ where $\theta^* \coloneq \theta'[x \mapsto
    \theta(x)]$. Then apply the inductive hypothesis with substitution $\theta \uh
    \dom(\theta) \setminus \{x\}$ to obtain $\varepsilon \vdash^{\leq \beta^+}_k
    \HH_\Lambda[\theta' \cup \theta], \Gamma[\theta' \cup \theta] \sdash
    \Delta[\theta' \cup \theta]$ as desired.
  \item[$\RComp$:] This is the same as $\RComp'$ with $\Lambda = \varepsilon$
    and $\theta' \coloneq \emptyset$.
  \item[$\RSub$:] Then the situation is as follows:
    \begin{comfproof}
      \AXC{$\varepsilon \vdash^\beta_k \Gamma'[\theta] \sdash \Delta[\theta]$}
      \LSC{$\RSub$}
      \UIC{$\varepsilon \vdash^{\beta + 1}_k \Gamma'[\theta'] \sdash \Delta[\theta']$}
    \end{comfproof}
    with $\dom(\theta) \cap \dom(\theta') = \emptyset$. Then apply the inductive
    hypothesis with substitution $\theta \cup \theta'$ to obtain $\varepsilon
    \vdash^{\leq \beta}_k \Gamma'[\theta \cup \theta'] \sdash \Delta'[\theta
    \cup \theta']$ as desired.
  \item[$\RCut$:] Then the situation is as follows:
    \begin{comfproof}
      \AXC{$\varepsilon \vdash^{\beta}_k \Gamma \sdash \varphi, \Delta$}
      \AXC{$\varepsilon \vdash^{\beta'}_k \Pi, \varphi \sdash \Sigma$}
      \BIC{$\varepsilon \vdash^{\beta + \beta'}_k \Gamma, \Pi \sdash \Delta, \Sigma$}
    \end{comfproof}
    The only interesting situation arises if $\varphi$ introduces new free
    variables, i.e.\ $\FV(\varphi) \not\subseteq
    \dom(\theta)$. In this case, derive
    \begin{comfproof}
      \AXC{$\varepsilon \vdash^{\leq\beta}_k \Gamma[\theta_L] \sdash (\varphi, \Delta)[\theta_L]$}
      \AXC{$\varepsilon \vdash^{\leq\beta'}_k (\Pi, \varphi)[\theta_R] \sdash \Sigma[\theta_R]$}
      \BIC{$\varepsilon \vdash^{\leq \beta + \beta'}_k (\Gamma, \Pi)[\theta]
        \sdash (\Delta, \Sigma)[\theta]$}
    \end{comfproof}
    where
    \[
      \theta_\bullet(x) \mapsto
      \begin{cases}
        \theta(x) & x \in \dom(\theta) \\
        0 & x \in \FV(\varphi) \setminus \dom(\theta)
      \end{cases}
    \]
  \item[$\forall$R:] Then the situation is as follows:
    \begin{comfproof}
      \AXC{$\varepsilon \vdash^{\beta}_k \Gamma \sdash \varphi[y / x], \Delta$}
      \LSC{$\forall$R}
      \UIC{$\varepsilon \vdash^{\beta + 1}_k \Gamma \sdash \forall x.\varphi, \Delta$}
    \end{comfproof}
    where $x$ is fresh. Then derive
    \begin{comfproof}
      \AXC{$\varepsilon \vdash^{\beta}_k \Gamma \sdash \varphi[y / x], \Delta$}
      \LSC{$\forall$R'}
      \UIC{$\varepsilon \vdash^{\beta + 1}_k \Gamma[\theta] \sdash (\forall
        x.\varphi, \Delta)[\theta]$}
    \end{comfproof}
    noting that we have reached the border of the end-part and need not continue
    transforming the proof.
  \item[Other rules:] Just follow the usual substitution argument.
  \end{description}
\end{proof}

The following are Propositions 7.34 and 7.34 from \parencite{mancosuIntroductionProofTheory2021}.
Because the end-parts in our setting and theirs are under identical
restrictions, their results transfer to our setting with little to no modification.

\begin{fact}[$\RWk$-elimination]\label{lem:cons-elim-wk}
  Let $\varepsilon \vdash^\alpha_k \Gamma \sdash \Delta$ with $\Gamma, \Delta$
  atomic and a closed end-part. Then $\varepsilon \vdash^\beta_{k'} \Gamma' \sdash
  \Delta$ with $\Gamma' \subseteq \Gamma$, $\beta \leq \alpha$, $k' < k$ and a closed,
  $\RWk$-free end-part.
\end{fact}

\begin{fact}\label{lem:cons-find-cut}
  Let $\Pi$ be a proof of $\varepsilon \vdash^\alpha_k \Gamma \sdash \Delta$
  with a closed, $\RWk$-free end-part. Then either all of $\Pi$ is in the
  end-part or the end-part contains an instance of a non-atomic $\RCut$ with
  both of its premises containing a boundary rule application to its $\RCut$-formula.
\end{fact}
% \begin{proof}
%   Consider any branch originating at a non-atomic $\RCut$-instance in the
%   end-part which `carries' a $\RCut$-formula.
%   Because the axiomatic rules do not allow for side-formulas and the end-part is
%   weakening free, any such a branch must contain a boundary rule instance.

%   We prove this claim by successively considering $\RCut$-instances with
%   increasing distance to the boundary.
%   First consider any highest non-atomic $\RCut$-instance in the end-part: By the
%   argument above, each of its premises must contain at least one boundary rule
%   application. If both of its premises contain a boundary rule application to
%   its $\RCut$-formula, a desired $\RCut$-instance has been found. Otherwise, it must contain a boundary rule
%   application to a $\RCut$-formula of a $\RCut$-instance further removed from
%   the boundary. In the latter case, iteratively consider an $\RCut$-instance such that there are at
%   most $n$ $\RCut$-instances on any branch between it and the boundary, i.e. its
%   $\RCut$-distance to the boundary is $n + 1$, and no
%   such $\RCut$-instance possesses the desired property. Then both of its
%   premises must contain at least one boundary rule application to
%   $\RCut$-formulas of $\RCut$-height at least $n + 1$. As before, one of the
%   premises does not contain a boundary rule application to the $\RCut$-formula
%   under consideration, it must contain one corresponding to a $\RCut$-instance
%   of height at least $n + 2$. Ultimately, this argument must reach a
%   $\RCut$-instance closest to the root, which must then posses the desired
%   property.
% \end{proof}

We can now prove the $\RCut$-reduction lemma.

\begin{lemma}[$\RCut$-reduction]\label{lem:cons-cut-red}
  Let $\Pi$ be a proof of $\varepsilon \vdash^\alpha_0 \Gamma \sdash \Delta$ with $\Gamma, \Delta$
  atomic and a closed, $\RWk$-free end-part. If $\Pi$ is not its own end-part then $\varepsilon \vdash^\beta_0
  \Gamma \sdash \Delta$ with $\beta < \alpha$.
\end{lemma}
\begin{proof}
  By \Cref{lem:find-cut} there exists a non-atomic $\RCut$-application with
  $\RCut$-formula $C$ in the
  end-part such that both of its premises contain a boundary rule application
  whose principal formula is the $\RCut$-formula. We reduce this $\RCut$,
  the reduction strategy depending on the principal connective of the $C$.
  Observe that, as all these reductions operate within the closed end-part and
  no $\RComp$-instances can occur within it, every sequent we mention has stack
  $\varepsilon$. We omit this information from the derivations below to save
  some space and reduce clutter. Furthermore, as the end-part does not contain
  any `cyclic rules', most of there reductions are standard. We thus only treat
  those reductions which involve novel rules we introduced here, referring the
  reader to, for example, \parencite{mancosuIntroductionProofTheory2021} for the
  remaining cases.

  \begin{description}
  \item[$\forall x.A$:]
    As $s(\forall x.A) > 1$, there must be a $\RRaise$-application
    `underneath' the $\RCut$-application under consideration. We only handle the
    case of the right-rule being $\forall$R', the case of $\forall$R being
    analogous. Including the
    $\RRaise$, the situation can be sketched as follows
    \begin{comfproof}
      \AXC{$\Pi'_L$}
      \UIC{$\vdash^{\alpha'}_{k_L} \Gamma' \sdash \varphi_L[y / x], \Delta'$}
      \LSC{$\forall$R'}
      \UIC{$\vdash^{\alpha' + 1}_{k_L} \Gamma'[\theta] \sdash (\forall x. \varphi_L, \Delta')[\theta]$}
      \DNC{$\Pi_L$}
      \UIC{$\vdash^{\alpha}_{k} \Gamma \sdash \forall x. \varphi, \Delta$}
      \AXC{$\Pi'_R$}
      \UIC{$\vdash^{\beta'}_{k_R} \Pi', \varphi_R(t) \sdash \Sigma'$}
      \RSC{$\forall$L}
      \UIC{$\vdash^{\beta' + 1}_{k_R} \Pi', \forall x. \varphi_R \sdash \Sigma'$}
      \DNC{$\Pi_R$}
      \UIC{$\vdash^{\beta}_{k} \Pi, \forall x. \varphi \sdash \Sigma$}
      \LSC{$\RCut$}
      \BIC{$\vdash^{\alpha + \beta}_{k} \Gamma, \Pi \sdash \Delta, \Sigma$}
      \DNC{$\Pi$}
      \UIC{$\vdash^{\gamma}_{k} \Xi \sdash \Theta$}
      \LSC{$\RRaise$}
      \UIC{$\vdash^{\omega^\gamma}_{k - 1} \Xi \sdash \Theta$}
      \DOC{}
    \end{comfproof}
    where $t$ closed and $\varphi, \varphi_L, \varphi_R$ differ only by terms
    and bounds,
    which may be caused by instances of $=$L and $\RShift$ along $\Pi_L$ and $\Pi_R$.
    Further, $\alpha' + 1 \leq \alpha$, $\beta' + 1 \leq \beta$ and $\alpha +
    \beta \leq \gamma$.

    Replace this with the following derivation. We write $\Pi_\bullet,
    \varphi(t)$ for the derivation $\Pi_\bullet$
    weakened by $\varphi(t)$, in which case all substitutions caused by $=$L and
    shifts caused by $\RShift$ in
    $\forall x. \varphi$ are `replayed' in this additional $\varphi(t)$.
    Further, observe that the $\RCut$ on $\varphi(t)$ satisfies $s(\varphi(t)) < k - 1$
    as $s(\forall x. \varphi) = s(\varphi) + 1 < k$. For the application of
    \Cref{lem:cons-close-end} we choose the closing substitution $\theta[y
    \mapsto t]$.
    \begin{scprooftree}{0.6}
      \AXC{$\Pi'_L$}
      \UIC{$\vdash^{\alpha'}_{k_L} \Gamma' \sdash \varphi_L[y / x], \Delta'$}
      \LSC{\Cref{lem:cons-close-end}}
      \UIC{$\vdash^{\alpha'}_{k_L} \Gamma'[\theta] \sdash (\varphi_L(t), \Delta')[\theta]$}
      \LSC{$\RWk$}
      \UIC{$\vdash^{\alpha'}_{k_L} \Gamma'[\theta] \sdash (\forall x. \varphi_L, \varphi_L(t), \Delta')[\theta]$}
      \DNC{$\Pi_L, \varphi(t)$}
      \UIC{$\vdash^{\alpha^*}_{k_L} \Gamma \sdash \forall x. \varphi, \varphi(t) \Delta$}

      \AXC{$\Pi'_R$}
      \UIC{$\vdash^{\beta'}_{k_R} \Pi', \varphi_R(t) \sdash \Sigma'$}
      \RSC{$\forall$L}
      \UIC{$\vdash^{\beta' + 1}_{k_R} \Pi', \forall x. \varphi_R \sdash \Sigma'$}
      \DNC{$\Pi_R$}
      \UIC{$\vdash^{\beta}_{k} \Pi, \forall x. \varphi \sdash \Sigma$}

      \LSC{$\RCut$}
      \BIC{$\vdash^{\alpha^* + \beta}_{k} \Gamma, \Pi \sdash  \varphi(t), \Delta, \Sigma$}
      \DNC{$\Pi, \varphi(t)$}
      \UIC{$\vdash^{\gamma_L}_{k} \Xi \sdash \varphi(t), \Theta $}
      \LSC{$\RRaise$}
      \UIC{$\vdash^{\omega^{\gamma_L}}_{k - 1} \Xi \sdash \varphi(t), \Theta $}


      \AXC{$\Pi'_L$}
      \UIC{$\vdash^{\alpha'}_{k_L} \Gamma' \sdash \varphi_L[y / x], \Delta'$}
      \LSC{$\forall$R'}
      \UIC{$\vdash^{\alpha' + 1}_{k_L} \Gamma'[\theta] \sdash (\forall x. \varphi_L, \Delta')[\theta]$}
      \DNC{$\Pi_L$}
      \UIC{$\vdash^{\alpha}_{k} \Gamma \sdash \forall x. \varphi, \Delta$}

      \AXC{$\Pi'_R$}
      \UIC{$\vdash^{\beta'}_{k_R} \Pi', \varphi_R(t) \sdash \Sigma'$}
      \RSC{$\RWk$}
      \UIC{$\vdash^{\beta'}_{k_R} \Pi', \varphi_R(t), \forall x. \varphi_R \sdash \Sigma'$}
      \DNC{$\Pi_R, \varphi(t)$}
      \UIC{$\vdash^{\beta^*}_{k} \Pi, \varphi(t), \forall x. \varphi \sdash \Sigma$}
      \RSC{$\RCut$}
      \BIC{$\vdash^{\alpha + \beta^*}_{k} \Gamma, \Pi, \varphi(t) \sdash \Delta, \Sigma$}
      \DNC{$\Pi, \varphi(t)$}
      \UIC{$\vdash^{\gamma_R}_{k} \Xi, \varphi(t) \sdash \Theta$}
      \LSC{$\RRaise$}
      \UIC{$\vdash^{\omega^{\gamma_R}}_{k - 1} \Xi, \varphi(t) \sdash \Theta$}

      \LSC{$\RCut$}
      \BIC{$\vdash^{\omega^{\gamma_L} + \omega^{\gamma_R}}_{k - 1} \Xi, \Xi \sdash \Theta, \Theta$}
      \LSC{$\RCtr$}
      \UIC{$\vdash^{\omega^{\gamma_L} + \omega^{\gamma_R}}_{k - 1} \Xi \sdash \Theta$}
      \DOC{}
    \end{scprooftree}
    We continue by justifying that the claimed ordinal bounds hold. Observe
    that for any sub-derivation of the end-part with one open premise, such as
    $\Pi_L$, $\Pi_R$, $\Pi$ or the remaining derivation underneath the raise
    rule, decreasing the ordinal bound at its open premise will lead to a strict
    decrease in the ordinal bound at its root, e.g. $\alpha^* < \alpha$ at the
    left instance of $\Pi_L$ above because the bound at its premise is $\alpha'$
    instead of $\alpha' + 1$. This is because the only rule which could
    `disturb' the propagation of such a decrease is the $\RComp$-rule, which we
    know does not occur in the closed end-part. For this reason, we have $\alpha^* <
    a$, $\beta^* < \beta$ and hence $\gamma_L, \gamma_R < \gamma$. Hence
    $\omega^{\gamma_L} + \omega^{\gamma_R} < \omega^\gamma$, meaning the ordinal
    bound of the resulting derivation is lower than that of the initial
    derivation over all.
  % \item[$\varphi \to \psi$:]
  %   \
  %   \begin{comfproof}
  %     \AXC{$\Pi'_L$}
  %     \UIC{$\vdash^{\alpha'}_{k_L} \Gamma', \varphi_L \sdash \psi_L, \Delta'$}
  %     \LSC{$\to$R}
  %     \UIC{$\vdash^{\alpha' + 1}_{k_L} \Gamma' \sdash \varphi_L \to \psi_L, \Delta'$}
  %     \DNC{$\Pi_L$}
  %     \UIC{$\vdash^{\alpha}_{k} \Gamma \sdash \varphi \to \psi, \Delta$}
  %     \AXC{$\Pi^1_R$}
  %     \UIC{$\vdash^{\beta_1}_{k_R} \Omega_1 \sdash \varphi_R, \Sigma_1$}
  %     \AXC{$\Pi^2_R$}
  %     \UIC{$\vdash^{\beta_2}_{k_R} \Omega_2, \psi_R \sdash \Sigma_2$}
  %     \RSC{$\to$L}
  %     \BIC{$\vdash^{\beta_1 + \beta_2}_{k_R} \Omega_1, \Omega_2, \varphi_R \to \psi_R
  %       \sdash \Sigma_1, \Sigma_2$}
  %     \DNC{$\Pi_R$}
  %     \UIC{$\vdash^{\beta}_{k} \Omega, \varphi \to \psi \sdash \Sigma$}
  %     \LSC{$\RCut$}
  %     \BIC{$\vdash^{\alpha + \beta}_{k} \Gamma, \Omega \sdash \Delta, \Sigma$}
  %     \DNC{$\Pi$}
  %     \UIC{$\vdash^{\gamma}_{k} \Xi \sdash \Theta$}
  %     \LSC{$\RRaise$}
  %     \UIC{$\vdash^{\omega^\gamma}_{k - 1} \Xi \sdash \Theta$}
  %     \DOC{}
  %   \end{comfproof}

  %   \begin{scprooftree}{0.5}
  %     \AXC{$\Pi'_L$}
  %     \UIC{$\vdash^{\alpha'}_{k_L} \Gamma', \varphi_L \sdash \psi_L, \Delta'$}
  %     \LSC{$\to$R}
  %     \UIC{$\vdash^{\alpha' + 1}_{k_L} \Gamma' \sdash \varphi_L \to \psi_L, \Delta'$}
  %     \DNC{$\Pi_L$}
  %     \UIC{$\vdash^{\alpha}_{k} \Gamma \sdash \varphi \to \psi, \Delta$}

  %     \AXC{$\Pi^1_R$}
  %     \UIC{$\vdash^{\beta_1}_{k_R} \Omega_1 \sdash \varphi_R, \Sigma_1$}
  %     \RSC{$\RWk$}
  %     \UIC{$\vdash^{\beta_1}_{k_R} \Omega_1, \Omega_2, \varphi_R \to
  %       \psi_R \sdash \varphi_R, \Sigma_1, \Sigma_2$}
  %     \DNC{$\Pi_R$}
  %     \UIC{$\vdash^{\beta^*}_{k} \Omega, \varphi \to \psi \sdash \varphi, \Sigma$}

  %     \LSC{$\RCut$}
  %     \BIC{$\vdash^{\alpha + \beta^*}_{k} \Gamma, \Omega \sdash \varphi, \Delta, \Sigma$}
  %     \DNC{$\Pi$}
  %     \UIC{$\vdash^{\gamma_1}_{k} \Xi \sdash \varphi, \Theta$}
  %     \LSC{$\RRaise$}
  %     \UIC{$\vdash^{\omega^{\gamma_1}}_{k - 1} \Xi \sdash \varphi, \Theta$}



  %     \AXC{$\Pi'_L$}
  %     \UIC{$\vdash^{\alpha'}_{k_L} \Gamma', \varphi_L \sdash \psi_L, \Delta'$}
  %     \LSC{$\RWk$}
  %     \UIC{$\vdash^{\alpha'}_{k_L} \Gamma', \varphi_L \sdash \varphi_L \to
  %       \psi_L, \psi_L, \Delta'$}
  %     \DNC{$\Pi_L$}
  %     \UIC{$\vdash^{\alpha^*}_{k} \Gamma, \varphi \sdash \varphi \to \psi, \psi, \Delta$}

  %     \AXC{$\Pi^1_R$}
  %     \UIC{$\vdash^{\beta_1}_{k_R} \Omega_1 \sdash \varphi_R, \Sigma_1$}
  %     \AXC{$\Pi^2_R$}
  %     \UIC{$\vdash^{\beta_2}_{k_R} \Omega_2, \psi_R \sdash \Sigma_2$}
  %     \RSC{$\to$L}
  %     \BIC{$\vdash^{\beta_1 + \beta_2}_{k_R} \Omega_1, \Omega_2, \varphi_R \to \psi_R
  %       \sdash \Sigma_1, \Sigma_2$}
  %     \DNC{$\Pi_R$}
  %     \UIC{$\vdash^{\beta}_{k} \Omega, \varphi \to \psi \sdash \Sigma$}

  %     \LSC{$\RCut$}
  %     \BIC{$\vdash^{\alpha^* + \beta}_{k} \Gamma, \Omega, \varphi \sdash \psi, \Delta, \Sigma$}
  %     \DNC{$\Pi$}
  %     \UIC{$\vdash^{\gamma_2}_{k} \Xi, \varphi \sdash \psi, \Theta$}
  %     \LSC{$\RRaise$}

  %     \UIC{$\vdash^{\omega^{\gamma_2}}_{k - 1} \Xi, \varphi \sdash \psi, \Theta$}



  %     \AXC{}
  %     \DNC{}
  %     \UIC{$\vdash^{\gamma_3}_{k} \Xi \sdash \psi, \Theta$}
  %     \RSC{$\RRaise$}
  %     \UIC{$\vdash^{\omega^{\gamma_3}}_{k - 1} \Xi \sdash \psi, \Theta$}

  %     \BIC{$\vdash^{\omega^{\gamma_2} + \omega^{\gamma_3}}_{k - 1} \Xi, \Xi \sdash \psi, \Theta, \Theta$}
  %     \BIC{$\vdash^{\omega^{\gamma_1} + \omega^{\gamma_2} + \omega^{\gamma_3}}_{k - 1} \Xi, \Xi, \Xi \sdash \Theta, \Theta, \Theta$}
  %     \LSC{$\RCtr$}
  %     \UIC{$\vdash^{\omega^{\gamma_1} + \omega^{\gamma_2} + \omega^{\gamma_3}}_{k - 1} \Xi \sdash \Theta$}
  %     \DOC{}
  %   \end{scprooftree}
  \item[$\forall y <^\sigma n$:]
    As $s(\forall x <^\sigma n.A) > 1$, there must be a $\RRaise$-application
    `underneath' the $\RCut$-application under consideration. Including the
    $\RRaise$, the situation can be sketched as below. Again, $\varphi,
    \varphi_L, \varphi_R$ differ only by terms, which may be caused by instances
    of $=$L along $\Pi_L$ and $\Pi_R$. Because upper bounds cannot be subject to
    changes via $=$L, because $m$ must have
    been arrived at from $n$ using only $\RShift$-applications, meaning $m
    <^\sigma n$ and thus that the $\varphi_L(\ol{m})$-case must be among the premises of the $\Asig$R-application.
    Further, $1 + \sum \alpha_i \leq \alpha$, $\beta' + 1 \leq \beta$ and $\alpha +
    \beta \leq \gamma$.
    \begin{comfproof}
      \AXC{$\ldots$}
      \AXC{$\Pi^m_L$}
      \UIC{$\vdash^{\alpha_m}_{k_L} \Gamma' \sdash \varphi_L(\ol{m}), \Delta'$}
      \LSC{$\Asig$R}
      \AXC{$\ldots$}
      \TIC{$\vdash^{1 + \sum \alpha_i}_{k_L} \Gamma' \sdash \forall x <^\sigma \ol{n}. \varphi_L, \Delta'$}
      \DNC{$\Pi_L$}
      \UIC{$\vdash^{\alpha}_{k} \Gamma \sdash \forall x <^\sigma \ol{n}. \varphi, \Delta$}

      \AXC{$\Pi'_R$}
      \UIC{$\vdash^{\beta'}_{k_R} \Omega', \varphi_R(\ol{m}) \sdash \Sigma'$}
      \RSC{$\Aleq$L}
      \UIC{$\vdash^{\beta' + 1}_{k_R} \Omega', \forall x \leq \ol{m}. \varphi_R \sdash \Sigma'$}
      \DNC{$\Pi_R$}
      \UIC{$\vdash^{\beta}_{k} \Omega, \forall x <^\sigma \ol{n}. \varphi \sdash \Sigma$}

      \LSC{$\RCut$}
      \BIC{$\vdash^{\alpha + \beta}_{k} \Gamma, \Omega \sdash \Delta, \Sigma$}
      \DNC{$\Pi, \varphi(\ol{m})$}
      \UIC{$\vdash^{\gamma}_{k} \Xi \sdash \Theta$}
      \LSC{$\RRaise$}
      \UIC{$\vdash^{\omega^\gamma}_{k - 1} \Xi \sdash \Theta$}
      \DOC{}
    \end{comfproof}

    Replace this with the following derivation. We again write $\Pi_\bullet,
    \varphi(\ol{m})$ for the derivation $\Pi_\bullet$
    weakened by $\varphi(\ol{m})$, in which case all substitutions caused by $=$L and
    shifts caused by $\RShift$ in
    $\forall x <^\sigma \ol{n}.\varphi$ are `replayed' in this additional $\varphi(\ol{m})$.
    Further, observe that the $\RCut$ on $\varphi(\ol{m})$ satisfies $s(\varphi(\ol{m})) < k - 1$
    as $s(\forall x <^\sigma \ol{n}. \varphi) = s(\varphi) + 1 < k$.
    \begin{scprooftree}{0.5}
      \AXC{$\Pi^m_L$}
      \UIC{$\vdash^{\alpha_m}_{k_L} \Gamma' \sdash \varphi_L(\ol{m}), \Delta'$}
      \LSC{$\RWk$}
      \UIC{$\vdash^{\alpha_m}_{k_L} \Gamma' \sdash \forall x <^\sigma \ol{n}. \varphi_L, \varphi_L(\ol{m}), \Delta'$}
      \DNC{$\Pi_L, \varphi(\ol{m})$}
      \UIC{$\vdash^{\alpha^*}_{k} \Gamma \sdash \forall x <^\sigma \ol{n}. \varphi,\varphi(\ol{m}), \Delta$}

      \AXC{$\Pi'_R$}
      \UIC{$\vdash^{\beta'}_{k_R} \Omega', \varphi_R(\ol{m}) \sdash \Sigma'$}
      \RSC{$\Aleq$L}
      \UIC{$\vdash^{\beta' + 1}_{k_R} \Omega', \forall x \leq \ol{m}. \varphi_R \sdash \Sigma'$}
      \DNC{$\Pi_R$}
      \UIC{$\vdash^{\beta}_{k} \Omega, \forall x <^\sigma \ol{n}. \varphi \sdash \Sigma$}

      \LSC{$\RCut$}
      \BIC{$\vdash^{\alpha^* + \beta}_{k} \Gamma, \Omega \sdash \varphi(\ol{m}), \Delta, \Sigma$}
      \DNC{$\Pi$}
      \UIC{$\vdash^{\gamma_L}_{k} \Xi \sdash \varphi(\ol{m}), \Theta$}
      \LSC{$\RRaise$}
      \UIC{$\vdash^{\omega^{\gamma_L}}_{k - 1} \Xi \sdash \varphi(\ol{m}), \Theta$}



      \AXC{$\ldots$}
      \AXC{$\Pi^m_L$}
      \UIC{$\vdash^{\alpha_m}_{k_L} \Gamma' \sdash \varphi_L(\ol{m}), \Delta'$}
      \LSC{$\Asig$R}
      \AXC{$\ldots$}
      \TIC{$\vdash^{\sum \alpha_i}_{k_L} \Gamma' \sdash \forall x <^\sigma \ol{n}. \varphi_L, \Delta'$}
      \DNC{$\Pi_L$}
      \UIC{$\vdash^{\alpha}_{k} \Gamma \sdash \forall x <^\sigma \ol{n}. \varphi, \Delta$}

      \AXC{$\Pi'_R$}
      \UIC{$\vdash^{\beta'}_{k_R} \Omega', \varphi_R(\ol{m}) \sdash \Sigma'$}
      \RSC{$\RWk$}
      \UIC{$\vdash^{\beta'}_{k_R} \Omega', \varphi_R(\ol{m}), \forall x \leq \ol{m}. \varphi_R \sdash \Sigma'$}
      \DNC{$\Pi_R, \varphi(\ol{m})$}
      \UIC{$\vdash^{\beta^*}_{k} \Omega, \varphi(\ol{m}), \forall x <^\sigma \ol{n}. \varphi \sdash \Sigma$}

      \LSC{$\RCut$}
      \BIC{$\vdash^{\alpha + \beta^*}_{k} \Gamma, \Omega, \varphi(\ol{m}) \sdash \Delta, \Sigma$}
      \DNC{$\Pi, \varphi(\ol{m})$}
      \UIC{$\vdash^{\gamma_R}_{k} \Xi, \varphi(\ol{m}) \sdash \Theta$}
      \LSC{$\RRaise$}
      \UIC{$\vdash^{\omega^{\gamma_R}}_{k - 1} \Xi, \varphi(\ol{m}) \sdash \Theta, \Theta$}




      \LSC{$\RCut$}
      \BIC{$\vdash^{\omega^{\gamma_L} + \omega^{\gamma_R}}_{k - 1} \Xi, \Xi \sdash \Theta, \Theta$}
      \LSC{$\RCtr$}
      \UIC{$\vdash^{\omega^{\gamma_L} + \omega^{\gamma_R}}_{k - 1} \Xi \sdash \Theta$}
      \DOC{}
    \end{scprooftree}
    The justification of the decrease in the ordinal bound is analogous to the
    case of $\forall x.\varphi$, noting that $\alpha_m < 1 + \sum \alpha_i$ and
    $\beta' < \beta' + 1$.
  \end{description}
\end{proof}

\begin{theorem}[$\RCut$-elimination]\label{lem:cpa-cons}
  If $\Pi$ is a proof of $\varepsilon \vdash^\alpha_0 \Gamma \sdash \Delta$ with $\Gamma, \Delta$
  closed atomic then there are $\Gamma' \subseteq \Gamma, \Delta' \subseteq
  \Delta$ and a proof $\Pi'$ of $\varepsilon \vdash^\beta_0 \Gamma' \sdash
  \Delta'$ such that $\Pi'$ is closed and contains no non-atomic $\RCut$s.
\end{theorem}
\begin{proof}
  We shall use the procedure below to arrive at the proof $\Pi'$ starting from
  the proof $\Pi : \varepsilon \vdash^\alpha_0 \Gamma \sdash \Delta$:
  \begin{enumerate}
  \item Apply \Cref{lem:cons-close-end} with $\theta = \emptyset$ to obtain
    $\Pi_c : \varepsilon \vdash^{\leq \alpha}_0 \Gamma \sdash \Delta$
    with a closed end-part.
  \item Apply \Cref{lem:cons-elim-wk} to $\Pi_c$ to obtain $\Pi_w : \varepsilon
    \vdash^{\leq \alpha}_0 \Gamma' \sdash \Delta'$ with $\Gamma \subseteq
    \Gamma', \Delta \subseteq \Delta'$ and a closed, $\RWk$-free end-part.
  \item By \Cref{lem:cons-find-cut} there are two possibilities:
    \begin{enumerate}[label=(\alph*)]
    \item $\Pi_w$ is not its own end-part: Then, by \Cref{lem:cons-cut-red}
      there exists $\Pi_r : \varepsilon \vdash^{< \alpha}_0 \Gamma' \sdash
      \Delta'$. Continue from step 1.\ using $\Pi \coloneq \Pi_r$.
    \item $\Pi_w$ is its own closed end-part: First, this means there are now
      `cyclic rules', such as $\RComp$, $\RBud$ or $\RCase$ in $\Pi_w$.
      Further, observe that any trace of a non-atomic $\RCut$-formula must end
      either in $\RWk$ or a boundary rule application. As $\Pi_w$ contains
      neither, it cannot contain non-atomic $\RCut$s. Thus, we may pick $\Pi'
      \coloneq \Pi_w$ and are done.
    \end{enumerate}
  \end{enumerate}
  If the procedure above always terminates, we can always obtain the desired
  $\Pi'$. Termination is ensured by the fact that whenever the procedure `jumps'
  from step 3a.\ to step 1., the ordinal height of the proof has decreased.
  Thus, non-termination of the procedure leads an infinitely decreasing sequence
  in $\varepsilon_0$ which we know cannot exist.
\end{proof}

\begin{corollary}
  $\PRA + \TIi(\varepsilon_0) \vdash $ ``$\CPA$ is consistent''
\end{corollary}
\begin{proof}
  If no $\CPA$ derivation of the empty sequent, i.e.\ $~ \sdash ~$, exists then
  $\CPA$ is consistent. Working within $\PRA$, suppose there was such a
  derivation $\Pi$. By \Cref{lem:cons-embed-cpa}, this derivation can be
  transformed, primitive recursively, into $\Pi' : \varepsilon \vdash^\alpha_0 ~
  \sdash ~$ in \TODO{system name}. By \Cref{lem:cpa-cons} one can obtain a closed
  derivation $\Pi'' : \varepsilon \vdash^\beta_0 ~ \sdash ~$ with $\beta \leq
  \alpha$ without non-atomic $\RCut$s. Observe that each `iteration step' of the
  procedure used to obtain $\Pi''$ from $\Pi'$ is primitive recursive and the
  termination of the procedure over all follows from $\TIi(\varepsilon_0)$. Now
  $\PRA$ can prove that a proof of $~ \sdash ~$ involving only closed, atomic
  formulas, like $\Pi''$, cannot exist.
\end{proof}

\section{Unravelling Cyclic Gödel's T}
\label{sec:unrvl-gödel}

% \begin{definition}
%   The \emph{derivation rules} of Cyclic Gödel's T $\CGT$ are:
%   \begin{mathpar}
%     \inference[\RId]{}{x : A \sdash A}

%     \inference[$0$]{}{\sdash N}

%     \inference[S]{}{x : N \sdash N}

%     \inference[L]{\Gamma \sdash \rho \quad \Gamma, x : A \sdash B}{\Gamma, x : \rho \to A \sdash B}

%     \inference[R]{\Gamma, x : A \sdash B \quad x \text{ fresh}}{\Gamma \sdash A \to B}

%     \inference[\RCond]{\Gamma \sdash A \quad \Gamma, x : N \sdash A}{\Gamma, x :
%       N \sdash A}

%     \inference[\REx]{\Gamma_0, x : A, y : B, \Gamma_1 \sdash C}{\Gamma_0, y : B,
%       x : A, \Gamma_1 \sdash C}

%     \inference[\RWk]{\Gamma \sdash B}{\Gamma, x : A \sdash B}

%     \inference[\RCtr]{\Gamma, x : A, y : A \sdash B \quad y \text{
%         fresh}}{\Gamma, x : A \sdash B}

%     \inference[\RCut]{\Gamma \sdash B \quad \Gamma, x : B \sdash A \quad x
%       \text{ fresh}}{\Gamma \sdash A}
%   \end{mathpar}
% \end{definition}

