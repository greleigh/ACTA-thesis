% Emacs, this is -*-latex-*-

\chapter{Constructivity of Cyclic Proof Theory}
\label{chap:constructivity}

\section{Working in the CIC}
\label{sec:cic-intro}

All arguments in this chapter are carried out over the base theory of the
Calculus of Inductive Constructions (CIC) \needcite{}, the type theory
underpinning the Rocq interactive theorem prover~\needcite{}. Giving a full,
formal account of the CIC would be both very finicky and not very useful for the
rest of the chapter. Instead, we shall attempt to convey the important
intuitions relating to the `user's perspective' of the CIC.

% Constructive
The CIC provides a \emph{constructive} foundation of mathematics. In fact, for
the purpose of this chapter, we take `constructive' to always refer to the
notion of constructivity embodied by the CIC. \TODO{Expand}

% Type theory
The CIC is a type theory, meaning its central judgment is of the
form $a : A$, read as `$a$ is of type $A$'.
% - Two type levels
Because the CIC is a
\emph{dependent} type theory, types themselves have types called \emph{type
  universes}. There are two kinds of such type universes, the universe $\Prop$
of propositions and the universes $\Type_i$ of computational types. Their
relationships are as follows:
\[
  \Prop \subset \Type_0 \subset \ldots \subset \Type_n \subset \ldots
  \qquad \qquad
  \Prop : \Type_0 \quad \Type_0 : \Type_1 \quad \ldots
\]
No type universe $\Univ$ contains itself, i.e. $\Univ : \Univ$ is never the
case, as this would give rise to an inconsistency. To aid readability by not
bothering with having to identify universe levels for every type (instance), we
following the CIC's implementation by Rocq in identifying all universes
$\Type_i$ with a singular universe $\Type$.

% - Type are computational types
% - A -> B are functions
Intuitively, the universe $\Type$ of computational types contains types of
concrete values which one wishes to reason about, for example the types $\NN$ of
natural numbers and $\BB$ of booleans. A function $f : A \to B$ with $A, B :
\Type$ should be read\footnote{\label{ft:cic}in the absence of any axioms beyond
the CIC} as a \emph{computable transformation} of $A$-values into
$B$-values.

The type $\Prop$ of propositions, on the other hand, contains types which
represent logical expressions in the sense of the Curry-Howard-correspondence,
such as the type $a =_A b$ of equality proofs between two values $a, b : A$ or
the empty type $\bot$ of falsity. A function $f : A \to B$ with $A, B : \Prop$
should be read\footref{ft:cic} as a \emph{constructive proof} that $A$ implies
$B$. A dependent function $g : \forall a : A.~B(a)$, even with $A :
\Type$, yields a proof $g(a) : B(a)$ for every $a : A$ and should thus be viewed
as a proof of the equinimous universal statement.

% - Consintency of Prop are propositions
While the `behavior' of the CIC within $\Type$ is unsurprising, within $\Prop$
it exhibits some properties which may not be expected. The CIC was designed in
such a way that principles like those below are consistent.
\[
  \forall P : \Prop.~P =_{\Prop} \top \vee P =_{\Prop} \bot
  \qquad \qquad
  \forall P : \Prop. \forall a, b : P.~a =_{\Prop} b
\]
To ensure such consistencies, all functions $f : A \to B$ with $A : \Prop$ and
$B : \Type$ are constricted by the so-called \emph{elim-restriction}. While the
details are intricate, the upshot is that `the transfer of information from
$\Prop$ to $\Type$ is blocked'.
% - Props are to be erased ?

% - Two kinds of connectives
Because of the elim-restriction, the CIC features two variants of the
`constructively interesting' connectives: the \emph{logical} connectives $\vee$ and
$\exists$ in $\Prop$ and the \emph{computational} connectives $+$ and $\Sigma$
in $\Type$. The crucial difference is that the `outcome' of $A + B$ and the
witness of $\Sigma x : X.~C(x)$ may be used to construct values of a type $Y :
\Type$, the outcome of $A \vee B$ and the witness of $\exists x : X.~C(x)$ may not.
\TODO{$\vee_{\Univ}$ and $\text{exists}_{\Univ}$ notation}

To illustrate the difference, consider the standard definition of the
decidability of a predicate $P : X \to \Prop$ given below, which requires a
computationally `usable' choice between $P(x)$ and $\neg P(x)$ for every $x : X$.
\begin{definition}
  For $X : \Type$ a predicate $P : X \to \Prop$ is \emph{decidable} if $\forall
  x : X.~P(x) + \neg P(x)$.
\end{definition}

The computational `usability' is illustrated below. Observe that the outcome of
$P(x) + \neg P(x)$ may be used in the definition of the \emph{computational}
function $f : X \to \BB$ with $X, \BB : \Type$.
\begin{proposition}
  If $P : X \to \Prop$ is decidable, there exists a function $f : X \to \BB$
  such that $f(x) = \btrue$ iff $P(x)$ holds for every $x : X$.
\end{proposition}
\begin{proof}
  Let $g : \forall x : X.~P(x) + \neg P(x)$ be the witness of decidability. Then define
  \[
    f(x) \coloneq
    \begin{cases}
      \btrue & \text{ if } g(x) = \sleft \\
      \bfalse & \text{ if } g(x) = \sright
    \end{cases}.
  \]
  Correctness of $f : X \to B$ easily follows.
\end{proof}

All `classical' reverse mathematics (see e.g.~\needcite{}) is done relative to
subsystems of second-order arithmetic.
A difficulty of constructive reverse mathematics, in comparison, is that that
there is no such common base over which it is carried out. Indeed, some of the
bases used in the literature differ in the `computational content of their
constructively interesting connectives': In essence, whether ``there exists'' is
interpreted as $\exists$ or $\Sigma$ in the sense of the CIC and similarly for ``or''.
From this perspective, the CIC is an ideal setting for constructive analyses as
it can accommodate various combinations of logical and computational
interpretations of `constructively interesting' connectives, allowing results
derived within it to naturally extend to many other bases of constructive
mathematics (see \Cref{sec:other-con} for more on this).

\section{Non-constructive Principles}
\label{sec:ncp}

\section{Cyclic Proofs in the CIC}
\label{sec:cic-cp}

\section{Constructive Analysis}
\label{sec:canalysis}

\section{Other constructive bases}
\label{sec:other-con}

\TODO{Vis-a-vis HoTT/Agda: Unique choice not provable}
