% Emacs, this is -*-latex-*-

\chapter{Stack-controlled Proof Systems}
\label{chap:stacks}

\section{Definition}
\label{sec:define-stacks}

For the remainder of the section, fix $\RR = (\Seq, \RR, \rho)$ to be a proof
system and $\iota : \RR \to \TT_{\BB}$ a trace interpretation as defined in
\Cref{sec:abs-ps}.

We define $\stacked(\RR)$, the stack-controlled proof system corresponding to
$\iota(\RR)$. A sequent of $\stacked(\RR)$ has the form $\Gamma \mid \Lambda$
where $\Gamma \in \Seq$ is an $\RR$-sequent and $\Lambda$ is a \emph{jump-label
  stack} for $\Gamma$. That means $\Lambda$ is a sequence of
\emph{jump-labels} of the form $X ; \vec{Y} \mapsto \Delta$ where $\Delta$ is an
$\RR$-sequent, $X$ is a finite set and $\vec{Y}$ is a sequence of finite sets.
To be able to characterize the nature of $X$ and $\vec{Y}$ further, denote the
$i$th jump-label in $\Lambda$ (starting at $0$) by $P_i; \vec{G^i} \mapsto
\Delta_i$. $\Lambda$ is a well-formed stack of jump-labels for $\Gamma$ if for
every $i < \abs{\Lambda}$
\begin{enumerate}[label=(\roman*)]
\item $\abs{\vec{G^i}} = i + 1$ and every $G^i_j \subseteq
  \prg(\iota(\Delta_j))$ for $j \leq i$
\item $P_i \subseteq \prg(\iota(\Gamma))$
\end{enumerate}
where $\prg(X) \coloneq X \times \{+, -\}$ is the set of \emph{progress
  annotated $X$-values} and we denote $(x, \sigma) \in \prg(X)$ by $x^\sigma$.

\begin{remark}
  The set $P_i$ tracks the \emph{progress} of the cycles corresponding to the
  jump-label whereas the $\vec{G^i}$ act as \emph{guards} for it and its lower cycles.
\end{remark}

Let $\Gamma, \Delta \in \Seq$ and $\tau : \iota(\Gamma) \to \iota(\Delta)$ and
$\Lambda$ be a jump-label stack for $\Gamma$. We define $\tau(\Lambda)$ as the
jump-label stack for $\Delta$ with $\abs{\tau(\Lambda)} = \abs{\Lambda}$ such
that for $i < \abs{\Lambda}$, if the $i$th label of $\Lambda$ is $(P_i;
\vec{G^i} \mapsto \Xi_i)$ then the $i$th label of $\tau(\Lambda)$ is
\[
  (\{y^+ \mid x^\sigma \in P_i \wedge (x, 1, y) \in \tau\} \cup \{y^\sigma \mid
  x^\sigma \in P_i \wedge (x, 0, y) \in \tau\}, \vec{G^i} \mapsto \Xi_i).
\]

Given two sets $X, Y \subseteq \prg(Z)$ we write
\[
  X \preceq Y \coloneq \forall t^\sigma \in X. \exists \sigma' \geq
  \sigma.~t^{\sigma'} \in Y
\]
imposing on $\{+, -\}$ the ordering of $- < +$.

The rules of $\stacked(\RR)$ consist of annotated rules and additional
\emph{structural} rules. The structural rules are given below
\begin{mathpar}
  \inference[$\RPop$]{\Gamma \mid \Lambda}{\Gamma \mid \Lambda , \Lambda'}

  \inference[$\RDrop$]{\Gamma \mid \Lambda^*}{\Gamma \mid \Lambda}

  \inference[$\RComp$]{\Gamma \mid \Lambda, (X^-; \pi(\Lambda), X^+
    \mapsto \Gamma)}{\Gamma \mid \Lambda}

  \inference[$\RBud$]{\vec{Y} \preceq \pi(\Lambda)}{\Gamma \mid \Lambda, (X^+; \vec{Y}, X^+ \mapsto \Gamma)}
\end{mathpar}
where $\pi(\Lambda) = (P_j)_{j < \abs{\Lambda}}$ and $X^\sigma = X \times
\{\sigma\}$ for $\sigma \in \{+, -\}$. For the $\RDrop$ rule, we have that
$\Lambda$ and $\Lambda^*$ are identical except that $\pi(\Lambda^*) \subseteq
\pi(\Lambda)$, i.e. a label $(P; \vec{G} \mapsto \Delta) \in \Lambda$ may be replaced by
$(P'; \vec{G} \mapsto \Delta) \in \Lambda^*$ with $P' \subseteq P$.

For each rule $r \in \RR$ with $\rho(r) = (\Gamma, \Delta_1, \ldots, \Delta_n)$
and maps $r_i : \iota(\Gamma) \to \iota(\Delta_i)$ given by the trace
interpretation, the following schema gives rules for each jump-label stack
$\Lambda$ for $\Gamma$:
\[
  \inference[r]{\Delta_1 \mid r_1(\Lambda) \quad \ldots \quad \Delta_n \mid r_n(\Lambda)}{\Gamma \mid \Lambda}
\]

\begin{remark}
  \TODO{
    Maybe for $\Aa$ we can simply annotate with $\iota(\Gamma) \times \Aa$
    and take $\tau(\Lambda)$ to map $x^a$ via $(x, b, y)$ to $y^{a \vee b}$?
    But what about guarding??
  }
\end{remark}

\section{Soundness}
\label{sec:stacks-sound}

\section{Completeness}
\label{sec:complete}

\section{Functional trace semantics}
\label{sec:func-stacks}

\section{Deriving Concrete Stack-Controlled Proof Systems}
\label{sec:concrete-stack}

\subsection{Simpson's CHA}
\label{sec:simpson-stacks}

\subsection{Simple CHA}
\label{sec:simple-stacks}

% \section{Dump: Unravelling}

%   Cyclic proof systems for Heyting and Peano arithmetic eschew induction axioms
%   by accepting proofs which are finite graphs rather than trees. Proving that such
%   a cyclic proof system coincides with its more conventional variants is often
%   difficult: Previous proofs in the literature rely on intricate arithmetisations
%   of the metamathematics of the cyclic proof systems.

%   In this article, we present a simple and direct embedding of cyclic proofs for Heyting
%   and Peano arithmetic into
%   purely inductive, i.e. `finitary', proofs by adapting a translation introduced
%   by Sprenger and Dam for a cyclic proof system of $\mFOL$ with explicit ordinal
%   approximations. We extend their method to recover Das' result of $\CP_n \subseteq
% \IP_{n + 1}$ for Peano arithmetic.
%   %to instead give a simple embedding which does not
% %  rely on the arithmetisation of mathematics. 
%   As part of
%   the embedding we present a novel representation of cyclic proofs as a labelled
%   sequent calculus.

% \section{Introduction}
% \label{sec:intro}

% % What is the topic?
% % - Cyclic proofs are graphs
% Cyclic proof systems admit derivations whose underlying structures are finite
% graphs, rather than finite trees.
% % - Soundness condition needed (?)
% Such proof systems are especially well-suited to logics and theories which
% feature fixed points or (co-)inductively defined objects.
% % - Drop induction rules; good for PT and proof search
% Compared to conventional `inductive' proof systems, cyclic proof systems often do not
% require explicit induction rules, which are instead simulated by cycles. This
% makes cyclic proof systems particularly well-suited to proof theoretic
% investigations and applications in automated theorem proving, which are both
% often hampered by induction rules.
% % - Specifically: cyclic HA
% %In this article, we consider a cyclic proof system for Heyting arithmetic
% %motivated by the inductive nature of the natural numbers.

% % Why is the problem solved hard/interesting/worthwhile?
% % - Usually designed for existing proof system/logic
% There are two common methods for proving that a given cyclic proof system
% corresponds to a particular logic or theory.
% % - Often: Sound & Complete
% The usual method is via direct soundness and completeness arguments.
% % It should
% % be noted that often, proving completeness is easier and proving soundness is
% % harder for cyclic proof systems when compared non-cyclic proof system.
% % - But what if semantics not so clear? -> Proof theoretic translation
% This approach is often not well-suited to arithmetical
% theories.
% In such cases, equivalence is instead established via proof theoretic translations
% between
% the cyclic and a conventional inductive proof system.
% Translating inductive proofs into cyclic proofs
% is usually straightforward whereas the opposite direction requires considerable effort.

% % What is the state of the art? What are the limits of current practice?
% % - Arithmetics: Relying on arithmetic expressivity; internalizing expressive
% %   power
% For Peano and Heyting arithmetic, two general strategies for obtaining such
% translations are present in the literature.
% Simpson's approach~\cite{simpsonCyclicArithmeticEquivalent2017},
% later refined by Das~\cite{dasLogicalComplexityCyclic2020},
% formalises a soundness argument for cyclic proofs in subsystems of
% second-order arithmetic, using reflection and conservativity results to
% `return to' the first-order setting. Berardi and
% Tatsuta~\cite{berardiEquivalenceInductiveDefinitions2017,berardiEquivalenceIntuitionisticInductive2017}
% instead rely on Ramsey-style order-theoretic principles which can be
% formalised in Heyting arithmetic. Both
% approaches involve the formalisation of complex mathematical
% concepts in first- and second-order arithmetic.
% % They are thus unlikely
% % to transfer to non-arithmetical settings and, at least, challenging
% % to lift to stronger systems of arithmetic.

% % - Others: Unfolding, less demanding on underlying logic
% The literature contains other translation methods, although not applied to
% arithmetical theories. Of significance to this article is a method first
% employed by Sprenger and Dam~\cite{sprengerStructureInductiveReasoning2003a} for
% first-order logic with fixed points $\mFOL$. This approach unfolds a cyclic
% proof until inductive hypotheses can simply be `read off' the proof's structure.
% The inductive proof resulting from this procedure thus naturally corresponds to
% an unfolding of the initial cyclic proof.

% % What are you trying to do? Give a high-level description  without using any jargon.
% % - Apply unfolding approach to HA (and PA).
% In this article, we adapt Sprenger and Dam's `unfolding' method to present a proof translation between
% the cyclic and inductive proof system for Heyting arithmetic.
% The strategy can be adapted to Peano
% arithmetic with minor adjustments.
% Our construction of the inductive hypothesis in the translated proof refines
% those employed by Sprenger and Dam, which allows recovery Das' logical
% complexity result: In cyclic $\PA$, proofs with $\Pi_n$ cycles correspond to
% $\IP_{n + 1}$-proofs. We obtain a similar result for $\HA$.
% % What is new in your approach? Describe your method with sufficient details to enable a view.
% % - Give A-labelled proof system -> allows more explicit representation of IH
% As part of the proof, we introduce a novel syntactic representation of the
% cyclic proofs resulting from the unfoldings called
% \emph{stack-controlled cyclic proofs}. These allow a clear description the
% process of extracting invariants from such proofs
% and thus lead to a simpler argument for
% the translation from cyclic proofs to inductive proofs.
% % How will it contribute to other areas more broadly speaking
% % What difference does the result make? What are the shortcomings?
% % - A-labelled proof systems more widely applicable and seem suitable to cyclic
% %   proof theory investigations
% While we only present a stack-controlled system for Heyting arithmetic, the
% notion readily generalises to cyclic proof systems for other logics.
% %   - Method likely more widely applicable
% Furthermore, we provide detailed summaries of previous approaches to translating
% cyclic proof systems for Peano or Heyting arithmetic to inductive proof systems.
% % This should provide a good overview of the state-of-the-art of such methods.

% \paragraph{Overview}
% \Cref{sec:cha} introduces the inductive and cyclic
% proof systems for Heyting arithmetic.
% \Cref{sec:sha} presents \emph{stack-labelled} proofs of cyclic Heyting
% arithmetic, a proof system designed to ease the translation.
% In \Cref{sec:translation}, we present the translation of cyclic, stack-labelled
% proofs to plain, inductive proofs.
% The full translation result is obtained in \Cref{sec:combinatorics} by proving that every proof
% in cyclic Heyting arithmetic can be unfolded into a stack-labelled proof.
% We conclude in \Cref{sec:conclusion} by extending the result to Peano
% arithmetic, analysing the translation in terms of logical and proof size
% complexity and discussing the wider applicability of the method presented in
% this article.
% % and a discussion of other applications of
% %the Sprenger and Dam method in the literature (\Cref{sec:related-work}).

% \section{Cyclic Heyting Arithmetic}
% \label{sec:cha}

% The term and formula language of first-order arithmetic is given below, fixing a
% countable set of variables $\var$. Write $\FV(\varphi)$
% for the set of \emph{free variables} which occur unbound in a formula $\varphi$.
% \begin{align*}
%   s,t \in \term \langeq~& x ~|~ 0 ~|~ S s ~|~ s + t ~|~ s \cdot t  & x \in \var \\
%   \varphi, \psi \in \form \langeq~& s = t ~|~ \bot ~|~ \varphi \wedge \psi ~|~ \varphi \vee \psi ~|~ \varphi \to \psi ~|~ \forall x.\varphi ~|~ \exists x. \varphi
% \end{align*}
% We define $\neg \varphi \coloneq \varphi \to \bot$ and $\top \coloneq \bot \to \bot$.
% Denote by $[t / x]$ the usual \emph{substitution operation}, substituting the term
% $t$ into all free occurrences of the variable $x$ in a term or formula. This is
% a partial operation, $\varphi[t / x]$ being undefined if the free variables in
% $t$ are not distinct from the bound variables in $\varphi$. Henceforth, writing
% $\varphi[t / x]$ implicitly asserts that the substitution is defined.
% We sometimes write $\varphi(x)$ as a shorthand for
% $\varphi[x / y]$ for an unspecified variable $y \neq x$ if $x \not\in
% \FV(\varphi)$. By analogy, the notation $\varphi(t)$ in the vicinity of
% $\varphi(x)$ then denotes $\varphi[t / y]$.

% We begin by recalling the inductive proof system for Heyting arithmetic.

% \begin{definition}[Heyting arithmetic]
%   The \emph{sequents} of \emph{Heyting arithmetic} are expressions $\Gamma \sdash \delta$ where
%   $\Gamma$ is a finite set and $\delta$ a singular formula of first-order arithmetic.
%   Write $\Gamma, \varphi$ for
%   $\Gamma \cup \{\varphi\}$ and $\Gamma, \Gamma'$ for $\Gamma \cup \Gamma'$.
%   The \emph{derivation rules} of $\HA$ comprise of the following choice of standard rules for
%   intuitionistic first-order logic with equality,
%   \begin{mathpar}
%     \inference[\RAx]{}{\Gamma{}, \delta{} \sdash{} \delta}

%     \inference[$\to$L]{\Gamma{}, \varphi{} \sdash{} \delta{} \quad \Gamma{} \sdash{}
%       \psi{}}{\Gamma{}, \varphi{} \to \psi{} \sdash{} \delta}

%     \inference[$\to$R]{\Gamma{}, \varphi{} \sdash{} \psi{}}{\Gamma{} \sdash{} \varphi{} \to \psi{}}

%     \inference[$\wedge$L]{\Gamma{}, \varphi{}, \psi{} \sdash{} \delta{}}{\Gamma{},
%       \varphi{} \wedge{} \psi{} \sdash \delta{}}

%     \inference[$\wedge$R]{\Gamma \sdash \varphi \quad \Gamma \sdash \psi}{\Gamma \sdash \varphi \wedge
%       \psi}

%     \inference[$\vee$L]{\Gamma, \varphi \sdash \delta \quad \Gamma, \psi \sdash \delta}{\Gamma, \varphi \vee{}
%       \psi \sdash \delta}

%     \inference[$\vee$R]{\Gamma{} \sdash \varphi_i}{\Gamma{} \sdash
%       \varphi_1 \vee{} \varphi_2}

%     \inference[$\forall$L]{\Gamma, \varphi[t / x] \sdash \delta}{\Gamma, \forall x.\varphi \sdash \delta}

%     \inference[$\forall$R]{\Gamma \sdash \varphi \quad x \not\in
%       \FV(\Gamma, \forall x. \varphi)}{\Gamma
%       \sdash \forall x.\varphi}

%     \inference[$\exists$L]{\Gamma, \varphi \sdash \delta \quad x \not\in
%       \FV(\Gamma, \exists x. \varphi, \delta)}{\Gamma, \exists x. \varphi \sdash \delta}

%     \inference[$\exists$R]{\Gamma \sdash \varphi[t / x]}{\Gamma \sdash \exists x. \varphi}

%     \inference[$\bot$L]{}{\Gamma, \bot \sdash \delta}

%     \inference[$=$L]{\Gamma[x / y] \sdash \delta[x / y] \quad x,
%       y \not\in \FV(s, t)}{\Gamma[s / x, t / y], s = t \sdash \delta[s / x, t
%       / y]}

%     \inference[$=$R]{}{\Gamma \sdash t = t}
%   \end{mathpar}
%   with the following two structural rules,
%   \begin{mathpar}
%     \inference[\RWk]{\Gamma \sdash \delta}{\Gamma, \Gamma' \sdash \delta}
  
%     \inference[\RCut]{\Gamma \sdash \varphi \quad \Gamma, \varphi \sdash \delta}{\Gamma \sdash \delta}
%   \end{mathpar}
%   the following axiomatic sequents which characterise the function symbols of
%   first-order arithmetic,
%   \begin{align*}
%     & \sdash 0 \neq S t & S s = S t & \sdash s = t && \sdash s + 0 = 0 \\
%     &\sdash s + St = S(s + t) & & \sdash s \cdot 0 = 0 && \sdash s \cdot St = (s \cdot t) + s
%   \end{align*}
%   and the axiomatic sequents of \emph{induction}, for any formula $\varphi$,
%   \[ \varphi(0), \forall x. \varphi(x) \to \varphi(Sx) \sdash \varphi(s) . \]

%   Formally, a proof in Heyting arithmetic is
%   a pair $\pi = (T, \rho)$ consisting of a finite tree $T$ and a function
%   $\rho : \Nds(T) \mapsto \RR$, assigning instances of derivation rules to nodes of
%   $T$. We define $\lambda : \Nds(T) \mapsto \Seq$ to map a $t \in \Nds(T)$ to
%   the conclusion of $\rho(t)$. If $t \in \Nds(T)$ and $\rho(t)$ is a rule with $n$ premises $\Gamma_1 \sdash
%   \delta_1, \ldots, \Gamma_n \sdash \delta_n$ then $\Chld(t)$ must be $t_1,
%   \ldots, t_n$ such that $\lambda(t_i) = \Gamma_i \sdash
%   \delta_i$. We consider axiomatic sequents as premiseless derivation rules.
%   We call the sequent $\lambda(r)$ labelling the root of $T$ the \emph{endsequent}.
%   If a proof with endsequent $\Gamma \sdash \delta$
%   exists then $\Gamma \sdash \delta$ is \emph{provable in Heyting arithmetic},
%   denoted by $\HA \vdash \Gamma \sdash \delta$.
% \end{definition}

% Recall that the order $x < y$ can be defined as $\exists z. y = x + Sz$ and $x
% \leq y$ as $x < Sy$. Further, we employ the shorthands $\forall x < t. \varphi
% \coloneq \forall x.~x < t \to \varphi$ and $\exists x < t. \varphi \coloneq
% \exists x.~x < t \wedge \varphi$ for terms $t$.
% The
% following admissible rules and structural property are relied on by the
% translation given in \Cref{sec:translation}.

% \begin{lemma}\label{lem:rind}
%   The following derivation rules are admissible in $\HA$
%   for $x \not\in \FV(\Gamma, \varphi)$.
%   \begin{mathpar}
%     \inference[$\RInd$]{\Gamma, \forall y < x.~\varphi[y / x] \sdash \varphi}{\Gamma \sdash \forall x. \varphi}

%     \inference[$\RCase_x$]{\Gamma(0) \sdash \delta(0) \quad \Gamma(Sx) \sdash
%       \delta(Sx)}{\Gamma(x) \sdash \delta(x)}

%     \inference[$\forall$L*]{\Gamma, \varphi[s / x] \sdash \delta}{\Gamma, s < t,
%       \forall x < t.\varphi \sdash \delta}

%     \inference[$\forall$R*]{\Gamma, x < t \sdash \varphi \quad x \not\in
%       \FV(\Gamma, \forall x < t. \varphi)}{\Gamma \sdash \forall x < t.\varphi}
%   \end{mathpar}
% \end{lemma}

% \begin{fact}\label{lem:ha-ren}
%   If $\HA \vdash \Gamma \sdash \delta$ then $\HA \vdash \Gamma[t / x] \sdash
%   \delta[t / x]$ for any term $t$ and variable $x$.
% \end{fact}

% We proceed by defining a cyclic proof system $\CHA$ for Heyting arithmetic. It eschews
% the induction axioms in favour of \emph{cycles} which allow leaves to
% be considered closed if they are labelled by a sequent that appears elsewhere in
% the proof.

% A \emph{cyclic tree} is a pair $C = (T, \beta)$ consisting of a finite tree $T$ and
% a partial function $\beta \colon \Leaf(T) \to \Inner(T)$ mapping some leaves of
% $T$ to its inner nodes.
% If $t \in \dom(\beta)$ we call it a \emph{bud} and $\beta(t)$ \emph{its companion}.

% \begin{definition}[$\CHA$-pre-proofs]\label{def:pre-proof}
%   The sequents of \emph{cyclic Heyting arithmetic} are the same as those of Heyting
%   arithmetic. The \emph{derivation rules} of $\CHA$ are the same as those of
%   $\HA$ except the induction axioms are replaced by a $\RCase$-rule:
%   \[
%     \inference[$\RCase_x$]{\Gamma(0) \sdash \delta(0)
%       \quad \Gamma(Sx) \sdash \delta(Sx)}{\Gamma(x) \sdash \delta(x)}
%   \]

%   A \emph{pre-proof} of $\CHA$ is a pair $\pi = (C, \rho)$
%   consisting of a cyclic tree $C = (T, \beta)$
%   and a function $\rho : \Nds(T) \setminus \dom(\beta) \to \RR$ labelling all non-bud
%   nodes with derivation rules. We define $\lambda : \Nds(T) \to \Seq$ as follows
%   \[
%     \lambda(t) \coloneq
%     \begin{cases}
%       \text{conclusion of } \rho(t) & t \in \dom(\rho) \\
%       \text{conclusion of } \rho(\beta(t)) & t \in \dom(\beta)
%     \end{cases}.
%   \]
%   If $t \in \Nds(T)$ and $\rho(t)$ is a rule with $n$ premises $\Gamma_1 \sdash
%   \delta_1, \ldots, \Gamma_n \sdash \delta_n$ then $\Chld(t)$ must be $t_1,
%   \ldots, t_n$ such that $\lambda(t_i) = \Gamma_i \sdash \delta_i$.
% \end{definition}

% The structures defined by \Cref{def:pre-proof} are called pre-proofs, rather than
% proofs, as they need not be sound. Such is the case for the pre-proof of
% $0 \neq 1$ below, the two nodes marked with $\star$ forming a cycle.

% \begin{comfproof}
%   \AXC{$\sdash 0 + 0 = 0$}
%   \AXC{$\sdash 0 \neq 1~\star$}
%   \RightLabel{\small\RWk}
%   \UIC{$0 + 0 = 0 \sdash 0 \neq 1$}
%   \RightLabel{\small\RCut}
%   \BIC{$\sdash 0 \neq 1~\star$}
% \end{comfproof}

% Cyclic Heyting arithmetic, like most cyclic proof systems, thus requires an
% additional \emph{soundness condition} delineating between proofs and mere
% pre-proofs. The soundness condition we give below is a so-called \emph{global
%   trace condition}. In \Cref{sec:induction-orders}, we discuss induction orders
% which are an alternative soundness condition for $\CHA$.

% \begin{definition}[$\CHA$-proofs]\label{def:cha-proof}
%   Let $\pi = (C, \rho)$ be a pre-proof.
%   Consider an infinite branch $(\Gamma_i \sdash \delta_i)_{i \in \omega}$
%   through $\pi$.
%   A variable $x$ is said to have \emph{a
%     trace along} $(\Gamma_i \sdash \delta_i)_{i \in \omega}$ if there exists an
%    $n \in \omega$ such that $x \in \FV(\Gamma_i, \delta_i)$ for all $i \geq n$.
%    Such a trace on $x$ is said to be \emph{progressing} if it passes through
%    instances of the $\RCase_x$-rule (for the same variable \( x \)) infinitely often.

%   A pre-proof $\pi$ is a \emph{proof} in cyclic Heyting arithmetic if for every
%   infinite branch through $\pi$ there exists a variable which has a progressing trace along
%   it. $\CHA \vdash \Gamma \sdash \delta$ denotes provability in $\CHA$.
% \end{definition}

% To illustrate the global trace condition, we prove that $\CHA$ is
% sound over the standard model of the natural numbers $\omega$. 
% For $V \subseteq \var$, an assignment is a function $\rho : V \to \omega$ mapping each 
% first-order variable in $V$ to a natural number.
% The satisfaction relation $\rho \vDash \Gamma \sdash \delta$, for an assignment
% \( \rho \) with $\dom(\rho) \supseteq \FV(\Gamma, \delta)$ and sequent $\Gamma \sdash
% \delta$, is
% defined in the usual manner.
% In the following, \( \omega \vDash \Gamma \sdash \delta \) expresses that for
% all suitable assignments \( \rho \), we have \( \rho \vDash \Gamma \sdash \delta \).

% \begin{proposition}[Soundness]\label{lem:sound}
%   If $\CHA \vdash \Gamma \sdash \delta$ then $\omega \vDash \Gamma \sdash \delta$.
% \end{proposition}
% \begin{proof}
%   Suppose $\pi$ was a $\CHA$-proof of $\Gamma \sdash \delta$ and suppose, towards
%   contradiction, that there was an assignment $\rho : \FV(\Gamma,\delta) \to \omega$ such that
%   $\rho \not\vDash \Gamma \sdash \delta$. Consider the $\CHA$-rule applied
%   to the root of $\pi$, deriving $\Gamma \sdash \delta$: As the rule is sound, it must have
%   a premise $\Gamma_1 \sdash \delta_1$ for which there exists an assignment
%   $\rho_1$ with $\rho_1 \not\vDash \Gamma_1 \sdash \delta_1$. Note that
%   $\rho_1$ can always be chosen to extend $\rho$, except for the variable $x$ if the last applied rule
%   is $\RCase_x$ and $\rho(x) > 0$, in which case $\rho_1(x) = \rho(x) - 1$. This argument can
%   be iterated infinitely, following the
%   cycles `back down' the proof when reaching buds, because the
%   contradicted premise of a derivation rule can never be an axiom.
%   Thus, iterating this
%   argument yields an infinite branch $(\Gamma_i \sdash \delta_i)_{i \in \omega}$
%   through $\pi$ and a sequence $(\rho_i)_{i \in \omega}$ of contradicting
%   assignments. Then there exists a progressing $x$-trace along $(\Gamma_i \sdash
%   \delta_i)_{i \in \omega}$ and by the aforementioned relation of the $(\rho_i)_{i \in
%     \omega}$, the sequence $(\rho_i(x))_{i \in \omega}$ in $\omega$ never
%   increases but decreases whenever $\RCase_x$ is passed, which takes place
%   infinitely often. Such an infinitely decreasing sequence contradicts the
%   well-foundedness of $\omega$.
% \end{proof}

% The system $\CHA$ is a variation of the cyclic proof system for Heyting
% arithmetic considered by Berardi and Tatsuta
% in~\cite{berardiEquivalenceIntuitionisticInductive2017}. Their system is an
% extension of Brotherston's cyclic proof system for first-order logic with
% inductive definitions~\cite{brotherstonSequentCalculusProof2006} with the usual
% axiomatisations of $s + t, s \cdot t$ and $S t$. They employ a
% unary inductive predicate $N(t)$ to assert that the term $t$ denotes a natural
% number, their $\RCase$-rule and trace condition being defined in terms of the
% predicate. As in our system every term denotes a natural number, we have chosen
% to not include the predicate $N t$ and adjust the $\RCase$-rule accordingly.
% A crucial difference to Berardi and Tatsuta's system, which we rely on
% throughout the article, is that traces only ever follow a single variable,
% without branching or rejoining of traces. This makes the notion of `progress' in
% our system much simpler, easing the combinatorics of \Cref{sec:combinatorics}.

% $\HA$ and $\CHA$ prove the same theorems. The direction from $\HA$ to $\CHA$
% follows from the fact that induction can be simulated by the use of cycles in
% $\CHA$, as is shown below. The opposite direction is much more difficult to
% prove and the subject of the remaining sections of this article.

% \begin{theorem}
%   If $\HA \vdash \Gamma \sdash \delta$ then $\CHA \vdash \Gamma \sdash \delta$.
% \end{theorem}
% \begin{proof}
%   We begin by showing that $\CHA$ proves every instance $\varphi(0), \forall x.
%   \varphi(x) \to \varphi(Sx) \sdash \varphi(s)$ of the induction scheme.
%   Take $\ol{\varphi}$ to be a shorthand for the left-hand side of the induction scheme. The two
%   sequents marked with a $\star$ form a cycle.
%   \begin{comfproof}
%     \AXC{}
%     \LSC{\textsc{Ax}}
%     \UIC{$\ol{\varphi} \sdash \varphi(0)$}
%     \AXC{$\ol{\varphi} \sdash \varphi(x) ~~\star$}
%     \AXC{}
%     \RSC{\textsc{Ax}}
%     \UIC{$\ol{\varphi}, \varphi(S\,x) \sdash \varphi(S\,x)$}
%     \RSC{$\forall$L,$\to$L}
%     \BIC{$\ol{\varphi} \sdash \varphi(S\,x)$}
%     % \insertBetweenHyps{\hskip 0pt}
%     \LSC{$\textsc{Case}_x$}
%     \BIC{$\ol{\varphi} \sdash \varphi(x) ~~\star$}
%     \LSC{$\forall$R}
%     \UIC{$\ol{\varphi} \sdash \forall x. \varphi(x)$}
%     \AXC{}
%     \RSC{\textsc{Ax}}
%     \UIC{$\ol{\varphi}, \varphi(s) \sdash \varphi(s)$}
%     \RSC{$\forall$L}
%     \UIC{$\ol{\varphi}, \forall x. \varphi(x) \sdash \varphi(s)$}
%     \insertBetweenHyps{\hskip -6ex}
%     \LSC{\textsc{cut}}
%     \BIC{$\ol{\varphi} \sdash \varphi(s)$}
%   \end{comfproof}
%   There exists precisely one infinite branch through the pre-proof above, following the
%   $\star$-cycle infinitely. This infinite branch has a progressing $x$-trace starting in
%   the premise of $\forall$R. Thus, the pre-proof above constitutes a proof.

%   Towards the original claim, suppose $\HA \vdash \Gamma \sdash \delta$. A
%   $\HA$-proof $\pi$ witnessing this fact differs from a $\CHA$-pre-proof only by the
%   fact that some of its leaves may be labelled with instances of the induction
%   schemes. Thus, a $\CHA$-pre-proof $\pi'$ of $\Gamma \sdash \delta$ may be obtained by
%   `grafting' copies of the $\CHA$-proof given above onto each such leaf of
%   $\pi$. As there are no cycles in $\pi$, every infinite branch of $\pi'$
%   must follow the $\star$-cycle of one of the `grafted on' induction scheme
%   proofs and thus has a successful trace. Hence $\pi'$ is a $\CHA$-proof
%   witnessing $\CHA \vdash \Gamma \sdash \delta$.
% \end{proof}

% \section{Stack-labelled Heyting arithmetic}
% \label{sec:sha}

% Roughly, the translation of $\CHA$-proofs into $\HA$-proofs can be split into
% two steps: First $\CHA$-proofs are arranged into a normal form, codified by an intermediate proof system
% $\AHA$. As the second step, $\AHA$-proofs are translated into $\HA$ proofs.
% All `logical content' of the translation is limited to the second step, the
% first step being pure combinatorics. We thus begin by presenting the second
% step of the translation, deferring the combinatorial arguments to
% \Cref{sec:combinatorics}. This section introduces the system $\AHA$ of
% \emph{stack-labelled Heyting arithmetic}. In \Cref{sec:translation}, we present
% the translation of $\AHA$-proofs to $\HA$-proofs.

% The sequents of $\AHA$ are expressions $\Lambda \mid \Gamma \sdash \delta$,
% consisting of an $\HA$ sequent $\Gamma \sdash \delta$ and an ordered list
% $\Lambda$ of \emph{companion labels}, which is treated as a stack. We denote the
% empty list by $\varepsilon$. Intuitively,
% the companion labels in $\Lambda$ collect the sequents of \emph{companions} to
% which one may `cycle back' at the current point of the proof.
% Formally, a \emph{companion label} is an expression of the shape $x^\bullet \mapsto \Gamma
% \sdash \delta$ for a $\HA$-sequent $\Gamma \sdash \delta$, a variable $x \in
% \FV(\Gamma, \delta)$ and where $x^\bullet \in \{x^+, x^-\}$. We write
% $\var(\Lambda)$ for the set $\{x \mid (x^\bullet \mapsto \Gamma \sdash \delta)
% \in \Lambda\}$ of variables with associated companion labels in $\Lambda$.
% Returning to the intuitive reading of companion labels, $x^+
% \mapsto \Gamma \sdash \delta$ in $\Lambda$ indicates that the current point of
% the proof may `cycle back' to a companion node with $\HA$-sequent $\Gamma \sdash
% \delta$ because the companion's associated variable $x$ has been subject to a
% $\RCase$-rule between the companion and the current point in the proof.
% Conversely, a companion label $x^- \mapsto \Gamma \sdash \delta$ indicates that
% $x$ has not yet been subject to $\RCase$ and thus no cycle may be formed yet.

% A $\AHA$-sequent $\Lambda \mid \Gamma \sdash \delta$ is \emph{well-formed} if
% for every $x^\bullet \mapsto \Delta \sdash \gamma \in \Lambda$ we have $x \in
% \FV(\Gamma, \delta)$. Going forward, we assume that every $\AHA$-sequent
% $\Lambda \mid \Gamma \sdash \delta$ under consideration is well-formed, unless
% stated otherwise.

% Before presenting the rules of the calculus \( \AHA \), we fix further
% operations on stacks of companion labels. Given a stack $\Lambda$ and $n \in
% \omega$, we define $\Lambda \uh n$ to be the longest prefix $\Lambda'$ of
% $\Lambda$ with $\abs{\Lambda'} \leq n$. Given a
% variable \( x \), we define $\Lambda^{+x}$ to be the result of replacing in
% $\Lambda$ all occurrences of $x^-$ by $x^+$:
% \[
%   \varepsilon^{+x} \coloneq \varepsilon
%   \qquad
%   (y^\bullet \mapsto \Gamma \sdash \delta; \Lambda)^{+x} \coloneq
%   \begin{cases}
%      x^- \mapsto \Gamma \sdash \delta; \Lambda^{+x}, &\text{if \( y = x \),}
%     \\
%     y^\bullet \mapsto \Gamma \sdash \delta; \Lambda^{+x} , &\text{otherwise.}
%   \end{cases}
% \]

% \begin{definition}
%   The \emph{derivation rules} of $\AHA$ are of split into three categories:
%   %
%   \begin{description}
% 	\item [Logical rules,] which do not interact with companion labels:
%   \begin{mathpar}
%     \inference[\RAx]{}{\varepsilon \mid \Gamma, \delta \sdash \delta}

%     \inference[$\to$L]{\Lambda \mid \Gamma{}, \varphi{} \sdash{} \delta{} \quad \Lambda \mid \Gamma{} \sdash{}
%       \psi{}}{\Lambda \mid \Gamma{}, \varphi{} \to \psi{} \sdash{} \delta}

%     \inference[$\to$R]{\Lambda \mid \Gamma{}, \varphi{} \sdash{} \psi{}}{\Lambda \mid \Gamma{} \sdash{} \varphi{} \to \psi{}}
    
%     \\

%     \inference[$\wedge$L]{\Lambda \mid \Gamma{}, \varphi{}, \psi{} \sdash{} \delta{}}{\Lambda \mid \Gamma{},
%       \varphi{} \wedge{} \psi{} \sdash \delta{}}

%     \inference[$\wedge$R]{\Lambda \mid \Gamma \sdash \varphi \quad \Lambda \mid \Gamma \sdash \psi}{\Lambda \mid \Gamma \sdash \varphi \wedge
%       \psi}

%     \inference[$\vee$L]{\Lambda \mid \Gamma, \varphi \sdash \delta \quad \Lambda \mid \Gamma, \psi \sdash \delta}{\Lambda \mid \Gamma, \varphi \vee{}
%       \psi \sdash \delta}

%     \inference[$\vee$R]{\Lambda \mid \Gamma{} \sdash \varphi_i}{\Lambda \mid \Gamma{} \sdash
%       \varphi_1 \vee{} \varphi_2}
    
%     \\

%     \inference[$\forall$L]{\Lambda \mid \Gamma, \varphi[t / x] \sdash \delta}{\Lambda \mid \Gamma, \forall x.\varphi \sdash \delta}

%     \inference[$\forall$R]{\Lambda \mid \Gamma \sdash \varphi \quad x \not\in
%       \FV(\Lambda \mid \Gamma, \forall x. \varphi)}{\Lambda \mid \Gamma
%       \sdash \forall x.\varphi}

%     \inference[$\exists$L]{\Lambda \mid \Gamma, \varphi \sdash \delta \quad x \not\in
%       \FV(\Lambda \mid \Gamma, \exists x. \varphi, \delta)}{\Lambda \mid \Gamma, \exists x. \varphi \sdash \delta}

%     \inference[$\exists$R]{\Lambda \mid \Gamma \sdash \varphi[t / x]}{\Lambda \mid \Gamma \sdash \exists x. \varphi}

%     \inference[$=$L]{\Lambda \mid \Gamma[x / y] \sdash \delta[x / y] \quad x, y \not\in \FV(s, t)}{\Lambda \mid \Gamma[s / x, t / y], s = t \sdash \delta[s / x, t / y]}

%     \inference[$=$R]{}{\Lambda \mid \Gamma \sdash t = t}
    
% 	\inference[$\bot$L]{}{\Lambda \mid \Gamma, \bot \sdash \delta}

%   \inference[\RWk]{\Lambda \mid \Gamma \sdash \delta}{\Lambda \mid \Gamma, \Gamma' \sdash \delta}

%   \inference[\RCut]{\Lambda \mid \Gamma \sdash \varphi \quad \Lambda \mid \Gamma, \varphi \sdash \delta}{\Lambda \mid \Gamma \sdash \delta}
%   \end{mathpar}
%   \item [Arithmetical axioms] of the form:
%   \begin{align*}
%     \varepsilon \mid \Gamma &\sdash 0 \neq S t & \varepsilon \mid \Gamma , S s = S t & \sdash s = t & \varepsilon \mid \Gamma &\sdash s + 0 = 0 \\
%     \varepsilon \mid \Gamma &\sdash s + St = S(s + t) & \varepsilon \mid \Gamma &\sdash s \cdot 0 = 0 &\varepsilon \mid \Gamma &\sdash s \cdot St = (s \cdot t) + s
%   \end{align*}
%   \item [Label rules] which manipulate companion labels:
%   \begin{mathpar}
%     \inference[$\RComp$]{\Lambda; (x^- \mapsto \Gamma \sdash \delta) \mid \Gamma \sdash \delta}{\Lambda \mid \Gamma \sdash \delta} 

%     \inference[$\RBud$]{}{\Lambda; (x^+ \mapsto \Gamma \sdash \delta) \mid \Gamma \sdash \delta}

%     \inference[$\RDrop$]{\Lambda \mid \Gamma \sdash \delta}{\Lambda; \Lambda' \mid \Gamma \sdash \delta}

%     \inference[$\RCase_{x}$]{\Lambda \mid \Gamma(0) \sdash \delta(0)
%       \qquad \Lambda^{+x} \mid \Gamma(S\,x) \sdash \delta(S\,x)}{\Lambda \mid \Gamma(x) \sdash \delta(x)}
%   \end{mathpar}
% \end{description}

%   We only consider finite, non-cyclic derivations in $\AHA$ to be proofs. That is,
%   a proof in $\AHA$ is
%   a pair $\pi = (T, \rho)$ consisting of a finite tree $T$ and a function
%   $\rho : \Nds(T) \mapsto \RR$, assigning instances of $\AHA$-derivation rules to nodes of
%   $T$. If $t \in \Nds(T)$ and $\rho(t)$ is a rule with $n$ premises $\Lambda_1
%   \mid \Gamma_1 \sdash
%   \delta_1, \ldots, \Lambda_n \mid \Gamma_n \sdash \delta_n$ then $\Chld(t)$ must be $t_1,
%   \ldots, t_n$ such that the conclusion of the $\rho(t_i)$ is $\Lambda_i \mid \Gamma_i \sdash
%   \delta_i$. We consider axiomatic sequents as premiseless derivation rules.
%   Every rule but $\RDrop$
%   may only applied to derive sequents well-formed sequents.
%   We call the conclusion of the rule labelling the root of $T$ the \emph{endsequent}.
%   If such a
%   proof with endsequent $\varepsilon \mid \Gamma \sdash \delta$
%   exists then $\Gamma \sdash \delta$ is \emph{provable in stack-controlled Heyting arithmetic},
%   denoted $\AHA \vdash \Gamma \sdash \delta$.
% \end{definition}

% Formally, $\AHA$ is not a cyclic proof system because its derivations are
% non-cyclic trees. However, through the $\RComp$- and $\RBud$-rules, it retains a
% `cyclic character'. In this sense, it can be considered as a mid-point between
% the fully cyclic system $\CHA$ and the fully non-cyclic system $\HA$.

% The exception stipulating that $\RDrop$ may derive ill-formed sequents is in
% place to cover for a situation which may arise in $\RCase$-applications.
% Suppose $\Lambda = \Lambda_0 ; (x^\bullet \mapsto \Delta \sdash
% \gamma); \Lambda_1$ where $x \not\in \var(\Lambda_0)$.
% Consider the following application of
% $\RCase_x$:
% \[
%     \inference[$\RCase_{x}$]{\Lambda \mid \Gamma(0) \sdash \delta(0)
%       \qquad \Lambda^{+x} \mid \Gamma(S\,x) \sdash \delta(S\,x)}{\Lambda \mid \Gamma(x) \sdash \delta(x)}
% \]
% Then $x \not\in \FV(\Gamma(0), \delta(0))$ in the left-hand premise, but
% $\Lambda$ still contains the label $x^\bullet \mapsto \Delta \sdash \gamma$,
% making the left-hand premise an ill-formed $\AHA$-sequent. This may be rectified
% by an application of the $\RDrop$-rule, reducing $\Lambda$ to $\Lambda_0$,
% yielding a well-formed sequent, at which point other derivation rules may be
% applied again.

% The following key theorem is proven in \Cref{sec:combinatorics}:

% \begin{restatable}{theorem}{chatoaha}\label{lem:cha-to-aha}
%   If $\CHA \vdash \Gamma \sdash \delta$ then $\AHA \vdash \Gamma \sdash \delta$.
% \end{restatable}

% \section{Translating from $\AHA$ to $\HA$}
% \label{sec:translation}

% This section describes a translation of $\AHA$-proofs into $\HA$-proofs.
% The general strategy is to transform
% each $\AHA$-derivation $\pi$ of $\Lambda \mid \Gamma \sdash \delta$ into a
% $\HA$-derivation of $\HH_\Lambda(\pi), \Gamma \sdash \delta$ where
% $\HH_\Lambda(\pi)$ is a formula `computed from the data of $\Lambda$ and $\pi$'
% called the \emph{induction invariant}. While this general strategy is similar to
% the approach of Sprenger and
% Dam~\cite{sprengerStructureInductiveReasoning2003a}, some of the details
% differ: Their method would yield a sequent $\HH_\Lambda, \Gamma \sdash \delta$
% where $\HH_\Lambda$ is a set of inductive hypotheses `computed from the data of
% $\Lambda$'. By also taking into account the shape of $\pi$ in computing
% $\HH_\Lambda(\pi)$, we are able to combine the inductive hypotheses into a
% single formula, which allows us to control the inductive complexity of the
% resulting $\HA$-proof, mirroring a result obtained by
% Das~\cite{dasLogicalComplexityCyclic2020} for cyclic $\PA$.

% The construction of the inductive hypothesis $\HH_\Lambda(\pi)$ can be split
% into multiple steps. First, we define $I_\Lambda$, which should be considered
% the `inductive hypothesis of the companion label on the top of the stack $\Lambda$'.
% Thus, let $\Lambda$ be non-empty, i.e.\ $\Lambda \coloneq \Lambda'; x^\bullet \mapsto \Gamma \sdash \delta$.
% We separate the free variables of $\Gamma \sdash \delta$, other than $x$, into
% $\vec{v} = \var(\Lambda) \setminus \{x\}$ and
% $\vec{w} = \FV(\Gamma, \delta) \setminus (\var(\Lambda) \cup \{x\})$. Then
% define
% \[
%   I_\Lambda \coloneq
%   \begin{cases}
%     \forall x' \,<\, x.~\widehat{I}_\Lambda [x' / x] & \text{ if } \bullet = - \\
%     \forall x' \,\leq\, x.~\widehat{I}_\Lambda [x' / x] & \text{ if } \bullet = +
%   \end{cases} \quad \text{ where } \quad
%   \widehat{I}_\Lambda \coloneq \forall \vec{u} \leq \vec{v}. \forall \vec{w}.~(\bigwedge
%   \Gamma \to \delta)[\vec{u} / \vec{v}].
% \]
% \begin{proposition}\label{lem:end-lem}
%   \
%   \begin{enumerate}[(i)]
%   \item $\FV(I_\Lambda) = \FV(\widehat{I}_\Lambda) = \var(\Lambda)$
%   \item $\HA \vdash I_\Lambda, x \leq y \sdash I_\Lambda[x / y]$ for any
%     $\Lambda$ and $x,y \in \var$
%   \item $\HA \vdash I_\Lambda[Sx / x] \sdash I_{\Lambda^{+x}}$ for $x \in \FV(I_\Lambda)$
%   \end{enumerate}
% \end{proposition}

% The inductive hypothesis $\HH_\Lambda(\pi)$ is defined in terms of the recursive
% function $H_n(\pi)$ which calculates the inductive hypothesis for a derivation
% $\pi \vdash \Lambda \mid \Gamma \sdash \Delta$ where $n \leq \abs{\Lambda}$.
% Intuitively, $H_n(\pi)$ computes `the inductive hypothesis for the first $n$
% labels of $\Lambda$ along $\pi$'. We thus fix $\HH_\Lambda(\pi) \coloneq H_{\abs{\Lambda}}(\pi)$.
% \newsavebox{\budpt}
% \begin{lrbox}{\budpt}% Store prooftree in \mypt
%   \begin{varwidth}{\linewidth}
%     \begin{comfproof}
%       \AXC{}
%       \LSC{$\RBud$}
%       \UIC{$\Lambda; (x^+ \mapsto \Gamma \sdash \delta) \mid \Gamma \sdash \delta$}
%     \end{comfproof}
%   \end{varwidth}
% \end{lrbox}
% \newsavebox{\droppt}
% \begin{lrbox}{\droppt}% Store prooftree in \mypt
%   \begin{varwidth}{\linewidth}
%     \begin{comfproof}
%       \AXC{$\pi'$}
%       % \noLine
%       % \UIC{$\Lambda \mid \Gamma \sdash \delta$}
%       \LSC{$\RDrop$}
%       \UIC{$\Lambda; \Lambda' \mid \Gamma \sdash \delta$}
%     \end{comfproof}
%   \end{varwidth}
% \end{lrbox}
% \newsavebox{\casept}
% \begin{lrbox}{\casept}% Store prooftree in \mypt
%   \begin{varwidth}{\linewidth}
%     \begin{comfproof}
%       \AXC{$\pi_0$}
%       % \noLine
%       % \UIC{$\Lambda \mid \Gamma(0) \sdash \delta(0) $}
%       \AXC{$\pi_s$}
%       % \noLine
%       % \UIC{$\Lambda^{+x} \mid \Gamma(S\,x) \sdash \delta(S\,x)$}
%       \LSC{$\RCase_x$}
%       \BIC{$\Lambda \mid \Gamma \sdash \delta$} % 'x' already in label "Case_x"
%     \end{comfproof}
%   \end{varwidth}
% \end{lrbox}
% \newsavebox{\unpt}
% \begin{lrbox}{\unpt}% Store prooftree in \mypt
%   \begin{varwidth}{\linewidth}
%     \begin{comfproof}
%       \AXC{$\pi'$}
%       % \noLine
%       % \UIC{$\Lambda' \mid \Gamma' \sdash \delta'$}
%       \LSC{\textsc{Un}}
%       \UIC{$\Lambda \mid \Gamma \sdash \delta$}
%     \end{comfproof}
%   \end{varwidth}
% \end{lrbox}
% \newsavebox{\binpt}
% \begin{lrbox}{\binpt}% Store prooftree in \mypt
%   \begin{varwidth}{\linewidth}
%     \begin{comfproof}
%       \AXC{$\pi_l$}
%       % \noLine
%       % \UIC{$\Lambda_l \mid \Gamma_l \sdash \delta_l$}
%       \AXC{$\pi_r$}
%       % \noLine
%       % \UIC{$\Lambda_r \mid \Gamma_r \sdash \delta_r$}
%       \LSC{\textsc{Bin}}
%       \BIC{$\Lambda \mid \Gamma \sdash \delta$}
%     \end{comfproof}
%   \end{varwidth}
% \end{lrbox}
% \begin{align*}
%   H_0(\pi) & \coloneq \top \\
%   H_n\left(
%     \hspace{-0.5em}\usebox{\budpt}
%   \right) & 
%       \coloneq
%                                                            \begin{cases}
%                                                              I_{\Lambda}
%                                                              & \text{ if } \abs{\Lambda} = n \\
%                                                              \top & \text{ otherwise}
%                                                            \end{cases} \\
%   H_n \left( 
%     \hspace{-0.6em}\usebox{\droppt}
%   \right) & 
%    \coloneq
%                                       H_{\min(\abs{\Lambda}, n)}(\pi)
%                                     \\
%   H_{n} \left(
%     \hspace{-0.6em}\usebox{\unpt}
%   \right) & \coloneq H_{n}(\pi) \\
%   H_{n}\left(
%     \hspace{-0.6em}\usebox{\binpt}
%   \right)& \coloneq H_{n}(\pi_l) \wedge H_{n}(\pi_r) \\
%   H_n \left(
%     \hspace{-0.6em}\usebox{\casept}
%   \right) &
%                                  \coloneq \begin{cases}
%                                    \begin{aligned}
%     \forall x' & \leq x.~(x' = 0 \wedge H_{n}(\pi_0)[x' / x]) \vee \\ & (\exists x''.~x' = Sx'' \wedge H_{n}(\pi_s)[x'' / x]) 
%                                    \end{aligned}
%                                    & \text{ if } x \in \var(\Lambda \uh n) \\
%                                    H_{n}(\pi_0) \wedge H_{n}(\pi_s) & \text{ otherwise}
%                                  \end{cases}
% \end{align*}
% % \begin{mathpar}
% %   H_n \left(
% %     \hspace{-0.2em}\usebox{\casept}
% %   \right)

% %                                  \coloneq \begin{cases}
% %                                    \begin{aligned}
% %     \forall x' & \leq x.~(x' = 0 \wedge H_{n}(\pi_0)[x' / x]) \vee \\ & (\exists x''.~x' = Sx'' \wedge H_{n}(\pi_s)[x'' / x]) 
% %                                    \end{aligned}
% %                                    & \text{ if } x \in \var(\Lambda \uh n) \\
% %                                    H_{n}(\pi_0) \wedge H_{n}(\pi_s) & \text{ otherwise}
% %                                  \end{cases} \\
% % \end{mathpar}
% The clauses $\textsc{Un}$ and $\textsc{Bin}$ are catch-all for unary and binary
% rules not covered by the other clauses.

% Intuitively, for a proof $\pi \vdash \Lambda \mid \Gamma \sdash \delta$, the
% partial inductive hypothesis $H_n(\pi)$ accounts for all instances of the invariants $x
% \mapsto \Gamma' \sdash \delta' \in \Lambda \uh n$ needed to `close the proof'. They are collected by
% recursively traversing the proof tree $\pi$. For example in the $\RBin$-case, the instances needed
% to close the whole proof are precisely those needed to close the left and the right
% premise, hence their conjunction is chosen.

% The most complicated clause of $H_n(\pi)$ is that of the $\RCase$-rule. Observe
% that it is an embodiment of the theorem $\forall y.~y = 0 \vee \exists
% y'.~y = Sy'$ of arithmetic. The variable $x$ in the conclusion of $\pi_S$ thus
% corresponds to $y'$ in the theorem while that same $x$ corresponds to $y$ in the
% conclusion of $\RCase$. The complicated case of the $\RCase$-clause, if $x \in
% \var(\Lambda \uh n)$ and thus possibly $x \in \FV(H_n(\pi))$, accounts for this
% change of reference. However, this alone does not explain the structure of
% $H_n(\pi)$ in this case. There are two more considerations which
% influence it: First, it is important that the sequent $H_n(\pi), x \leq y \sdash
% H_n(\pi)[x / y]$ is provable in $\HA$ (\Cref{lem:ih-lem} (ii) below), for which
% the `downwards closing' with the prefix of $\forall x' \leq x.$ is required.
% Second, we have designed the $\RCase$-clause such that the resulting $H_n(\pi)$
% embodies the aforementioned case-distinction theorem. In the translation of a
% $\AHA$-proof to a $\HA$-proof, this will allow us to translate every application of
% a $\RCase$-rule to a variable in $\var(\Lambda)$ as an `interaction' with
% $\HH_\Lambda(\pi)$ in a straight-forward manner.

% \begin{lemma}\label{lem:ih-lem}
%   \
%   \begin{enumerate}[(i)]
%   \item $\FV(H_n(\pi \vdash \Lambda \mid \Gamma \sdash \delta)) \subseteq \var(\Lambda \uh n)$
%   \item $\HA \vdash H_n(\pi), x \leq y \sdash H_n(\pi)[x / y]$ for any $\pi$,
%     $n$ and $x, y \in \var$.
%   \item Let $\abs{\Lambda} > n$ then $\HA \vdash H_{n}(\pi), I_{\Lambda \uh n +
%       1} \sdash H_{n + 1}(\pi)$.
%   \end{enumerate}
% \end{lemma}
% \begin{proof}
%   \
%   \begin{enumerate}[(i)]
%   \item Follows immediately from \Cref{lem:end-lem} (i) and by scrutinising the
%     definition of $H_n(-)$.
%   \item Proceed by induction on $\pi$. Only 3 cases are of interest because they
%     influence the free variables of $H_\bullet(\pi)$:
%     \begin{description}
%     \item[Bud:] Then $H_n(\pi) = \top$ or $H_n(\pi) = I_{\Lambda}$ for some
%       $\Lambda$. In the latter case, the claim follows directly
%       from \Cref{lem:end-lem} (ii).
%     \item[Pop:] It might be the case that $y \not\in H_n(\pi)$ because the
%       corresponding label is erased by $\RDrop$. In this case, simply close by
%       $\RAx$. Otherwise continue via the IH.
%     \item[$\RCase_z$:] The only interesting case is if $z = y$. In this case,
%       \[
%         H_n(\pi) = \forall y' \leq y.~\underbrace{(y' = 0 \wedge H_n(\pi_0)[y' / y]) \vee
%         (\exists y''.~y' = sy'' \wedge H_n(\pi_s)[y''/ y])}_{\varphi }.
%       \]
%       Then simply derive the following:
%       \begin{comfproof}
%         \AXC{}
%         \LSC{$\RAx$}
%         \UIC{$A, y' \leq y \sdash A$}
%         \LSC{$\forall$L*}
%         \UIC{$\forall y' \leq y. A, y' \leq y \sdash A$}
%         \UIC{$\forall y' \leq y. A, x \leq y, y' \leq x \sdash A$}
%         \LSC{$\forall$R*}
%         \UIC{$\forall y' \leq y. A, x \leq y \sdash \forall y' \leq x. A$}
%       \end{comfproof}
%     \end{description}
%   \item 
%   Per induction on $\pi$.
%   \begin{description}
%   \item[Bud:] If $\abs{\Lambda} \neq n + 1$, this is trivial as then
%     $H_{n + 1}(\pi) = \top$. If $\abs{\Lambda} = n + 1$ then
%     $H_{n + 1} = I_{\Lambda \uh n + 1}$ which follows immediately.
%   \item[Pop:] The only interesting case is if $\Lambda$ is partitioned into $\Lambda_1 ; \Lambda_2$ with $\Lambda_2$
%     being discarded by the rule and $\abs{\Lambda_1} < n + 1$. In
%     this case, $H_{n}(\pi) = H_{\abs{\Lambda_1}}(\pi') =
%     H_{n + 1}(\pi)$ and the proof can be closed with $\RAx$.
%   \item[$\RCase_y$:] The case in which $y \not\in \var(\Lambda \uh n + 1)$ can be
%     treated as the generic binary case. Thus assume $y \in \var(\Lambda \uh n + 1)$. Then
%     $H_{n + 1}(\pi) = \forall y' \leq y.~L \vee R$ with $L =
%     y' = 0 \wedge H_{n + 1}(\pi_0)[y' / y]$ and $R = \exists y''.~y' = sy''
%     \wedge H_{n + 1}(\pi_S)[y'' / y]$. Write $I \coloneq I_{\Lambda \uh n + 1}$
%     and $I^+ \coloneq I_{(\Lambda \uh n + 1)^{+y}}$.
%     We must distinguish two cases:
%     \begin{description}
%     \item[$y \in \var(\Lambda \uh n)$:] Let \( \psi_0 \coloneq y' = 0 \wedge H_{n}(\pi_0)[y' / y] \) and \( \psi_s \coloneq \exists y''.~y' = sy'' \wedge H_{n}(\pi_s)[y'' / y] \). We derive:
%       \begin{scprooftree}{1}
%         \AXC{$\pi_L$}
%         \UIC{$\psi_L, I[y' / y] \sdash L \vee R$}

%         \AXC{$\pi_R$}
%         \UIC{$\psi_s, I[y' / y] \sdash L \vee R$}

%         \LSC{$\vee$L}
%         \BIC{$\psi_0 \vee \psi_s , I[y' / y] \sdash L \vee R$}
%         \LSC{\Cref{lem:end-lem} (ii), $\RWk$}
%         \UIC{$\psi_0 \vee \psi_s , I,  y' \leq y \sdash L \vee R$}
%         \LSC{$\forall$L*}
%         \UIC{$\forall y' \leq y.~\psi_0 \vee \psi_s , I, y' \leq y \sdash L \vee R$}
%         \LSC{$\forall$R*}
%         \UIC{$\forall y' \leq y.~\psi_0 \vee \psi_s , I \sdash \forall y' \leq y. L \vee R$}
%       \end{scprooftree}
%       \vspace{0.5em}
%       where \( \pi_L \) and \( \pi_R \) are, respectively, the proofs
%       \begin{scprooftree}{1}
%           %
%         \AXC{$y' = 0 \sdash y' = 0 $}
%         \AXC{IH of $\pi_0$}
%         \UIC{$H_{n}(\pi_0), I \sdash H_{n + 1}(\pi_0)$}
%         \LSC{\Cref{lem:ha-ren}}
%         \UIC{$H_{n}(\pi_0)[y' / y], I[y' / y] \sdash H_{n + 1}(\pi_0)[y' / y]$}
%         \LSC{$\wedge$R, $\wedge$L, $\RWk$}
%         \BIC{$\psi_0 , I[y' / y] \sdash y' = 0 \wedge H_{n + 1}(\pi_0)[y' / y]$}
%         \LSC{$\vee$R}
%         \UIC{$\psi_0 , I[y' / y] \sdash L \vee R$}
%       \end{scprooftree}
%       %
%       \begin{scprooftree}{1}
        
%         \AXC{$y' = sy'' \sdash y' = sy''$}
%         \AXC{IH of $\pi_s$}
%         \UIC{$H_{n}(\pi_s), I^+ \sdash H_{n + 1}(\pi_s)$}
%         \RSC{\Cref{lem:ha-ren}}
%         \UIC{$H_{n}(\pi_s)[y'' / y], I^+[y'' / y] \sdash H_{n + 1}(\pi_s)[y'' / y] $}
%         \RSC{\Cref{lem:end-lem} (iii)}
%         \UIC{$H_{n}(\pi_s)[y'' / y], I[sy'' / y] \sdash H_{n +
%             1}(\pi_s)[y'' / y] $}
%         \RSC{$\exists$L, $\vee$R, $\exists$R, $\wedge$L, $\wedge$R}
%         \BIC{$\psi_s, I[y' / y] \sdash L \vee R$}
%       \end{scprooftree}
%       \vspace{.5em}
%     \item[$y \in \var(\Lambda \uh n + 1) \setminus \var(\Lambda \uh n)$:] Observe that the proof must
%       have the form:
%       \begin{scprooftree}{0.88}
%         \AXC{$\pi'_0$}
%         \UIC{$\Lambda' \mid \Gamma[0 / y] \sdash \delta[0 / y]$}
%         \RSC{$\pi_0$}
%         \LSC{$\RDrop$}
%         \UIC{$\Lambda_1; y^\bullet \mapsto \Gamma' \sdash \delta'; \Lambda_2 \mid
%           \Gamma[0 / y] \sdash \delta[0 / y]$}
%         \AXC{$\pi_s$}
%         \UIC{$\Lambda_1; y^+ \mapsto \Gamma' \sdash \delta'; \Lambda_2 \mid
%           \Gamma[sy / y] \sdash \delta[sy / y]$}
%         \LSC{$\RCase_y$}
%         \BIC{$\Lambda_1; y^\bullet \mapsto \Gamma' \sdash \delta'; \Lambda_2 \mid
%           \Gamma \sdash \delta$}
%       \end{scprooftree}
%       \vspace{.5em}
%       with $\abs{\Lambda_1} = n$ and $\abs{\Lambda'} \leq n$. Hence,
%       $H_{n}(\pi_0) = H_{\abs{\Lambda'}}(\pi_0') = H_{n + 1}(\pi_0)$ with $y
%       \not\in \FV(H_{\abs{\Lambda'}}(\pi_0'))$, which we shall refer to by $(\star)$.
%       Further, observe that $y \not\in \FV(H_{n}(\pi_S))$.
%       Using the result of the induction hypothesis applied to \( \pi'_0 \) and \( \pi_s \), we derive:
%       \begin{scprooftree}{1}
%         \AXC{$\sdash
%           0 = 0$}
%         \AXC{IH of $\pi'_0$}
%         \UIC{$H_{\abs{\Lambda'}}(\pi_0) \sdash H_{\abs{\Lambda'}}(\pi_0)$}
%         \LSC{$\wedge$R}
%         \BIC{$H_{\abs{\Lambda'}}(\pi_0) \sdash 0 = 0 \wedge H_{\abs{\Lambda'}}(\pi_0)$}
%         \LSC{$(\star)$}
%         \UIC{$H_{n}(\pi_0)[0 / y] \sdash 0 = 0 \wedge H_{n + 1}(\pi_0)[0 / y]$}

%         \AXC{$\pi_0$}
%         \LSC{$\RCase_{y'}, \RWk, \vee$R}
%         \BIC{$H_{n}(\pi_0), H_{n}(\pi_s)[y' / y], I[y' / y] \sdash L
%           \vee R$}
%         \LSC{$y \not\in \FV(H_n(\pi_S))$}
%         \UIC{$H_{n}(\pi_0), H_{n}(\pi_s), I[y' / y], y' \leq y \sdash L
%           \vee R$}
%         \LSC{\Cref{lem:end-lem} (ii)}
%         \UIC{$H_{n}(\pi_0), H_{n}(\pi_s), I, y' \leq y \sdash L
%           \vee R$}
%         \LSC{$\wedge$L, $\forall$R*}
%         \UIC{$H_{n}(\pi_0) \wedge H_{n}(\pi_s), I \sdash \forall y' \leq y. L
%           \vee R$}
%       \end{scprooftree}
%       where \( \pi_0 \) is the derivation 
%       \begin{scprooftree}{1}
%         \AXC{IH of $\pi_s$}
%         \UIC{$H_{n}(\pi_s), I \sdash H_{n + 1}(\pi_s)$}
%         \RSC{\Cref{lem:ha-ren}}
%         \UIC{$H_{n}(\pi_s)[y'' / y], I[y'' / y]
%           \sdash H_{n + 1}(\pi_s)[y'' / y]$}
%         \RSC{(ii)}
%         \UIC{$H_{n}(\pi_s)[sy'' / y], I^+[y'' / y]
%           \sdash H_{n + 1}(\pi_s)[y'' / y]$}
%         \RSC{\Cref{lem:end-lem} (iii)}
%         \UIC{$H_{n}(\pi_s)[sy'' / y], I[sy'' / y]
%           \sdash H_{n + 1}(\pi_s)[y'' / y]$}
%         \RSC{$\exists$R, $\wedge$R, $=$R}
%         \UIC{$H_{n}(\pi_s)[sy'' / y], I[sy'' / y]
%           \sdash \exists z.~sy'' = sz \wedge H_{n + 1}(\pi_s)[z / y]$}
%       \end{scprooftree}
%       \vspace{.5em}
%     \item[Un:] The inductive hypothesis directly yields the desired claim.
%     \item[Bin:] This is easily handled via $\wedge$R and subsequent $\wedge$L,
%       weakenings and application of the corresponding IH.
%     \end{description}
%   \end{description}
% \end{enumerate}
% \end{proof}

% \begin{theorem}\label{lem:key}
%   If $\AHA \vdash \Gamma \sdash \delta$ then $\HA \vdash \Gamma \sdash \delta$.
% \end{theorem}
% \begin{proof}
%   Let $\pi$ be a the derivation of $\varepsilon \mid \Gamma \sdash \delta$ in $\AHA$.
%   We will prove a generalisation of the claim: If $\pi$ is
%   an $\AHA$-derivation of $\Lambda \mid \Gamma \sdash \delta$ then $\HA \vdash
%   \HH_\Lambda(\pi), \Gamma \sdash \delta$. For the original claim, this yields
%   $\HA \vdash \HH_\varepsilon, \Gamma \sdash \delta$. Observe that
%   $\HH_\varepsilon(\pi) = H_0(\pi)$ is always equivalent to $\top$, allowing us
%   to conclude $\HA \vdash \Gamma \sdash \delta$ as desired. We now prove the
%   generalisation by induction on the derivation $\pi$:
%   \begin{description}
%   \item[Bud:] In that case, $\HH_\Lambda(\pi) = I_\Lambda$, from which $\Gamma
%     \to \delta$ readily follows.
%   \item[$\RCase_x$:] Suppose the proof $\pi$ was of shape
%     \begin{comfproof}
%       \AXC{$\pi_0$}
%       \UIC{$\Lambda' \mid \Gamma[0 / x] \sdash \delta[0 / x]$}
%       \LSC{$\RDrop$}
%       \UIC{$\Lambda \mid \Gamma[0 / x] \sdash \delta[0 / x]$}
%       \AXC{$\pi_s$}
%       \UIC{$\Lambda^{+x} \mid \Gamma[sx / x] \sdash \delta[sx / x]$}
%       \LSC{$\RCase_x$}
%       \BIC{$\Lambda \mid \Gamma \sdash \delta$}
%     \end{comfproof}
%     then derive
%     \begin{comfproof}
%       \AXC{IH for $\pi_0$}
%       \UIC{$\HH_{\Lambda'}(\pi_0), \Gamma[0 / x] \sdash \delta[0 / x]$}
%       \LSC{$\wedge$L, $=$L}
%       \UIC{$x = 0 \wedge \HH_{\Lambda'}(\pi_0), \Gamma \sdash \delta$}
%       \AXC{IH for $\pi_s$}
%       \UIC{$\HH_\Lambda(\pi_s), \Gamma[sx / x] \sdash \delta[sx / x]$}
%       \RSC{\Cref{lem:ha-ren}}
%       \UIC{$\HH_\Lambda(\pi_s)[x'' / x], \Gamma[sx'' / x] \sdash \delta[sx'' / x]$}
%       \RSC{$\exists$L, $\wedge$L, $=$L}
%       \UIC{$\exists x'. x = sx' \wedge
%         \HH_\Lambda(\pi_s)[x' / x], \Gamma \sdash \delta$}
%       \LSC{$\vee$L}
%       \BIC{$(x = 0 \wedge \HH_{\Lambda'}(\pi_0)) \vee (\exists x''. x = sx'' \wedge
%         \HH_\Lambda(\pi_s)[x'' / x]), \Gamma \sdash \delta$}
%       \LSC{$\forall$L*}
%       \UIC{$\forall x' \leq x.~(x' = 0 \wedge \HH_{\Lambda'}(\pi_0)[x' / x]) \vee (\exists x''. x' = sx'' \wedge
%         \HH_\Lambda(\pi_s)[x'' / x]), \Gamma \sdash \delta$}
%     \end{comfproof}
%     where $\star$ is an application of the usual renaming lemma.
%   \item[Comp:] Suppose $\pi$ was of shape
%     \begin{comfproof}
%       \AXC{$\pi'$}
%       \UIC{$\Lambda; x^- \mapsto \Gamma \sdash \delta \mid \Gamma \sdash \delta$}
%       \UIC{$\Lambda \mid \Gamma \sdash \delta$}
%     \end{comfproof}
%     and fix $\Lambda' \coloneq \Lambda; x^- \mapsto \Gamma \sdash \delta$.
%     %
%     Derive the following in $\HA$:
%     \begin{scprooftree}{1}
%       \AXC{\Cref{lem:ih-lem} (iii)}
%       \UIC{$\HH_\Lambda(\pi), I_{\Lambda'} \sdash
%         \HH_{\Lambda'}(\pi')$}
%       \LSC{$\dagger$}

%       \AXC{IH}
%       \UIC{$\HH_{\Lambda'}(\pi'), \Gamma \sdash \delta$}
%       \RSC{\Cref{lem:ha-ren}}
%       \UIC{$\HH_{\Lambda'}(\pi')[\vec{u} / \vec{v}], \Gamma[\vec{u} / \vec{v}] \sdash \delta[\vec{u} / \vec{v}]$}
%       \RSC{\Cref{lem:ih-lem} (ii)}
%       \UIC{$\HH_{\Lambda'}(\pi'), \Gamma[\vec{u} / \vec{v}], \vec{u} \leq
%         \vec{v} \sdash \delta[\vec{u} / \vec{v}]$}
%       \RSC{$\forall$R*, $\to$L}
%       \UIC{$\HH_{\Lambda'}(\pi') \sdash \widehat{I}_{\Lambda'}$}
%       \LSC{$\RCut, \RWk$}
%       \BIC{$\HH_\Lambda(\pi), I_{\Lambda'} \sdash \widehat{I}_{\Lambda'}$}
%       \LSC{$\RInd$}
%       \UIC{$\HH_\Lambda(\pi) \sdash \forall x. \widehat{I}_{\Lambda'}$}

%       \AXC{$\Gamma, \bigwedge \Gamma \to \delta \sdash \delta$}
%       \RSC{$\forall$L*}
%       \UIC{$\Gamma, \forall x. \widehat{I}_{\Lambda'} \sdash \delta$}
%       \LSC{$\RCut$}
%       \insertBetweenHyps{\hskip -6em}
%       \BIC{$\HH_\Lambda(\pi), \Gamma \sdash \delta$}
%     \end{scprooftree}
%     where the right-most leaf is a trivial proof and at $\dagger$ we use that $\HH_{\Lambda}(\pi) = \HH_{\Lambda}(\pi')$.
%   \item[Pop:] Suppose $\pi$ was of shape
%     \begin{comfproof}
%       \AXC{$\pi'$}
%       \UIC{$\Lambda_0 \mid \Gamma \sdash \delta$}
%       \LSC{$\RDrop$}
%       \UIC{$\Lambda_0; \Lambda_1 \mid \Gamma \sdash \delta$}
%     \end{comfproof}
%     Then $\HH_\Lambda(\pi) = \HH_{\Lambda_0; \Lambda_1}(\pi) = \HH_{\Lambda_0}(\pi')$
%     and thus $\HA \vdash \HH_\Lambda(\pi), \Gamma \sdash \delta$ per IH of $\pi'$ directly.
%   \item[Bin:] As $\RCase$ was already treated above, we may assume that $\RBin$
%     directly corresponds to a logical rule $\RBin'$ of $\HA$. Thus $\pi$ is of the shape
%     below, notably $\Lambda$ appearing unchanged in both premises.
%     \begin{comfproof}
%       \AXC{$\pi_l$}
%       \noLine
%       \UIC{$\Lambda \mid \Gamma_l \sdash \delta_l$}
%       \AXC{$\pi_r$}
%       \noLine
%       \UIC{$\Lambda \mid \Gamma_r \sdash \delta_r$}
%       \LSC{\textsc{Bin}}
%       \BIC{$\Lambda \mid \Gamma \sdash \delta$}
%     \end{comfproof}
%     Then derive:
%     \begin{comfproof}
%       \AXC{IH of $\pi_l$}
%       \UIC{$\HH_\Lambda(\pi_l), \Gamma_l \sdash \delta_l$}
%       \LSC{$\wedge$L, $\RWk$}
%       \UIC{$\HH_\Lambda(\pi_l) \wedge \HH_\Lambda(\pi_r), \Gamma_l \sdash \delta_l$}
%       \RSC{$\wedge$L, $\RWk$}
%       \AXC{IH of $\pi_r$}
%       \UIC{$\HH_\Lambda(\pi_r), \Gamma_r \sdash \delta_r$}
%       \UIC{$\HH_\Lambda(\pi_l) \wedge \HH_\Lambda(\pi_r), \Gamma_r \sdash \delta_r$}
%       \LSC{$\RBin'$}
%       \BIC{$\HH_\Lambda(\pi_l) \wedge \HH_\Lambda(\pi_r), \Gamma \sdash \delta$}
%     \end{comfproof}
%   \item[Un:] As $\RComp$ was already treated above, we may assume that $\RUn$
%     directly corresponds to a logical or structural rule $\RUn'$ of $\HA$ and that
%     the derivation is of the following shape, with $\HH_\Lambda(\pi) = \HH_{\Lambda'}(\pi')$:
%     \begin{comfproof}
%       \AXC{$\pi'$}
%       \UIC{$\Lambda' \mid \Gamma' \sdash \delta'$}
%       \LSC{$\RUn$}
%       \UIC{$\Lambda \mid \Gamma \sdash \delta$}
%     \end{comfproof}
%     Then derive:
%     \begin{comfproof}
%       \AXC{IH}
%       \UIC{$\HH_\Lambda'(\pi'), \Gamma' \sdash \delta'$}
%       \LSC{$\RUn'$}
%       \UIC{$\HH_\Lambda'(\pi'), \Gamma \sdash \delta$}
%     \end{comfproof}
%   \end{description}
% \end{proof}

% \section{Unfolding Cyclic Arithmetic}
% \label{sec:combinatorics}

% This section covers the translation of $\CHA$-proofs into $\AHA$-proofs. The
% translation consists of two parts. First, $\CHA$-proofs are transformed into
% $\CHA$-proofs of the same end-sequent in a normal-form which simplifies the
% `interaction between cycles' within a proof. The $\CHA$-proofs in this normal-form
% are in fact essentially unfoldings of the original $\CHA$-proofs, meaning computation
% content is preserved. This first step is presented in
% \Cref{sec:induction-orders} and directly follows Sprenger and
% Dam~\cite{sprengerStructureInductiveReasoning2003a}, who show the analogous
% result for $\mFOL$, first-order logic extended with least and greatest fixed
% points. Their proof operates purely at the level of cyclic proofs and their
% traces, not interacting with the logical properties of $\mFOL$ at all, and thus
% readily transfers to $\CHA$ and $\CPA$. In the second step, $\CHA$-proofs in
% said normal-form are translated $\AHA$-proofs, which we present in \Cref{sec:io-to-sha}.

% We begin by fixing a concrete notion of \emph{tree} to work with in this section.
% For $s, t \in \omega^*$, we write $s < t$ if $s$
% is a strict prefix of $t$, and \( s \le t \) if either \( s < t \) or \( s = t \).
% A \emph{tree} $T \subseteq \omega^*$ is a non-empty set of sequences
% which is closed under prefixes: if $s \in T$ and $t < s$ then $t \in T$.
% We denote by
% {$[s, t]$} the path from $s$ to $t$, i.e. $[s, t] \coloneq \{u
% \in \omega^* \mid s \leq u \leq t\}$.
% We denote the extension of $s \in \omega^*$ with $n \in \omega$ by $s \cdot n$. Concatenation of sequences is denoted by juxtaposition such that \( s \le st \) for all \( s , t \in \omega^* \).
% Each $t \in T$ is called a \emph{node} and the nodes
% in $\Chld_T(t) \coloneq \{t \cdot i \in T ~|~ i \in \omega\}$ are called the
% \emph{children} of \( t \). We omit the parameter $T$, writing $\Chld(t)$,
% whenever this does not cause any confusion.
% A node $t$ is a \emph{leaf} of $T$ if $\Chld(t) = \emptyset$
% and an \emph{inner node} otherwise. We write \( \Leaf(T) \) and \( \Inner(T) \) for the set of leaves and inner nodes of \( T \), respectively.

% A \emph{cyclic tree} is a pair $C = (T, \beta)$ of a finite tree $T$ and
% a partial function $\beta \colon \Leaf(T) \to \Inner(T)$ mapping some leaves of $T$
% onto inner nodes of $T$.
% If $t \in \dom(\beta)$ one calls it a \emph{bud} and $\beta(t)$ its \emph{companion}.
% For a bud $t \in \dom(\beta)$, we call the path $[\beta(t), t]$ the \emph{local
%   cycle} of $t$ and denote it by $\gamma_C(t)$. We omit the parameter $C$,
% writing $\gamma(c)$, whenever this causes no confusion.

% Cyclic trees $(T, \beta)$ which satisfy $\beta(t) < t$ for every $t \in \dom(\beta)$ are
% said to be in \emph{cycle normal-form}. Every cyclic tree can be `unfolded'
% into cycle normal-form, although at an super-exponential size
% increase~\cite[Theorem 6.3.6]{brotherstonSequentCalculusProof2006}. We will thus
% assume, without loss of generality, that all cyclic trees in this section are in
% cycle normal-form. Note that cycle normal-form is not the normal-form of
% $\CHA$-proofs with which \Cref{sec:induction-orders} is concerned.

% Fix a cyclic tree $C = (T, \beta)$. A \emph{finite path} $p$ through $C$ is a
% non-empty sequence $p \in T^+$ such that for every $i < \abs{p} - 1$ we have
% $p_{i + 1} \in \Chld(p_i)$ if $p_i \in \Inner(T)$ and $p_i \in
% \dom(\beta)$ with $p_{i + 1} = \beta(p_i)$ otherwise. We say $p$
% \emph{starts at} $p_0$ and \emph{ends at} $p_{\abs{p} - 1}$. An \emph{infinite branch} through $C$
% is an infinite sequence $t \in T^\omega$ with $t_0 = \varepsilon$ and such
% that for each $i \in \omega$ either $t_{i + 1} \in \Chld(t_i)$ or $t_i \in
% \dom(\beta)$ and $t_{i + 1} = \beta(t_i)$. If $C$ is the underlying cyclic tree
% of a $\CHA$-proof $(C, \rho)$, this induces an infinite sequence
% $(\Gamma_i \sdash \delta_i)_{i \in \omega}$ of sequents with $\lambda(t_i) =
% \Gamma_i \sdash \delta_i $, which we use interchangeably with $(t_i)_{i \in
%   \omega}$ to denote an infinite branch.

% \subsection{Induction Orders}
% \label{sec:induction-orders}

% Sprenger and Dam~\cite{sprengerStructureInductiveReasoning2003a} work with a
% cyclic proof system using induction orders, a soundness condition of cyclic proofs different from the
% global trace condition given in \Cref{def:cha-proof}. We begin by transferring this soundness
% condition to the setting of Heyting arithmetic.

% Induction orders are characterised in terms of the strongly connected subgraphs of cyclic
% proofs, which can, in turn, be identified with a subset $\eta \subseteq
% \dom(\beta)$.
% \begin{definition}
%   Fix a cyclic tree $C = (T, \beta)$ and let $\eta \subseteq \dom(\beta)$.
%   \begin{enumerate}
%   \item The set of nodes $C[\eta] \coloneq \bigcup_{s \in \eta} \gamma(s)$ is
%     said to be part of $C$ \emph{covered} by the cycles corresponding to the buds in $\eta$.
%   \item
%     The set $\eta$ is \emph{strongly connected} if
%     there a finite path $p$ through $C$ which (1) starts and ends at the same
%     node and (2) visits precisely the nodes in $C[\eta]$, i.e.\ $p$ visits a node $t \in T$ if and only
%     if $t \in C[\eta]$.
%   \item A strongly connected $\eta$ is said to be \emph{$B$-maximal} for $B \subseteq \dom(\beta)$ if $\eta \subseteq B$ and there
%       is no $\eta \subsetneq \eta' \subseteq B$ which is strongly connected.
%   \item
%     Let $p$ be an infinite path through $C$ then $\Inf(p) \coloneq \{s \in T
% ~|~ p_i = s \text{ infinitely often}\}$.
%   \end{enumerate}
% \end{definition}

% The strongly connected components of a cyclic tree are closely linked to their infinite
% branches: For any infinite branch through a cyclic tree, the nodes visited
% infinitely often by it are precisely $C[\eta]$ for a strongly connected $\eta$.
% A proof of \Cref{lem:inf-cs}, for an equivalent definition of
% strongly connected component, result can be found in \cite[Lemma
% 5.3]{leighGTCResetGenerating2023}.

% \begin{lemma}\label{lem:inf-cs}
%   Let $t \in T^\omega$ be an infinite branch of a cyclic tree $C = (T, \beta)$.
%   Then there exists a strongly connected component $\eta$ of $C$ such that $\text{Inf}(t) =
%   C[\eta]$.
% \end{lemma}

% A key property of
% $B$-maximal strongly connected components is that they
% are disjoint iff they are distinct.

% \begin{lemma}
%   For a cyclic tree $C = (T, \beta)$ let $B \subseteq \dom(\beta)$ and let
%   $\eta, \eta' \subseteq B$ be $B$-maximal strongly connected components. Then either $\eta = \eta'$ or $\eta
%   \cap \eta' = \emptyset$.
% \end{lemma}
% \begin{proof}
%   Suppose there was $s \in \eta \cap \eta'$. Then, w.l.o.g. there exists paths
%   $p$ and $p'$ starting and ending at $s$ such that $p$
%   and $p'$ visit precisely the nodes of $C[\eta]$ and $C[\eta']$,
%   respectively. `Pasting together' $p$ and $p'$ yields a path from $s$ to
%   $s$ visiting precisely the elements of $C[\eta] \cup C[\eta'] = C[\eta \cup
%   \eta']$, meaning $\eta \cup \eta'$ is strongly connected. By the
%   $B$-maximality of $\eta$ and $\eta'$ it then follows that $\eta = \eta \cup
%   \eta' = \eta'$.
% \end{proof}

% Having established some properties of strongly connected components, we now define induction orders.

% \begin{definition}[Induction order]\label{def:io}
%   Let $\pi = (C, \rho)$ be a $\CHA$-pre-proof.
%  An \emph{induction order} is a preorder
%   $\mathrel{\preceq}\; \subseteq
%   \dom(\beta) \times \dom (\beta) $ together with a mapping $x_{-} : \dom(\beta)
%   \to \var$ such that $x_s \in \FV(\lambda(s))$ for all $s \in \dom(\beta)$
%   and
%   \begin{itemize}
%   \item if $s \preceq t$ then $\gamma(s)$ \emph{preserves} $x_t$, i.e.\ $x_t \in
%     \FV(\lambda(u))$ for every $u \in \gamma(s)$,
%   \item the cycle $\gamma(s)$ \emph{progresses} $x_s$, i.e.\ a $\RCase_{x_s}$
%     instance is applied along the path $\gamma(s)$,
%   \item every strongly connected $\eta \subseteq \dom(\beta)$ has a
%     $\preceq$-maximum, i.e.\ there is $s \in \eta$ such that $t \preceq s$ for
%     all $t \in \eta$
%   \end{itemize}
% \end{definition}

% In our cyclic system for Heyting arithmetic, every proof has an induction order.
% This fact is closely linked to the notion of trace and the trace condition of
% $\CHA$. Other cyclic proof systems for first-order arithmetic,
% such as those considered by Simpson~\cite{simpsonCyclicArithmeticEquivalent2017}
% or Berardi and Tatsuta~\cite{berardiEquivalenceIntuitionisticInductive2017}, do
% not exhibit this property.
% The proof of this property is based on \cite[Theorem
% 5.14]{sprengerGlobalInductionMechanisms2003} which proves the same property for
% the $\mFOL$-system Sprenger and Dam consider.

% \begin{restatable}{lemma}{existsio}
%   \label{lem:exists-io}
%   A $\CHA$-pre-proof $\pi$ is a proof iff there exists an induction order for it.
% \end{restatable}
% \begin{proof}
%   For the backwards direction, consider a branch $t \in T^\omega$ through $\pi$.
%   Then $\Inf(t) = C[\eta]$ for some strongly connected $\eta$ by \Cref{lem:inf-cs}.
%   Notably, there must be some $n \in \omega$ such that $t_i \in C[\eta]$ for all
%   $i > n$. As $\preceq$ is an induction order, $\eta$ must have a {$\preceq$-maximum $s
%   \in \eta$}. Then $x_s \in \bigcap_{i > n}
%   \var(\Gamma_i, \delta_i)$ because {$\gamma(u)$} for each $u \in \eta$ preserves $x_s$, meaning there is an $x_s$-trace along $t$. This trace
%   is progressing as there is a $\RCase_{x_s}$-instance along $\gamma(s)$ which
%   $t$ passes infinitely often.

%   Conversely, suppose $\pi$ was a proof. We obtain an induction order on
%   $\pi$ through an iterative process. This process produces a sequence $S_0,
%   S_1, \ldots$ of sets of disjoint strongly connected components of $\pi$ such that at each
%   step, some bud $s \in \dom(\beta)$ is removed from $\bigcup S_i$. To begin, define $S_0 \coloneq
%   \{\eta \subseteq \dom(\beta) ~|~ \eta \text{ is a } \dom(\beta)\text{-maximal
%     strongly connected component}\}$. To obtain $S_{i + 1}$, select some $\eta_i \in S_i$ and
%   consider a branch $t \in T^\omega$ through $\pi$ such that $\Inf(t) = \eta_i$.
%   As $\pi$ is a proof, there must be an $n \in \omega$ such that for some $x \in
%   \bigcap_{n < j} \var(\Gamma_i, \delta_i)$ there exists a progressing $x$-trace
%   along $p$. Hence, $x$ must be preserved along the cycle {$\gamma(t)$} for every $t
%   \in \eta$ and there must be an $s \in \eta$ for which the cycle $\gamma(s)$
%   progresses $x$. Fix $x_s \coloneq x$ and set
%   \[S_{i + 1} \coloneq (S_{i} \setminus \eta_i) \cup \{\eta' \subseteq (\eta_i
%     \setminus \{s\}) ~|~ \eta' \text{ is a } (\eta_i
%     \setminus \{s\})\text{-maximal strongly connected component} \}.\]
%   That is, $S_{i + 1}$ is obtained by replacing $\eta_i$ in $S_i$ by the maximal
%   strongly connected {components partitioning} $\eta_i$ after
%   removing $s$. {Because each $S_i$ consists of pairwise
%     disjoint strongly connected components, each $S_i$ partitions $\bigcup
%   S_i$.} This process terminates at $S_{\abs{\dom(\beta)}} = \emptyset$ as every
%   step removes a bud from $\bigcup S_i$. It remains to define the preorder
%   $\preceq$ on $\dom(\beta)$.
%   For $s, t \in \dom(\beta)$, let $i$ be the least index 
%   {
%   such that $t \not\in \bigcup S_{i + 1}$, i.e.\ $t$ was removed at step $i$ of
%   the procedure. We set $s \preceq t$ iff $s \in \eta_i$.
%   Now consider a strongly connected $\eta \subseteq \dom(\beta)$.} We claim that
%   there is a $\preceq$-maximum of $\eta$. As the
%   elements of $S_0$ are $\dom(\beta)$-maximal, there must be $\eta' \in S_0$
%   such that $\eta \subseteq \eta'$. Now at each step of the process, {if $\eta'
%   \in S_i$} then either
%   $\eta' = \eta_i$ or $\eta' \in S_{i + 1}$. Because $S_{\abs{\dom(\beta)}} =
%   \emptyset$ there thus must be $\eta_i = \eta'$. Consider the bud $s$ which is
%   removed from $\eta_i$ to construct $S_{i + 1}$; if $s \not\in \eta$ then there
%   is a $(\eta_i \setminus \{s\})$-maximal $\eta'' \in S_{i + 1}$ such that $\eta
%   \subseteq \eta''$. Again, because $S_{\abs{\dom(\beta)}} = \emptyset$ there eventually
%   must be a step $i$ such that $\eta \subseteq \eta_i$ and the bud $s$ removed
%   from $\eta_i$ is in $\eta$. Then $s$ is a $\preceq$-maximum in $\eta_i$, and thus
%   in $\eta$, per definition of $\preceq$.
%   Reflexivity of $\preceq$ is trivial. Transitivity of $\preceq$ follows from an
%   argument, similar to that for $\preceq$-maxima, along the
%   iterative process, observing that for $s \preceq
%   t \preceq u$ with corresponding indexes $t \not\in \bigcup S_{i + 1}, u \not\in
%   \bigcup S_{j + 1}$ and $s \in \eta_i, t \in \eta_j$, $\eta_i \subsetneq \eta_j$ because each step `separates'
%   some $\eta_k$ into strongly connected sub-components $\eta' \subsetneq \eta_k$.
% \end{proof}
% % \begin{proofsketch}
% %   Let $\pi$ have an induction order. Any infinite branch of $p$ must
% %   eventually precisely `follow' $C[\eta]$ for some strongly connected $\eta
% %   \subseteq \dom(\beta)$. Let $s \in \eta$ be a $\preceq$-maximum of $\eta$,
% %   then $x_s$ is preserved in $C[\eta]$ and progressed along $\gamma(s)$, which
% %   must be traversed infinitely often. Hence $p$ has a progressing trace on $x_s$.

% %   Conversely, let $\pi$ be a $\CHA$-proof. To iteratively compute the induction
% %   order, start by considering the `maximally
% %   disjoint' strongly connected $\eta_1, \ldots, \eta_n$ of $\pi$. A
% %   branch $p$ precisely `following' some $C[\eta_i]$ must have some progressing trace $x$,
% %   meaning $x$ is preserved in $C[\eta_i]$ and progressed along some $\gamma(s)$
% %   for $s \in \eta_i$. Set $x_s \coloneq s$ and replace $\eta_i$ by its
% %   `maximally disjoint' strongly connected subgraphs excluding $s$. Iterate this
% %   procedure until all simple cycles $\gamma(s)$ are assigned an $x_s$. Set $s
% %   \preceq t$ if, for the strongly connected $\eta$ `taken apart' to assign
% %   $x_t$, we have $s \in \eta$. This defines an induction order for $\pi$.
% % \end{proofsketch}

% The following combinatorial lemma is crucial to the proof normalisation of $\CHA$-proofs.
% A pre-proof $\pi = ((T, \beta), \lambda)$ is called \emph{injective} if the
% function $\beta$ is injective.

% \begin{lemma}\label{lem:exists-injective}
%   If $\CHA \vdash \Gamma \sdash \delta$ then there exists an injective $\CHA$-proof of
%   $\Gamma \sdash \delta$.
% \end{lemma}
% \begin{proof}
%   Suppose $\pi = ((T, \beta), \rho)$ was a proof of $\Gamma \sdash \delta$.
%   To obtain an injective proof of $\Gamma \sdash \delta$, apply the following
%   procedure iteratively: For $u \neq v \in \dom(\beta)$ with $\beta(u) =
%   \beta(v) = t$ and $\lambda(t) = \Gamma' \sdash \delta$ as depicted on the
%   left-hand side of \Cref{f-injective}, add an application of the $\RWk$-rule which does not
%   remove any formulas from $\Gamma'$ directly above $t$, `generating' a new node
%   $t'$ above $t$ with $\lambda(t') = \lambda(t) = \Gamma' \sdash \delta$ and
%   `redirect' $\beta(u) \coloneq t'$, as depicted on the right-hand side. After finitely
%   many iteration steps, this process yields an injective pre-proof. As these transformations
%   do not affect the traces through the
%   pre-proof, it remains a proof.
% %
% %  \begin{center}
% %  \end{center}
% \end{proof}
% \begin{figure}% Figures should be either top or bottom of page -- we can let LateX decide
% \centering
% \begin{tikzpicture}
%   % Original proof tree
%   \node[inner sep=1] (u) at (-0.8, 1.2) {$u$};
%   \node[inner sep=1] (v) at (0.8, 1.2) {$v$};
%   \node (sp) at (0, 0.4) {$\ldots$};
%   \draw (sp) -- (v);
%   \draw (sp) -- (u);
%   \draw (-1, 0.2) -- (1, 0.2);
%   \node at (-1.2, 0.2) {$R$};
%   \node (t) at (0, 0) {$t : \Gamma' \sdash \delta'$};

%   \draw[->, dashed] (u) -- (-1.4, 1.2) -- (-1.4, 0) -- (t);
%   \draw[->, dashed] (v) -- (1.2, 1.2) -- (1.2, 0) -- (t);

%   \draw[-{stealth}] (1.6, 0.8) -- (3.7, 0.8);

%   % Injective proof tree
%   \node[inner sep=1] (u) at (5, 1.65) {$u$};
%   \node[inner sep=1] (v) at (6.6, 1.65) {$v$};
%   \node (sp) at (5.8, 0.85) {$\ldots$};
%   \draw (sp) -- (v);
%   \draw (sp) -- (u);
%   \draw (4.8, 0.65) -- (6.8, 0.65);
%   \node at (4.6, 0.65) {$R$};
%   \node (t') at (5.8, 0.45) {$t' : \Gamma' \sdash \delta'$};
%   \draw (4.8, 0.2) -- (6.8, 0.2);
%   \node at (4.4, 0.2) {$\RWk$};
%   \node (t) at (5.8, 0) {$t : \Gamma' \sdash \delta'$};

%   \draw[->, dashed] (u) -- (4.4, 1.65) -- (4.4, 0.45) -- (t');
%   \draw[->, dashed] (v) -- (7, 1.65) -- (7, 0) -- (t);
% \end{tikzpicture}
% \caption{Transformation into injective cyclic proofs}
% \label{f-injective}
% \end{figure}

% The translation procedure of Sprenger and
% Dam~\cite{sprengerStructureInductiveReasoning2003a} operates on cyclic proofs
% whose induction order coincides with the structure of the proof's cycles.
% The \emph{structural dependency
%   order $\sqsubseteq$} is defined on $\dom(\beta)$ such that $s \sqsubseteq t$
% holds iff $\beta(s) \in \gamma(t)$, i.e. if the companion of $s$ occurs along
% the local cycle of $t$. The transitive closure of \( \sqsubseteq \) is denoted \( \sqsubseteq^* \).
% Sprenger and Dam show that every $\mFOL$-proof can be unfolded into a proof
% with induction order $\sqsubseteq^*$. This result readily transfers to the
% setting of $\CHA$.

% \begin{theorem}
%   \label{thm:ind-order-SD}
%   For every injective \( \CHA \)-proof \( \pi \) with induction order \( \sqsubseteq \) there is an injective \( \CHA \)-proof \( \pi' \) with induction order \( \sqsubseteq^* \) employing only sequents in \( \pi \).
% \end{theorem}
% %
% \begin{proof}
%   Theorem 5 and Lemma 5 of \cite{sprengerStructureInductiveReasoning2003a}.
%   While the proof system considered 
%   in~\cite{sprengerStructureInductiveReasoning2003a} is for a logic different
%   from $\HA$, the transformation proof reasons on the level of cyclic trees and thus readily
%   transfer to the setting of $\CHA$.
% \end{proof}

% \begin{restatable}{lemma}{goodio}
%   \label{lem:good-io}
%   If $\CHA \vdash \Gamma \sdash \delta$ then there exists an injective
%   $\CHA$-proof of $\Gamma \sdash \delta$ with induction order $\sqsubseteq^*$.
% \end{restatable}

% \begin{proof}
%   By \Cref{lem:exists-injective} and \Cref{thm:ind-order-SD}.
% \end{proof}

% \subsection{Translating Proofs with $\sqsubseteq^*$ induction orders to $\AHA$}
% \label{sec:io-to-sha}

% In the beginning of \Cref{sec:combinatorics} we hinted at a normal form of
% $\CHA$-proofs which is embodied by the proof system $\AHA$. We can now make this
% precise: A $\CHA$-proof $\pi$ is in said normal form if it is injective and its
% induction order is $\sqsubseteq^*$. Thus, \Cref{lem:good-io} proves that every
% $\CHA$-proof can be transformed into one exhibiting this normal form. The
% remainder of this section is concerned with translating $\CHA$-proofs in said
% normal form to $\AHA$-proofs.

% The translation from $\CHA$-proofs to $\AHA$-proofs is essentially verbatim,
% i.e. each $\CHA$-rule is translated to its corresponding $\AHA$-rule. The only
% difficulty of the translation is assigning each sequent a suitable stack
% $\Lambda$ of companion labels and inserting the `$\Lambda$-bookkeeping rules'
% $\RDrop$ and $\RComp$ at the right locations.
% Towards the assignment of the $\Lambda$, consider a $\CHA$-proof $\pi = ((T,
% \beta), \rho)$ with an induction order
% $\preceq$ and
% recall that \( \gamma(t) \) denotes the local cycle from $[\beta(t), t]$ for $t
% \in \dom(\beta)$. For any given node $s \in T$ of $\pi$, the induction order
% guarantees that $x_u \in \FV(\lambda(s))$, i.e. certain $x_u$ occurs free in the
% sequent at $s$, for certain $u \in \dom(\beta)$. More concretely, we know this
% to be the case if $s \in \gamma(t)$ and $u \preceq t$, i.e. $s$ occurs on a
% local cycle $\gamma(t)$ which must \emph{preserve} the variable $x_u$ associated
% with a local cycle $\gamma(u)$ such that $u \preceq t$.
% We consider the set $S(s) \subseteq \dom(\beta)$ which collects these $u$:
% \[
% 	S(s) = \{u \in \dom(\beta) ~|~ \exists t \in \dom(\beta).~s \in \gamma(t) \wedge t \sqsubseteq^* u\}
% \]
% and begin by proving that, when imposing a suitable ordering on them, the $S(s)$ can indeed be
% treated as a stack. In the following, we denote by \( <_\beta \) the order on \(
% \dom(\beta) \) induced by \( \beta \), given by $s <_\beta t$ iff $\beta(s) <
% \beta(t)$.

% \begin{restatable}{lemma}{alphatrans}
%   \label{lem:alpha}
%   Let $\pi = ((T, \beta), \lambda)$ be an injective $\CHA$-proof with induction order
%   $\sqsubseteq^*$. 
%   The set \( S(s) \) is linearly ordered by \( <_\beta \) for every \( s \in T \).
%   Moreover, the function $\sigma : T \to \dom(\beta)^*$ assigning to \( s \in T
%   \) the enumeration of \( S(s) \) according to \(<_\beta\) satisfies:
%   \begin{enumerate}
%   \item If $\varepsilon \not\in \im(\beta)$ then $\sigma(\varepsilon) =
%     \varepsilon$; otherwise $\sigma(\varepsilon) = \beta^{-1}(\varepsilon)$.
%   \item If $t \in \Chld(s)$ and \( t \not\in \im(\beta) \), then \( \sigma(t)
%     \leq \sigma(s) \).
%   \item If $t \in \Chld(s)$ and \( t \in \im(\beta) \), then there is $u \leq \sigma(s)$
%     such that $\sigma(t) = u \cdot \beta^{-1}(t)$.
%   \end{enumerate}
% \end{restatable}

% \begin{proof}
%   First, notice that per construction, if $t \in S(s)$ then $\beta(t) \leq s$.
%   Hence, $S(s)$ is indeed linearly ordered by $<_\beta$ and the
%   function $\sigma$ given above well-defined. Furthermore, $\abs{\beta^{-1}(t)}
%   \leq 1$ by the injectivity of $\beta$.

%   Define $\Down(t) \coloneq \{s \in \dom(\beta) ~|~ s <_\beta t\}$ for $t \in \dom(\beta)$.

%   \begin{enumerate}[1.]
%   \item We know that $S(\varepsilon) = \{\beta^{-1}(\varepsilon) \mid \text{if }
%     \varepsilon \in \im(\beta) \}$ because $\varepsilon \in \gamma(t)$ entails
%     $t = \beta^{-1}(\varepsilon)$.
%   \item[2. and 3.] 
%     We first prove
%     that if $t \in \Chld(s)$ then $S(t) \subseteq S(s) \cup \beta^{-1}(t)$: Fix
%     some $v \in S(t)$, i.e. such that there is $u$ with $t \in \gamma(u)$ and $u
%     \sqsubseteq^* v$. There are two cases:
%     \begin{description}
%     \item[$\beta(u) \neq t$:] Then $s \in \gamma(u)$, as $t \in \Chld(s)$, and thus $v \in S(s)$.
%     \item[$\beta(u) = t$:] Then we consider two cases arising from $u
%       \sqsubseteq^* v$:
%       \begin{description}
%       \item[$u = v$:] Then $u \in \beta^{-1}(t)$.
%       \item[$u \sqsubset v_1 \sqsubseteq^* v$:] Then $\beta(v_1) < t$ and thus $s \in \gamma(v_1)$, meaning $v \in S(s)$.
%       \end{description}
%     \end{description}

%     What remains to be shown towards 2.\ and 3.\ is that buds are only removed
%     in an $\leq_\beta$-upwards closed manner: if $t \in \Chld(s)$ and $u
%     \in S(s) \setminus S(t)$, i.e.\ $u$ was removed from $S(-)$ in stepping from
%     $s$ to $t$, then there is no $u' \in S(t)$ such that $u \leq_\beta u' \neq \beta^{-1}(t)$.

%     We instead prove an equivalent statement: For $t
%     \in \Chld(s)$, if $u \in S(t)$ with $u \neq \beta^{-1}(t)$ then $S(s) \cap
%     \Down(u) \subseteq S(t)$. If $u \in S(t)$, there must be $v$ such
%     that $t \in \gamma(v)$ and $v \sqsubseteq^* u$. As $\beta^{-1}(u) \neq t$, $s
%     \in \gamma(u)$ as well. Now fix $u' \in S(s)
%     \cap \Down(u)$, meaning there is $v'$ with $s \in \gamma(v')$ and $v'
%     \sqsubseteq^* u'$, as $u' \in S(s)$, and $u' <_\beta u$, as $u' \in \Down(u)$.
%     We distinguish two cases:
%     \begin{description}
%     \item[$v = v'$:] Then $u' \in S(s)$ as $v = v' \sqsubseteq^* u'$.
%     \item[$v \neq v'$:] Because of $\beta(v), \beta(v') < s < v, v'$, we may
%       distinguish two further cases:
%       \begin{description}
%       \item[$v \sqsubset v'$:]
%         Then $v \sqsubset v' \sqsubseteq^* u'$ and thus $u' \in S(t)$.
%       \item[$v' \sqsubset v$:] 
%         This case is resolved with a further case distinction:
%         For any $x$, if $v <_\beta x$ and $x \sqsubseteq^* u'$ then either $v \sqsubseteq^* u'$ or
%         $v <_\beta u'$. From $v' \sqsubset v$ it then follows that $v <_\beta v'$ and
%         thus:
%         \begin{description}
%         \item[$v \sqsubseteq^* u'$:] Then $u' \in S(t)$ as $t \in \gamma(v)$.
%         \item[$v <_\beta u'$:] But $u' <_\beta u$ together with $v \sqsubseteq^*
%           u$ yield $u' <_\beta u \leq_\beta v$, ruling out this option.
%         \end{description}

%         It remains to prove the aforementioned disjunction, which we do per induction on the $\sqsubset$-chain
%         witnessing $x \sqsubseteq^* u'$: If $x = u'$ then $v <_\beta x = u'$. Thus
%         suppose $x \sqsubset x' \sqsubseteq^* u'$. Because $v, x' \leq_\beta x$,
%         we may distinguish two cases:
%         \begin{description}
%         \item[$x' \leq_\beta v$:] Then $x' \leq_\beta v <_\beta x \leq x'$ and thus $v \sqsubset x' \sqsubseteq^* u'$.
%         \item[$v <_\beta x'$:] Then continue per inductive hypothesis with $x \coloneq x'$.
%         \end{description}
%       \end{description}
%     \end{description}
%   \end{enumerate}
% \end{proof}

% \begin{remark}
%   Intuitively, \Cref{lem:alpha} can be read as stating that the preorder of $\pi$ being
%   $\sqsubseteq$ means the $S(s)$ `behave as stacks' when ordered by $<_\beta$
%   and the injectivity of $\pi$ means that these `stacks' are extended by at most
%   one bud between two nodes, mirroring the $\RComp$-rule.
% \end{remark}

% Having accounted for the companion stacks $\Lambda$, the rest of the translation
% is straight-forward.

% \chatoaha*
% % \begin{lemma}\label{lem:cha-to-aha}
% %   If $\CHA \vdash \Gamma \sdash \delta$ then $\AHA \vdash \Gamma \sdash \delta$.
% % \end{lemma}
% \begin{proof}
%   By \Cref{lem:good-io}, there exists an injective $\CHA$-proof $\pi = ((T, \beta), \rho)$ of
%   $\Gamma \sdash \delta$ with $\sqsubseteq^*$ as its induction order. Let
%   $\sigma : T \to \dom(\beta)^*$ be the labelling function of $\pi$ as defined
%   in \Cref{lem:alpha}. We prove, per induction on the tree structure of
%   $T$, starting at the leaves, that for each $s \in T$ there exists a $\AHA$-derivation of
%   $\Lambda^s_{\sigma(s)} \mid \lambda(s)$. The companion stacks
%   $\Lambda^s_{\sigma(s)}$ are recursively defined as follows:
%   \[
%     \Lambda^s_\varepsilon \coloneq \varepsilon
%     \qquad
%     \Lambda^s_{t \cdot u} \coloneq
%     \begin{cases}
%       (x^+_t \mapsto \lambda(s)); \Lambda^s_u & \text{if } [\beta(t), s] \text{
%         passes through the right premise of } \RCase_{x_t} \\
%       (x^-_t \mapsto \lambda(s)); \Lambda^s_u & \text{otherwise}
%     \end{cases}
%   \]
%   We argue via a case distinction on the form of $s$:
%   \begin{description}
%   \item[$s \in \dom(\beta)$:] Then $\sigma(s) = u \cdot s \cdot u'$ for some $u, u' \in
%     \dom(\beta)^*$ because $s \in S(s)$ if $s \in \dom(\beta)$. As $\pi$ has an induction order,
%     $\RCase_{x_s}$ is applied along $\gamma(s)$. Hence, $\Lambda^s_{\sigma(s)} =
%     \Lambda^s_u; (x_s^+ \mapsto \lambda(s)) ; \Lambda^s_{u'}$. The derivation
%     below is as desired.
%     \begin{comfproof}
%       \AXC{}
%       \LSC{$\RBud$}
%       \UIC{$\Lambda^s_u; x^+_s \mapsto \lambda(s) \mid \lambda(s)$}
%       \LSC{$\RDrop$}
%       \UIC{$\Lambda^s_u; (x^+_s \mapsto \lambda(s)) ; \Lambda^s_{u'} \mid \lambda(s)$}
%     \end{comfproof}
%   \item[$\rho(s)$ an axiom:] Then $\varepsilon \mid \lambda(s)$ is an
%     axiom of $\AHA$, meaning the derivation below is as desired, where $\dagger$
%     is the axiom of $\AHA$ corresponding to $\rho(s)$.
%     \begin{comfproof}
%       \AXC{}
%       \LSC{$\dagger$}
%       \UIC{$\varepsilon; \mid \lambda(s)$}
%       \LSC{$\RDrop$}
%       \UIC{$\Lambda^s_{\sigma(s)}; \mid \lambda(s)$}
%     \end{comfproof}
%   \item[$s$ an inner node:] 
%     Then $s$ was derived from $\Chld(s) = \{t_1, \ldots, t_n\}$ by an
%     application of the $\CHA$-rule $R$ and there are $\AHA$-derivations of
%     $\Lambda^{t_i}_{\sigma(t_i)} \mid \lambda(t_i)$.

%     The rule $R$ has a
%     corresponding $\AHA$-rule of the same name. Because of the well-formedness
%     restriction on $\AHA$-sequents, the corresponding $\AHA$-rule may only be
%     applied to derive $\Lambda^s_{\sigma(s)} \mid \lambda(s)$ if for every companion label
%     $(x^\bullet \mapsto \ldots) \in \Lambda^s_{\sigma(s)}$ we have $x \in
%     \FV(\lambda(s))$. In other words, if $x_{s'} \in \FV(\lambda(s))$ for every
%     $s' \in S(s)$. As argued when we defined $S(s)$, the $s' \in S(s)$ are
%     precisely those for which the induction order guarantees that $x_{s'} \in
%     \FV(\lambda(s))$ and thus the $\AHA$-rule $R$ may be applied.

%     To derive $\Lambda_{\sigma(s)}^s \mid
%     \lambda(s)$ from $ \bigl\{ \Lambda_{\sigma(t_i)}^{t_i} \mid \lambda(t_i) \bigm| 1 \le i \le n \bigr\}$ we must
%     distinguish two cases for each $t_i$: either $t_i \in \im(\beta)$ or not.
%     For the purpose of illustration, let $t_1 \not\in \im(\beta)$ and $t_2 \in
%     \im(\beta)$. By \Cref{lem:alpha} $\sigma(t_1) \leq \sigma(s)$ and there
%     exists $u \leq \sigma(s)$ such that $\sigma(t_2) = u \cdot s'$ for
%     $\beta(s') = t_2$.
%     The following derivation, in which $t_3, \ldots, t_n$ are treated
%     analogously to $t_1$ or $t_2$ based on the aforementioned case-distinction,
%     is as desired:
%     \begin{comfproof}
%       \AXC{$\Lambda^{t_1}_{\sigma(t_1)} \mid \lambda(t_1)$}
%       \LSC{$\RDrop$}
%       \UIC{$\Lambda^{t_1}_{\sigma(s)} \mid \lambda(t_1)$}
%       \AXC{$\Lambda^{t_2}_{u}; x^-_{s'} \mapsto \lambda(t_2) \mid \lambda(t_2)$}
%       \LSC{$\RComp$}
%       \UIC{$\Lambda^{t_2}_{u} \mid \lambda(t_2)$}
%       \LSC{$\RDrop$}
%       \UIC{$\Lambda^{t_2}_{\sigma(s)} \mid \lambda(t_2)$}
%       \AXC{$\dotsm$}
%       \AXC{$\Lambda^{t_i}_{\sigma(t_i)} \mid \lambda(t_i)$}
%       \DOC{}
%       \noLine
%       \UIC{$\Lambda^{t_i}_{\sigma(s)} \mid \lambda(t_i)$}
%       \AXC{$\dotsm$}
%       \LSC{$R$}
%       \QuinaryInfC{$\Lambda^s_{\sigma(s)} \mid \lambda(s)$}
%     \end{comfproof}
%     In the case of $t_2$ (and other companions), observe that $\Lambda^{t_2}_{\sigma(t_2)} =
%     \Lambda^{t_2}_{u} ; x^-_{s'} \mapsto \lambda(t_2)$. Note also that the
%     $\RDrop$-applications account for the fact that $\sigma(t_1) \leq \sigma(s)$
%     and $\sigma(t_2) = u \cdot s'$ for $u \leq \sigma(s)$. For the derivation above
%     to be valid, it remains to argue that the premises of $R$ indeed have
%     companion stacks $\Lambda^{t_i}_{\sigma(s)}$ rather than
%     $\Lambda^{s}_{\sigma(s)}$ (note the difference in superscripts). If $R \neq
%     \RCase_{x}$ then $\Lambda^s_{u} = \Lambda^{t_i}_{u}$ because if $[\beta(s'),
%     t_i]$ passes through the right-hand premise of $\RCase_{x_{s'}}$, this
%     must have taken place below $s$ and thus the marker on
%     $x^\bullet_{s'}$ in $\Lambda^{s}_{u}$ and $\Lambda^{t_i}_{u}$ for each
%     $s'$ occurring in $u$ coincide. If $R = \RCase_{x}$ then this is more
%     subtle. The companion labelling of the right-hand premise of $\RCase_{y}$ is
%     $(\Lambda^s_{\sigma(s)})^{+x}$. This is precisely
%     $\Lambda^{t_2}_{\sigma(s)}$ which, by an argument analogous to that of the
%     previous case, differs from $\Lambda^s_{\sigma(s)}$ only by the markings on
%     $y$, which are all $y^+$ in $\Lambda^{t_2}_{\sigma(s)}$ whereas
%     $\Lambda^{s}_{\sigma(s)}$ may contain instances of $y^-$. On the
%     other hand, $\Lambda_{\sigma(s)}^{s} = \Lambda_{\sigma(s)}^{t_1}$ for the
%     left-hand of $\RCase_{y}$.
%   \end{description}

%   From this inductive argument, we can conclude that either $\varepsilon \mid
%   \Gamma \sdash \delta$ (if $\varepsilon \not\in \im(\beta)$) or
%   $x_{s} \mapsto \lambda(\varepsilon) \mid \Gamma \sdash \delta$ (if $\beta(s)
%   = \varepsilon$). From the latter, $\varepsilon \mid \Gamma \sdash \delta$ can
%   be derived by an application of $\RComp$. Thus $\AHA \vdash \Gamma \sdash \delta$.
% \end{proof}

% % The translation above 
% % \TODO{Adjust!}A companion label $\widehat{x} \mapsto \Gamma' \sdash
% % \delta'$ in the stack $\Lambda$ at a node indicates that there exists a node
% % along the branch in the derivation between the root and the current position 
% % which is labelled with the sequent $\Gamma' \sdash \delta'$. For a leaf of a
% % $\AHA$-proof to take said node as its companion, the local cycle between them
% % must preserve and progress the variable $x$. The former is ensured by the fact
% % that for each $\widehat{x} \mapsto \Gamma' \sdash \delta' \in \Lambda$ for
% % $\Lambda \mid \Gamma \sdash \Delta$, $x$ must occur freely in $\Gamma$ or
% % $\delta$. The progression is ensured by the markings on $\widehat{x}$: $\ol{x}$
% % indicates that the variable $x$ has progressed at least once between the
% % companion and the current node. These variable conditions together with the fact
% % that $\Lambda$ is managed as a stack ensure that a $\AHA$-proof has
% % $\sqsubseteq^*$ as its induction order.

% \section{Conclusion}
% \label{sec:conclusion}

% Combining the results of \Cref{sec:translation,sec:combinatorics} yields
% the desired translation.

% \begin{theorem}\label{lem:full}
%   If $\CHA \vdash \Gamma \sdash \delta$ then $\HA \vdash \Gamma \sdash \delta$.
% \end{theorem}
% \begin{proof}
%   If $\CHA \vdash \Gamma \sdash \delta$ then $\AHA \vdash \Gamma \sdash \delta$
%   by \Cref{lem:cha-to-aha}. Then $\HA \vdash \Gamma \sdash \delta$ by \Cref{lem:key}.
% \end{proof}

% % Order:
% % - Related work
% % - PA
% % - Logical complexity
% % - Size complexity?
% % - Applicability

% In this section, we analyse various aspects of \Cref{lem:full} and compare the
% method we present to others in the literature. We begin by discussing other
% translation methods for cyclic proof systems of first-order arithmetic in
% \Cref{sec:related-work}.
% In \Cref{sec:pa}, we extend the
% translation result to Peano arithmetic. \Cref{sec:log-comp} demonstrates that
% our method yields Das' \cite{dasLogicalComplexityCyclic2020} logical complexity
% bound $\CP_n \subseteq \IP_{n + 1}$ for $\PA$ and an analogous result for $\HA$.
% In \Cref{sec:proof-size} we analyse the proof size increase incurred by our
% method.
% We close by discussing the applicability of our method
% to other cyclic proof systems in \Cref{sec:applicabilty}.

% \subsection{Related work}
% \label{sec:related-work}

% The literature contains translations of cyclic proofs into inductive proofs
% for various logics~\cite{das_cut-free_2017,curzi_computational_2023,kuperberg_cyclic_2021,das_cyclic_2023,das_circular_2021,shamkanov_circular_2014,curzi_cyclic_2021,nollet_local_2018}. 
%   Restricting attention to Heyting or Peano arithmetic, there
% are two distinct approaches to such
% translations already present in the literature. The first, introduced by
% Simpson~\cite{simpsonCyclicArithmeticEquivalent2017} and later refined by
% Das~\cite{dasLogicalComplexityCyclic2020}, proceeds by formalising the soundness
% of cyclic Peano arithmetic in $\ACA$, obtaining $\PA$ proofs via the
% conservativity of $\ACA$ over said system. The second approach, put forward by
% Berardi and
% Tatsuta~\cite{berardiEquivalenceInductiveDefinitions2017,berardiEquivalenceIntuitionisticInductive2017},
% uses Ramsey-style order-theoretic principles formalisable in $\HA$ to obtain an
% induction principle mirroring the cyclic structure of a $\CHA$ or $\CPA$ proof,
% using which translating the original proof to $\HA$ or $\PA$, respectively,
% becomes straight-forward.
% % Summary Simpson & Das
% % Summary Berardi & Tatsuta

% \subsection{Peano arithmetic}
% \label{sec:pa}

% A proof system $\PA$ of Peano arithmetic can be obtained by modifying $\HA$ in the usual manner:
% Considering multi-conclusion sequents $\Gamma \sdash \Delta$ instead of
% single-conclusion sequents and replacing the intuitionistic logical rules of
% $\HA$ with their classical counterparts. A cyclic proof system $\CPA$ for Peano
% arithmetic can be obtained via analogous modifications to $\CHA$, also exchanging
% the $\RCase_x$-rule for its multi-conclusion variant. Crucially, the notion of
% $\CPA$-proofhood is exactly the same as that of $\CHA$.

% \Cref{lem:full} extends to Peano arithmetic in a straightforward manner:
% \begin{theorem}\label{lem:full-pa}
%   If $\CPA \vdash \Gamma \sdash \Delta$ then $\PA \vdash \Gamma \sdash \Delta$.
% \end{theorem}
% \begin{proof}
%   Let \( \varphi^* \) denote the Gödel--Gentzen translation of \( \varphi \). A
%   simple extension of the standard argument shows that $\CPA \vdash \Gamma \sdash \Delta$ entails $\CHA \vdash
%   \Gamma^*, \neg \Delta^* \sdash \bot$. By \Cref{lem:full}, one obtains $\HA
%   \vdash \Gamma^*, \neg \Delta^* \sdash \bot$ which entails $\PA \vdash \Gamma
%   \sdash \Delta$ by the standard argument.
% \end{proof}

% It should be noted that the method presented in this article can also be applied
% to $\CPA$ directly. For this, a system $\APA$ can be derived by modifying $\AHA$
% in the usual way. The translation from $\APA$ to $\PA$ proceeds analogously as
% that in \Cref{sec:translation}, except that $\widehat{I_\Lambda} := \forall \vec{u} \leq \vec{v}. \forall
%   \vec{w}.(\bigwedge \Gamma \to \bigvee \Delta)[\vec{u} / \vec{v}]$ to account
% for the multi-conclusion sequents of the classical sequent calculi. The
% results of \Cref{sec:combinatorics} also directly transfer to the setting of
% Peano arithmetic.

% \subsection{Logical complexity}
% \label{sec:log-comp}

% Using \Cref{lem:full}, one can translate a $\CHA$-proof $\pi$ into a $\HA$-proof
% $\pi'$ with the same endsequent. In this section, we characterise the logical
% complexity of $\pi'$ in terms of that of $\pi$.

% We begin by fixing appropriate notions of logical complexity for $\HA$- and $\CHA$-proofs.
% Note that the notion of $\CC$ presented below is based on the notions $\CP_n$
% and $\CS_n$ presented in~\cite{dasLogicalComplexityCyclic2020}.

% \begin{definition}
%   Let $\lcls$ be a set of formulas.

%   A $\HA$-proof $\pi$ is said to be in $\IC$
%   if every instance of the induction axioms in $\pi$ is of the form
%   \( \varphi(0), \forall x. \varphi(x) \to \varphi(Sx) \sdash \varphi(s)  \)
%   with $\varphi \in \lcls$. If a sequent $\Gamma \sdash \delta$ is proven by a
%   $\HA$-proof $\pi$ in $\IC$ we write $\IC \vdash \Gamma \sdash \delta$.

%   A $\CHA$-proof $\pi$ is said to be in $\CC$
%   if every bud and companion of $\pi$ is labelled by sequents $\Gamma \sdash
%   \delta$ with $\Gamma, \delta \subseteq \lcls$.
%   If a sequent $\Gamma \sdash \delta$ is proven by a
%   $\CHA$-proof $\pi$ in $\CC$ we write $\CC \vdash \Gamma \sdash \delta$.
% \end{definition}

% % \begin{proposition}
% %   \
% %   \begin{enumerate}[(i)]
% %   \item Let $\lcls$ be a set of formulas and let $\Lambda \coloneq \Lambda'; x \mapsto \Gamma \sdash \delta$ be such
% %     that $\Gamma \cup \{\delta\} \subseteq \lcls$. Then $\widehat{I}_\Lambda \in
% %     \fcls$.
% %   \end{enumerate}
% % \end{proposition}

% Our characterisation relies on the following two operations on sets of formulas $\lcls$.

% \begin{definition}
%   Let $\lcls$ be a set of formulas.
%   \begin{enumerate}
%   \item We define the set $\lcls \sdash \lcls \coloneq \{ (\bigwedge \Phi) \to \psi \mid \Phi
%     \subseteq \lcls, \psi \in \lcls\}$.
%   \item We define $\Pi(\lcls)$ to be the smallest set of formulas extending \( \Theta \) that is:
%     \begin{enumerate}[(i)]
%     % \item $\lcls \subseteq \Pi(\lcls)$ and $\bot, \top \in \Pi(\lcls)$, and
%     % \item $\Pi(\lcls)$ is $\wedge\vee$-closed: If $\varphi, \psi \in \Pi(\lcls)$
%     %   and $\circ \in \{\wedge, \vee\}$ then $\varphi \circ \psi \in \Pi(\lcls)$, and
%     \item closed under bounded quantification: If $\varphi \in
%       \Pi(\lcls)$, % $Q \in \{\forall, \exists\}$,
%       $x \in \var$ and $t$ is a term not containing \( x \), then $\forall x < t.\varphi \in
%       \Pi(\lcls)$ and $\exists x < t.\varphi \in \Pi(\lcls)$, and
%     \item closed under unbounded universal quantification: If $\varphi \in \Pi(\lcls)$ and $x
%       \in \var$ then $\forall x.\varphi \in \Pi(\lcls)$.
%     \end{enumerate}
%   \end{enumerate}
% \end{definition}

% \begin{theorem}\label{lem:ha-complex}
%   Let $\lcls$ be a set of formulas. If $\CC \vdash \Gamma \sdash \delta$ then
%   $\IF \vdash \Gamma \sdash \Delta$.
% \end{theorem}
% \begin{proof}
%   Let $\pi \vdash \Gamma \sdash \delta$ be a $\CHA$-proof in $\CC$. By
%   \Cref{lem:exists-injective} there exists an injective $\CHA$-proof $\pi_i
%   \vdash \Gamma \sdash \delta$. By inspecting the proof of
%   \Cref{lem:exists-injective}, we observe that $\pi_i$ is still in $\CC$ because it
%   has the same sequents as $\pi$. Applying \Cref{thm:ind-order-SD}
%   yields a $\CHA$-proof $\pi_{\sqsubseteq^*} \vdash \Gamma \sdash \delta$ in $\CC$ with induction order
%   $\sqsubseteq^*$. By
%   \Cref{lem:cha-to-aha}, there then exists a $\AHA$-proof $\pi_{\AHA} \vdash \Gamma
%   \sdash \delta$. Because all companions and buds of $\pi_{\sqsubseteq^*}$ are labelled with
%   sequents $\Gamma' \sdash \delta'$ such that $\Gamma', \delta' \subseteq
%   \lcls$, every companion label $x^\bullet \mapsto \Gamma' \sdash \delta'$
%   occurring in $\pi_{\AHA}$ is such that $\Gamma', \delta' \subseteq \lcls$.

%   It remains to scrutinise the translation of $\AHA$-proofs to $\HA$-proof in
%   \Cref{lem:key}. We begin by analysing the logical complexity of the inductive
%   hypotheses employed in the translation. Recall that the formula $\widehat{I}_\Lambda$ for
%   $\Lambda = \Lambda'; x^\bullet \mapsto \Gamma \sdash \delta$ is defined as $\forall \vec{u} \leq \vec{v}. \forall \vec{w}.~(\bigwedge
%   \Gamma \to \delta)[\vec{u} / \vec{v}]$. If $\Gamma, \delta \subseteq
%   \lcls$ then $\bigwedge \Gamma \to \delta \in \lcls \sdash \lcls$ and
%   $\widehat{I}_\Lambda \in \fcls$.
%   % Similarly, $I_\Lambda \in \fcls$ for either
%   % case of $\bullet \in \{+, -\}$.
%   % With a small adaptation to its definition, it
%   % can be ensured that $H_n(\pi^*) \in \fcls$ for any subderivation $\pi^*$ of
%   % $\pi_\AHA$: Recall that the non-trivial part of the $\RCase_x$-clause of the definition of $H_n(-)$ is
%   % \[
%   %   \forall x' \leq x.~(x' = 0 \wedge H_{n}(\pi_0)[x' / x]) \vee (\exists x''.~x' = Sx'' \wedge H_{n}(\pi_s)[x'' / x]).
%   % \]
%   % Assuming $H_n(\pi_0), H_n(\pi_s) \in \fcls$, the formula above is in $\fcls$,
%   % were it not for the unrestricted existential quantifier. This issue can be
%   % resolved by replacing $\exists x''$ by the equivalent bounded quantification $\exists x'' <
%   % x'$. If the definition of $H_n(-)$ is adjusted in this manner, $H_n(\pi^*) \in
%   % \fcls$ for any subderivation $\pi^*$ of $\pi_{\AHA}$.
%   Applying \Cref{lem:key} to $\pi_\AHA$ yields a $\HA$-proof $\pi' \vdash \Gamma
%   \sdash \delta$. To verify that $\pi'$ is $\IF$, it suffices to check that all
%   instances of induction axioms in $\pi'$ are on $\fcls$- or $\Delta_0$-formulas. Note first
%   that \Cref{lem:key} inserts various $\Delta_0$-inductions into $\pi'$ to
%   derive various properties about the relation $x < y$.
%   Except for these, the only
%   induction axioms inserted by \Cref{lem:key} occur in the case of the
%   $\RComp$-rule and are instances of the following derived rule of $\HA$
%   \[
%     \inference[$\RInd$]{\Gamma, \forall y < x.~\widehat{I}_\Lambda[y / x] \sdash \widehat{I}_{\Lambda}}{\Gamma \sdash \forall x. \widehat{I}_\Lambda}.
%   \]
%   To derive the rule above, only one induction on $I' \coloneq \forall x \leq
%   y.~\widehat{I}_\Lambda$ is required. Because every companion label of
%   $\pi_\AHA$ contains a $\lcls$-sequent, $\widehat{I}_\Lambda \in
%   \fcls$ and thus $I' \in \fcls$.
% \end{proof}

% The notions of $\IC$ and $\CC$ transfer directly to the setting of Peano arithmetic.
% In~\cite{dasLogicalComplexityCyclic2020}, Das proves for $\PA$ that $\CS_n = \IS_{n + 1}$,
% or equivalently, that $\CP_n = \IP_{n + 1}$. \Cref{lem:ha-complex} can be used
% to conclude $\CP_n \subseteq \IP_{n + 1}$, the more involved direction of this equality.

% \begin{theorem}
%   In $\PA$, if $\CP_n \vdash \Gamma \sdash \Delta$ then $\IP_{n+1} \vdash \Gamma
%   \sdash \Delta$.
% \end{theorem}
% \begin{proof}
%   Let $\pi \vdash \Gamma \sdash \Delta$ be a $\CPA$ proof in $\CP_n$. As
%   described in \Cref{lem:full-pa}, the usual argument yields a proof $\pi^*
%   \vdash \Gamma^*, \neg \Delta^* \sdash \bot$. Crucially, every companion in
%   $\pi$ with sequent $\Upsilon \sdash \Xi$ is transformed into a companion in
%   $\pi'$ with sequent $\Upsilon^*, \neg \Xi^* \sdash \bot$. Because $\Upsilon,
%   \Xi \subseteq \Pi_n$ we obtain $\Upsilon^*, \Xi^* \subseteq \Pi_n$ and $\neg
%   \Xi^* \subseteq \Sigma_n$, making $\Upsilon^*, \neg \Xi^* \sdash \bot$ a
%   $\Delta_{n + 1}$-sequent. As any companion in $\pi'$ originates from a
%   companion in $\pi$, this means $\pi'$ is in $\CD_{n + 1}$. Applying
%   \Cref{lem:ha-complex} yields a $\HA$-proof $\pi''' \vdash \Upsilon^*, \neg \Xi^*
%   \sdash \delta$ in $\IPar{\Delta_{n + 1}}$ which is extended into a $\PA$-proof $\pi'''' \vdash
%   \Upsilon \sdash \Xi$ via the standard argument. It is easily observed that
%   $\pi''''$ is still in $\IPar{\Delta_{n + 1}}$. Classically, $\rcls{\Delta_{n +
%     1}} \subseteq
%   \Pi_{n + 1}$ and thus $\pi''''$ is in $\IP_{n + 1}$ as desired.
% \end{proof}

% \subsection{Proof size complexity}
% \label{sec:proof-size}

% Another result of Das~\cite{dasLogicalComplexityCyclic2020} relates $\CPA$ and
% $\PA$ in terms of proof size complexity: they show that translating from $\CPA$
% to $\PA$ need only incur a singly-exponential size-increase, although this
% `proof size efficient' translation maps $\CS_n$ proofs to $\IS_{n + 2}$, rather
% than $\IS_{n + 1}$.

% % While the proof size is not explicitly
% % tracked, Das identifies at least one source of exponential blowup (\cite[Lemma
% % 8.4]{berardiEquivalenceIntuitionisticInductive2017}), meaning their translation
% % at least does not exhibit better bounds.

% Our translation contains two sources of significant blow-up in proof size: (1) a
% super-exponential blow-up when transforming arbitrary $\CHA$-proofs into
% $\CHA$-proofs of cycle normal form (see \cite[Theorem
% 6.3.6]{brotherstonSequentCalculusProof2006}) and (2) an exponential blow-up
% transforming injective $\CHA$-proofs in cycle normal form to $\CHA$-proofs with
% $\sqsubseteq^*$ induction orders via \Cref{lem:good-io}. All other translation
% steps only incur polynomial size increase. Thus, we only provide an
% improvement on Das' translation witnessing $\CS_n \subseteq \IS_{n + 1}$, which exhibits a
% non-uniform size increase.

% \subsection{Applicability of the method}
% \label{sec:applicabilty}

% We conjecture that both the method of Sprenger and Dam, and our refinement
% thereof, are more widely applicable
% for translating cyclic into inductive proofs, even for logics unrelated to
% arithmetic.
% % This is supported by the application of the method of Afshari and
% % Leigh~\cite{afshariCut-freeCompleteness} to the modal $\mu$-calculus.
% As we see it, there are three major requirements on a cyclic proof system
% and its logic to be amenable to the translation methods:

% \paragraph{Internalising progress} The relations $<$ and
%   $\leq$ are key to the formulation of the inductive hypotheses. They are used
%   to express which inductive hypothesis is `ready' to be applied and for which a
%   progress step must yet occur, both in Sprenger and Dam's method and our
%   refinement.
%   Transferring the methods to
%   other logics requires formulating suitable inductive hypotheses which will
%   likely rely on analogous methods of `internalising' the progress of the cyclic
%   proof system in the logic.
%   At this stage, however, it is unclear how the idea of internalisation can
%   be formulated in a general sense that is applicable to systems that lack
%   the explicit expressivity to reason about approximations of fixed points,
%   for example, type systems and modal logics.
%   % \geldiscuss{Is this the standard order? I think internalization is `progress in a proof'. Modal \( \mu \)-calculus cannot encode closure ordinals nor explicit approximations, but it can encode its own syntax trees/derivation. I believe this is plays an analogous role (and why mmc offers its own challenges).}
% \paragraph{Cut} The treatment of companions, both in Sprenger and
%   Dam~\cite{sprengerStructureInductiveReasoning2003a} and in our method, relies on
%   the $\RCut$-rule to introduce a new inductive hypothesis. This seems likely to
%   be a key feature of the translation method, meaning any system amenable to it
%   must feature a $\RCut$-rule or $\RCut$-admissability.
% \paragraph{Induction orders} To carry out the method as presented by Sprenger
%   and Dam~\cite{sprengerStructureInductiveReasoning2003a} or us, all cyclic proofs
%   of the system must be justifiable by induction orders, or it must at least be
%   possible to transform them in such a way that they always become justifiable by
%   induction orders. For trace conditions in
%   which the objects traced along branches (e.g. variables, terms, formulas) always
%   have a unique successor, as in $\CHA$, an argument analogous to
%   \Cref{lem:exists-io} can likely be carried out. It is known that not all cyclic
%   proof systems exhibit this property, for example those that allow
%   `contractions' in their traces such as
%     Simpson's~\cite{simpsonCyclicArithmeticEquivalent2017} cyclic proof system
%     for PA. A possible path to extending our method to
%   such cases is provided by the work of Afshari and
%   Leigh~\cite{afshariFinitaryProofSystems2016}: They provide a translation from
%   cyclic to inductive proofs for the modal $\mu$-calculus. Rather than relying on
%   induction orders, their translation is obtained via reset proof systems, another
%   soundness condition for cyclic proof system. Similarly to Sprenger and Dam's
%   approach, the method can be split into two parts: (1) bringing arbitrary cyclic
%   proofs into a combinatorial normal-form and (2) translating proofs in said
%   normal-form into inductive proofs. While part (2) of their method was found to
%   contain a mistake~(see \cite{kloibhofer_note_2023}), part (1) remains correct. Indeed, their notion of
%   normal-form bears a close resemblance to $\sqsubseteq^*$ induction orders.
%   As Leigh and
%   Wehr~\cite{leighGTCResetGenerating2023} demonstrates, most cyclic proof
%   systems admit equivalent reset systems, so adapting the method presented here to
%   the setting of reset-proofs should yield a method that is more widely applicable.
