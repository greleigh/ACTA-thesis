% Emacs, this is -*-latex-*-

\chapter*{Preface}
\label{chap:introduction}
\addcontentsline{toc}{chapter}{Preface}

% % - Cycles instead of trees
% Cyclic proof systems allow derivations whose underlying structure is a finite
% graph, rather than a well-founded tree.
% % - Cycles over induction
% Such proof systems often omit induction axioms which can be simulated by
% the use of cycles.
% % - Uses: Proof search and proof theory
% This can ease applications in automated theorem proving and ordinary proof
% theory.
% % - Requires soundness conditions
% Because cyclic proofs may have infinite branches, their soundness can often
% only be
% ensured by imposing additional conditions beyond well-formedness on
% cyclic derivation.
% % - Representation: Same logical rules; different soundness conditions
% Distinct cyclic representations of the same logic can often be obtained by
% combining the same stock of `logical rules' with different soundness conditions.

% Representation matters
The thesis' title can be read in two different ways:
\begin{enumerate}
\item as stating that the thesis is concerned with representation matters of cyclic
  proof theory
\item as stating that issues of representation are important to cyclic proof theory
  as a whole
\end{enumerate}
Combining both readings yields the most accurate representation of the thesis'
content. We highlight the importance of considerations relating to the
representation of cyclic proofs to cyclic proof theory and present
technical results relating to such representation matters.

This is a `compilation thesis' and thus consists of two parts. Part
\hyperref[p:acrp]{B} is a \emph{compilation} of papers which
constitute the thesis' main technical contributions. Part
\hyperref[chap:introduction]{A} is the so-called \emph{kappa}, or introduction,
which provides some background to the papers, highlighting
an overarching theme. This chapter is already part of the kappa. In
\Cref{chap:cyclic-proofs}, we give an introduction to the field of cyclic proof
theory. \Cref{chap:representation} illustrates what we mean by `representation'
in the context of cyclic proof theory and motivates why issues related to it are
worth considering. We close in \Cref{chap:papers} by giving brief summaries of
the two included papers, connecting them to matters discussed in the kappa.


% LocalWords:  axiomatisation sequents foundedness coinductive provers Peano
% LocalWords:  invariants endsequent subtree pre formedness Heyting GTC Gödel
% LocalWords:  morphisms intuitionistic subproofs infinitary Löb Grzegorczyk
% LocalWords:  decidability uncountably Ackermann PSPACE coNP derivedness iff
% LocalWords:  subgraph subgraphs subsequence subsequences combinatorial
% LocalWords:  interpolants interpolant subproof homomorphism functors subpath
% LocalWords:  semilattice arithmetics Sprenger formalisations
