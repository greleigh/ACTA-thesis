% Emacs, this is -*-latex-*-

\chapter{Abstract Cyclic Proof Theory}
\label{chap:acds}

\section{Abstract Cyclic Proof Systems}
\label{sec:abs-ps}

% Proof systems
It is easily observed that the global trace conditions of different cyclic proof
systems share a structural similarity, only differing by the notions of `trace'
and `progress'. The same is true for other soundness conditions. In this section, we present an abstract notion of `cyclic proof system'
which relies on an abstract notion of `trace' and `progress'. These abstract
cyclic proof systems are used in \papOne{} to give a generic method of deriving
reset proof systems from cyclic proof systems with a global trace condition.

We begin by making concrete our notion of (cyclic) tree: 
\begin{definition}[Tree]
  A \emph{tree} is a
  non-empty set $T \subseteq \omega^*$ which is closed under taking prefixes. Each
  $t \in T$ is called a \emph{node} and the nodes in $\Chld(t) \coloneq \{ti \in T
  ~|~ i \in \omega\}$ are called its \emph{children}. A node $t$ is a \emph{leaf} of $T$ if
  $\Chld(t) = \emptyset$ and an \emph{inner node} otherwise.
\end{definition}

\begin{definition}[Cyclic tree]
  A \emph{cyclic tree} is a pair $(T, \beta)$ of a finite tree $T$ and
  a partial function $\beta \colon \Leaf(T) \pto \Inner(T)$ mapping some leaves of $T$
  onto inner nodes of $T$ such that $\beta(t) < t$ by the prefix ordering for
  every $t \in \dom(\beta)$. 
  If $t \in \dom(\beta)$ one calls it a \emph{bud} and $\beta(t)$ its \emph{companion}.

  Fix a cyclic tree $C = (T, \beta)$. A sequence $\pi \in T^\omega$ is a
  \emph{branch} through $C$ if $\pi_0 = \varepsilon$ and it satisfies the
  following properties at every index $i \in \omega$: (a) if $\pi_i \not\in
  \Leaf(T)$ then $\pi_{i + 1} \in \Chld(\pi_i)$ and (b) if $\pi_i \in \Leaf(T)$
  then $\pi_i \in \dom(\beta)$ and $\pi_{i + 1} = \beta(\pi_i)$.
\end{definition}
% For two words $s, t \in \omega^*$ we say $s \lro t$ (``$s$ is
% \emph{left of} $t$'') if there is an index $i$ such that $s_i < t_i$ and $s_j =
% t_j$ for all $j < i$.

\begin{definition}[Derivation system]
  A \emph{derivation system} is a triple $(\Seq, \RR, \rho)$ consisting of a set of
  sequents $\Seq$ and a set $\RR$ of derivation rules and a rule-interpretation
  $\rho \colon \RR \to \Seq^*$ such that for each $R \in \RR$, $\rho(R) = 
  (\Gamma, \Delta_1, \ldots, \Delta_{n - 1}) \in \Seq^n$ for $n > 0$. 
  The sequent $\Gamma$ is \emph{conclusion} of $R$ and the $\Delta_i$ its
  \emph{premises}. Henceforth, we refer to a derivation system $(\Seq, \RR,
  \rho)$ simply by $\RR$.
\end{definition}

\begin{definition}[$\RR$-preproof]
  An \emph{$\RR$-preproof} is a triple
  $\Pi = (C, \lambda, \delta)$ consisting of a cyclic
  tree $C = (T, \beta)$ together with a labeling $\lambda \colon T \to \Seq$ such
  that for every $t \in \dom(\beta)$ one has $\lambda(t) = \lambda(\beta(t))$ and
  a partial function $\delta \colon (T \setminus \dom(\beta)) \pto \RR$ such that for
  each $t \in T \setminus \dom(\beta)$
  \begin{itemize}
  \item either $t \in \dom(\delta)$ with $\rho(\delta(t)) = (\Gamma, \Delta_1, \ldots,
    \Delta_n)$ and $\lambda(t) = \Gamma$ and furthermore $\Chld(t) = \{t_1,
    \ldots, t_n\}$ and $\lambda(t_i) = \Delta_i$ 
  \item or $t \in \Leaf(T)$.
  \end{itemize}
  Denote by $\Pp(\RR)$ for the set of $\RR$-preproofs.
  The sequent $\lambda(\varepsilon)$ is called the \emph{endsequent} of $\Pi$. Each leaf $o \in
  \Leaf(T) \setminus \dom(\delta)$ is called \emph{open} and its associated
  sequent $\lambda(o)$ is a \emph{assumption} of $\Pi$.
  % By convention, we list the
  % open leaves (and their corresponding assumptions) from left to right, according
  % to $\lro$.
\end{definition}

\begin{definition}[Cyclic proof system]
  A \emph{cyclic proof system} is a tuple $(\Seq, \RR, \rho, \SC)$ consisting of a
  derivation system $(\Seq, \RR, \rho)$ and a set $\SC \subseteq \Pp(\RR)$ of $\RR$-preproofs
  without assumptions called \emph{$\RR$-proofs}. Any $\Pi \in \SC$ with endsequent $\Gamma$ is called
  \emph{a proof of ~$\Gamma$}. Such a preproof is said to satisfy the
  \emph{soundness condition} of $\RR$. We extend the naming convention for
  derivation systems to cyclic derivation systems, referring to $(\Seq, \RR,
  \rho, \SC)$ by $\RR$.
\end{definition}

\TODO{Say something about \Cref{sec:abs-ps}}

Observe that the specification of $\CPA$ we given in \Cref{sec:cpa} fits the
definitions above. In that case, $\SC$ contains precisely those $\CPA$
pre-proofs which satisfy the global trace condition. Allowing different
derivation rules $r, r' \in \RR$ with the same interpretation, i.e.\ $\rho(r) =
\rho(r')$, is useful when considering cyclic proof systems in which there can
sometimes be
two rule-instances with identical premises and conclusion
which have different `effects' on traces through them.

\section{Trace Categories}
\label{sec:tcs}

In the previous section, we left the soundness conditions quite vague. In this
section we describe one kind of soundness condition: the global trace condition.
More specifically, we describe a generic way of specifying the global trace
condition in terms of certain categories. We then go on to define a family of
such categories that are sufficient to specify most global trace conditions from
the literature. The definitions we give in this
section are adapted from~\cite{afshariAbstractCyclicProofs2022}
and~\cite{wehrAbstractFrameworkAnalysis2021}.

% Trace categories
A \emph{semi-category} is a category which may not have identity morphisms.
That is, a semi-category $\CC$ consists of a collection of objects $\Ob(\CC)$ 
and collections of morphisms  $\Hom_{\CC}(X, Y)$ between each pair of objects  $X, Y \in
\Ob(\CC)$. There is a composition
$\mathrel{\circ}\, : \Pi X, Y, Z \in \Ob(\CC).~\Hom_{\CC}(Y, Z) \times \Hom_{\CC}(X,
Y) \to \Hom_{\CC}(X, Z)$ which is associative.
A \emph{semi-functor} is a semi-category homomorphism. That is, for
semi-categories $\CC, \DD$, a semi-functor $F : \CC \to \DD$ consists of a map
on objects $F_0 : \Ob(\CC) \to \Ob(\DD)$ and a further map $F_2$ on
morphisms of $C$ such that for $f : X \to Y$ we have $F_1(f) : F_0(X)
\to F_0(Y)$.
$F_1$ distributes over the composition
operation, i.e. $F_1(g \circ f) = F_1(g) \circ F_1(f)$. As is also common for standard functors,
we denote both $F_0$ and $F_1$ by $F$.

The standard $<$-ordering of the natural numbers $\omega$ induces a semi-category whose
objects are the natural numbers and in which there is a unique morphism
between $n$ and $m$, denoted `$n < m : n \to m$', if $n < m$. The $\leq$-ordering
 induces an analogous proper category. In this article, we denote both of
these categories by $\omega$; which one is meant will
be clear from the context.

A \emph{path} through a category $\TT$ is a functor $P \colon \omega \to \TT$.
Given $P, P' \colon \omega \to \TT$ we call $P$ a subpath of $P'$, written  $P \subseteq
  P'$, if there is a semi-functor $S \colon \omega \to \omega$ between the $\omega$-semi-categories such that $P = P'
  \circ S$. The transitive, symmetric closure of $\subseteq$ is
  denoted $\sim$.
In other words, $P \subseteq P'$ holds if $P'$ can be transformed
  into the $P$ by applying a combination of two path transformations:
  \begin{enumerate}
  \item 
    Discarding a finite prefix of path $P'$, for example by taking $S(m) \coloneq m + n$.
  \item 
    Composing morphisms along $P'$. For example, if $P'$ is of the form
    \[X_0^0 \xrightarrow{\tau^0_0} X_0^1 \xrightarrow{\tau^1_0} \cdots
      X_0^{n_0} \xrightarrow{\tau_0^{n_0}} X_1^0 \xrightarrow{\tau_1^0} \cdots X_1^{n_1}
      \xrightarrow{\tau_1^{n_1}} X_2^0 \xrightarrow{\tau_2^0} \cdots X_2^{n_2} \xrightarrow{\tau_2^{n_2}} \hdots\]
    then $P \subseteq P'$ for the path $P$ below
    \[X_0^0 \xrightarrow{\tau^{n_0}_0 \circ \cdots \circ \tau^0_0} X_1^0
      \xrightarrow{\tau^{n_1}_1 \circ \cdots \circ \tau^{0}_1 } X_2^0
      \xrightarrow {\tau^{n_2}_2 \circ \cdots \circ \tau^{0}_2 } \hdots\]
    witnessed by $S(i) \coloneq \sum_{j < i} (n_j + 1)$.
  \end{enumerate}

Infinite branches of cyclic proofs may be represented as paths through suitable categories.

\begin{definition}[Trace category]\label{def:trace-cond}
A \emph{trace category} is a category $\TT$ equipped with a \emph{trace
  condition} \( C_{\TT} \) on paths, invariant under $\sim$, i.e.
  if $P \sim P'$ then $C_{\TT}(P)$ if and only if $C_{\TT}(P')$ holds.
\end{definition}

The notion of trace category is abstract, only requiring the trace
condition to be closed under the subpath relation. However, this already gives
some insight in the nature of trace conditions. For example, all trace
conditions must be closed under discarding and adding prefixes to branches. In
the trace condition of $\CPA$, as in most other
trace conditions in the literature, this is ensured by explicitly allowing the
progressing trace to start at any point of the branch. Indeed, we are not aware
of any trace condition in the cyclic proof theory literature which is not closed
under prefixes.

We next define trace interpretations, which allow specifying notion of trace and
progress of a derivation system $\RR$ in terms of a trace category $\TT$. The
sequents of $\RR$ are identified with objects of $\TT$ and each `step' from a
conclusion to a premise in a derivation rule of $\RR$ is associated with a
morphism between the objects associated to said conclusion and premise. Under
this interpretation, every branch through a preproof induces a path $\omega \to
\TT$ which allows for a general specification of the global trace condition in
terms of the trace condition of $\TT$. Multiple examples of such trace
interpretations for cyclic proof systems from the literature can be found in
\Cref{sec:concrete} and in \cite[Chapter 6]{wehrAbstractFrameworkAnalysis2021}.

\begin{definition}[Trace intrepretation]
  Let $(\Seq, \RR, \rho)$ be a derivation system and $\TT$ a trace category. A
  \emph{trace interpretation} $\iota \colon \RR \to \TT$ consists of a map $\iota \colon \Seq
  \to \Ob(\TT)$ and for each rule $r \in \RR$ with $\rho(r) = (\Gamma, \Delta_1,
  \ldots, \Delta_n)$ morphisms $r_i \colon \iota(\Gamma) \to \iota(\Delta_i)$ for each \( 1 \le i \le n \).
\end{definition}

\begin{definition}[$\iota(\RR)$]\label{def:tc-induced}
  Let $(\Seq, \RR, \rho)$ be a derivation system with a trace interpretation 
  \( \iota \colon \RR \to \TT\). Let $\Pi = (C, \lambda, \delta)$ be a preproof and $\pi$ be a a path
  through $C$. This induces a path $\hat{\pi} \colon \omega \to \TT$ given by \(
  \hat{\pi}(i) \coloneq \iota(\lambda(\pi_i)) \) and
  \[ \hat{\pi}(i < i + 1) \coloneq
    \begin{cases}
      \delta(\pi_i)_j \colon \iota(\lambda(\pi_i)) \to \iota(\lambda(\pi_{i + 1})) & \pi_{i + 1} = \pi_{i + 1} j \\
      1_{\hat{\pi}(i)} & \pi_i \in \dom(\beta)
    \end{cases}
  \]

  This induces a cyclic proof system $\iota(\RR) \coloneq (\Seq, \RR, \rho,
  \SC)$ with $\Pi \in \SC$ iff $\Pi$ has no open assumptions and for every path
  $\pi$ through $\Pi$ the trace condition $C_{\TT}(\hat{\pi})$ of $\TT$ holds.
\end{definition}

The trace conditions in the literature are usually formulated in terms of a
notion of `trace', or rather some syntactic feature which is `traced' along a
branch, and a notion of `progress'. We present a family of trace categories
which formalises this observation. The notion of `progress' is modelled by
\emph{activation algebras}.
An activation algebra is a tuple $\Aa = (A, \leq, \vee, 0, \alpha)$
consisting of
a finite semilattice $(A, \leq, \vee, 0)$ and an \emph{activation
  element} $\alpha \in A$ where $0 \neq \alpha$. We often write $\Aa$ to refer
to the carrier set $A$. For a proof that the categories given below are indeed trace
categories, i.e.\ that their trace condition is closed under $\sim$, refer to
\parencite[Proposition 5.9]{wehrAbstractFrameworkAnalysis2021}. \TODO{Maybe include proofs?}

\begin{definition}[Trace category $\TT_{\Aa}$]\label{def:aa-activ-tc}
  Let $\Aa$ be an activation algebra.
  The \emph{$\Aa$-activated trace category $\TT_{\Aa}$} has as its objects the finite sets. The
  morphisms between sets $X, Y$ are relations $\tau \subseteq X \times \Aa
  \times Y$.
  The identities are $1_X \coloneq \{(x, 0, x) ~|~ x \in X\}$. We often write $x \tau^a y$ to mean $(x, a, y) \in \tau$.
  Given morphisms $\tau \colon X \to Y, \tau' \colon Y \to Z$ their composition is specified by $(x, c, z) \in \tau' \circ R$ iff
  \[\exists y \in Y \exists a, b
    \in A.\;(x, a, y)
    \in \tau \text{ and } (y, b, z) \in \tau' \text{ and } a \vee b = c.\]

  A path $P \colon \omega \to \TT_{\Aa}$
  \emph{satisfies the trace condition} if there exists a subpath $P' \subseteq
  P$ and a sequence $\sigma \colon \Pi i \in \omega. P'(i)$ along it such that $\sigma_i P'(i <
  i + 1)^\alpha \sigma_{i + 1}$ for all $i \in \omega$.
\end{definition}

\begin{example}
  The simplest example of an activation algebra is the binary Boolean algebra $\BB =
  \{0, 1\}$.
  The choice $\alpha \coloneq 1$ is forced by $0 \neq \alpha$.
  This activation algebra is implicit in the
  abstract notion of trace previously considered by \textcite{brotherstonSequentCalculusProof2006}
  and suffices to model most trace conditions in the literature:
  traces have
  progress points (represented by triples $(x, 1, y)$ in maps of $\TT_{\BB}$) and a
  path satisfies the trace condition if it has infinitely many progress points.
  It is easily verified that the trace condition of $\TT_{\BB}$ is precisely this.
\end{example}

\begin{example}
  Some logics, especially $\mu$-calculi (see \Cref{def:ff-mu-tc} for an example), require more complex activation algebras
  to model their trace conditions. An example is the \emph{failure algebra} $\FF
  \coloneq (\{0, 1, 2\}, \leq, \vee, 0, 1)$. The value $2 > \alpha$ can be used to
  represent `failure' of trace condition satisfaction. That is,
  traces through $\TT_{\FF}$ satisfy the trace condition if they have
  infinitely many progress points (triples $(x, 1, y)$ in maps of $\TT_{\FF}$) and
  \emph{no} failure points (triples $(x, 2, y)$ in maps of $\TT_{\FF}$). Again,
  the trace condition of $\TT_{\FF}$ ensures precisely this condition.
  The failure
  algebra appears in the literature as one of the common trace conditions for
  the modal $\mu$-calculus ().
\end{example}

The trace categories $\TT_{\Aa}$ are a natural medium for the study of cyclic
proof theory. They are abstract enough to capture many trace conditions from the
literature but also concrete enough to allow various theorems of cyclic proof
theory to be derived for them, such as the decidability result below.

\begin{proposition}\label{lem:tc-dec}
  Fix a derivation system $\RR$ and an $\RR$-preproof $\Pi$. Given a trace
  interpretation $\iota \colon \RR \to \TT_{\Aa}$, it is decidable whether $\Pi$ is a
  proof in $\iota(\RR)$.
\end{proposition}
\begin{proof}
  There are various ways of proving this. For example by appealing to infinite
  word automata~\cite[Theorem 4.4]{wehrAbstractFrameworkAnalysis2021} or to
  Ramsey's theorem~\cite[Theorem 3]{afshariAbstractCyclicProofs2022}.
\end{proof}


We illustrate the use of $\TT_{\Aa}$ to specify trace conditions by supplying such
a specification for $\CPA$ in
terms of a trace interpretation $\iota : \CPA \to \TT_{\BB}$.

\begin{definition}\label{def:bb-cpa-tc}
  The trace interpretation $\iota \colon \CPA \to \TT_{\BB}$ is given by
  \(
  \iota(\Gamma \sdash \Delta) \coloneq \FV(\Gamma, \Delta)\).
  For each $r \in \CPA$ with $\rho(r) = (\Gamma \sdash \Delta, \Gamma_1 \sdash
  \Delta_1, \ldots, \Gamma_n \sdash \Delta_n)$
  the trace maps $r_i \colon \iota(\Gamma \sdash \Delta) \to \iota(\Gamma_i
  \sdash \Delta_i)$ are defined by
  \(
  r_i \coloneq \{(x, a(r, i, x), x) ~|~ x \in \FV(\Gamma, \Delta) \cap \FV(\Gamma_i, \Delta_i) \}
  \) where
  \[
    a(r, i, x) \coloneq
    \begin{cases}
      1, & \text{ if } r \text{ instance of } \RCase_x \text{ and } i = 2 \\
         & \text{ (i.e.\ } \Gamma_i \sdash \Delta_i \text{ is the right-hand premise)} \\
      0, & \text{ otherwise.}
    \end{cases}
  \]
\end{definition}

From the fact that the only $(x, 1, x)$ pair can be found in the trace map
leading from the conclusion to the right-hand premise of the $\RCase_x$-rule, it
is easily concluded that a $\CPA$ pre-proof satisfies the global trace condition
induced by \Cref{def:bb-cpa-tc} iff it satisfies that specified in
\Cref{def:bb-cpa-gtc}. A detailed proof of this requires a rather combinatorial
argument.
The curious reader can find a detailed version of a similar argument in
\parencite[Proposition 2]{afshariAbstractCyclicProofs2022}.\TODO{Include this?}

\TODO{$\CHAl$ TI}
\begin{definition}
  The \emph{trace interpretation} $\iota \colon \CA \to \TT_{\BB}$ is given by $\iota(\Gamma
  \sdash \Delta) \coloneq \term(\Gamma \sdash \Delta)$ and for any $R \in \CA$
  with $\rho(R) = (\Gamma \sdash \Delta, \Gamma_1 \sdash \Delta_1, \ldots,
  \Gamma_n \sdash \Delta_n)$ the trace map $r_i \colon \term(\Gamma \sdash \Delta) \to
  \term(\Gamma_i \sdash \Delta_i)$ is given by
  \begin{align*}
    r_i \coloneq & \{(t, 0, t') ~|~ t \in \term(\Gamma \sdash \Delta), t' \in \term(\Gamma_i
                   \sdash \Delta_i) \text{ and } t' \leftarrow^i_R t\} \,\cup \\
                 & \left\{(t, 1, s) ~\middle|~
                   \begin{array}{l}
                     t \in \term(\Gamma \sdash \Delta) \text{ and } t', s \in \term(\Gamma_i \sdash \Delta_i) \\
                     \text{such that } t \leftarrow^i_R t' \text{ and } s < t' \in \Gamma_i
                   \end{array}
                   \right\}
  \end{align*}
  This trace interpretation induces the soundness condition of $\CA$ as described in
  \Cref{def:tc-induced}.
\end{definition}

\section{Abstract Soundness Conditions?}
\label{sec:asc}

% GTCs vs IOs
Most soundness conditions
can be
made sense of in terms of this more abstract notion of trace. For example, the
induction orders in \Cref{def:io} can be presented in terms of $\TT_{\Aa}$.
In a slight abuse of notation, we denote the composition of the trace maps along
a simple cycle $\gamma(t)$ by $\gamma(t)$.

\begin{definition}[Induction order]\label{def:io-abstract-a}
  Let $(\Seq, \RR, \rho)$ be a deduction system and $\iota : \RR \to \TT_{\Aa}$ be
  a trace interpretation such that for every $r \in \RR$ and $1 \leq i < n$ for
  $\rho(r) = (\Gamma, \Delta_1, \ldots, \Delta_n)$ it is the case that $(x, a,
  y) \in r_i : \iota(\Gamma) \to \iota(\Delta_i)$ implies $x = y$.

  Then an induction order for an $\iota(\RR)$ pre-proof $\Pi = (C, \lambda, \delta)$ is
  a pair $(\preceq, x_{-})$ consisting of a
  pre-order
  $\mathrel{\preceq}\: \subseteq
  \dom(\beta) \times \dom (\beta) $ together with a mapping $x_{-} : \Pi s \in \dom(\beta).~\iota(\lambda(s)))$
  such that
  \begin{itemize}
  \item every strongly connected subgraph $C[\eta]$ has a
    $\preceq$-maximal element,
  \item if $s \preceq t$ then the cycle $\gamma(s)$ \emph{preserves} $x_t$, i.e.\ $(x_t,
    a, x_t) \in \gamma(s)$ for some $a \leq \alpha$,
  \item the cycle $\gamma(s)$ \emph{progresses} $x_s$, i.e.\ $(x_s, \alpha, x_s) \in \gamma(s)$
  \end{itemize}
\end{definition}

This definition reveals a syntactic feature of $\CPA$ we relied on in our
definition of induction orders in \Cref{def:io}: Traces always trace a single
variable, thus allowing the trace values $x_t$ to be specified at a per-cycle level.
The notion of induction order can be generalised to arbitrary trace
interpretations $\iota : \RR \to \TT_{\Aa}$, which may introduce more
`complicated' traces, albeit this requires a more complicated
definition, given below.

\begin{definition}[Induction order]\label{def:io-abstract-b}
  Let $(\Seq, \RR, \rho)$ be a deduction system and let $\iota : \RR \to
  \TT_{\Aa}$ be a trace interpretation.
  An induction order for an $\iota(\RR)$ pre-proof $\Pi = (C, \lambda, \delta)$
  is a pre-order
  $\mathrel{\preceq}\: \subseteq
  \dom(\beta) \times \dom (\beta)$ such that
  the following
  condition holds for every $t \in \dom(\beta)$: Let $\eta$ be maximal for $t$,
  i.e.\ the greatest $\eta \subseteq \dom(\beta)$ for which $t$ is
  $\preceq$-maximal and $C[\eta]$ is a strongly connected subgraph of $C$.
  We write $\tau_{uv}$ for $u, v \in \Node(T)$ for the trace map obtained by
  composing the maps given by the trace interpretation for the shortest path
  from $u$ to $v$.
  There must exist elements $x_t^{-} : \Pi s \in \eta.~\iota(s)$ for each bud in
  $\eta$ such that,
  \begin{enumerate}[label=\roman*.]
  \item for any $u, v \in \eta$ with $\beta(u)$ below $v$, $(x_t^{u}, a, x_t^{v}) \in
    \tau_{\beta(u) v}$ with $a \leq \alpha$
  \item for any $u, v \in \eta$ with $\beta(u)$ below $\beta(v)$, $(x_t^u, a, x_t^v) \in
    \tau_{\beta(u) \beta(v)}$ with $a \leq \alpha$
  \item $(x_t^t, \alpha, x_t^t) \in \tau_{\beta(t) t}$ for the $\preceq$-maximal $t$
  \end{enumerate}
\end{definition}
Conditions i.~and
ii.~correspond to the \emph{preservation} condition and iii.~to the
\emph{progress} condition of induction orders as presented in \Cref{def:io}.
The selection of elements $x_t^{-}$ corresponds to the fixed elements $x_t$ in
the simpler \Cref{def:io-abstract-a}, taking into account that the identity of
the traced element may change between nodes of the proof.
Under the preconditions of \Cref{def:io-abstract-a}, the two definitions
coincide, as $x_t^u = x_t^v$ for any $u, v \in \eta$. For both conditions, one
can prove that any $\iota(\RR)$ pre-proof with an induction order satisfies the
soundness condition of $\iota(\RR)$ with an argument analogous to \Cref{lem:io-correct}.

% Simple vs complex traces
In this thesis, we often describe the traces of $\CPA$ as `simple', contrasting
them with the more `complex' traces of the cyclic Peano arithmetic
considered by \textcite{simpsonCyclicArithmeticEquivalent2017}. Using the trace
categories $\TT_{\Aa}$, we can characterise some trace phenomena which introduce
complexity. For example, in any trace map $\tau$ introduced via the trace
interpretation $\iota : \CPA \to \TT_{\BB}$ in \Cref{def:bb-cpa-tc}, it follows
from $(x, a, y), (x, b, z) \in \tau$ that $a = b$ and $x = y = z$. In contrast,
the interpretation of Simpson's proof system may produce trace maps $\tau$ such
that for $(x, a, y), (x, b, z) \in \tau$ both $a \neq b$ and $y \neq z$ are
possible. Furthermore, it may also produce trace maps with $(x, a, y), (z, b, y)
\in \tau$. When working with Simpson's proof system, one must be much more
careful in definitions involving traces, for example having to rely on the more
complex definition of induction orders. Hence, another important point when choosing
the representation of one's logic as a cyclic proof system is thus to strive to
keep the traces as `simple', in the sense above, as possible.

The discussion in this section reveals a crucial fact about the abstract notions
of \textcite{wehrAbstractFrameworkAnalysis2021}: While the notion of trace
category is sufficient to define the global trace condition, the
family $\TT_{\Aa}$ of trace categories which feature a concrete notion of `trace'
and `progress' are often better suited to the abstract discussion of soundness
conditions more generally. Indeed, \papOne{} also operates at the level of
$\TT_{\Aa}$, rather than general trace categories.

\section{Counterexample Zoo}
\label{sec:zoo}

\section{PSPACE-completeness of Cyclic Proof Checking}
\label{sec:pspace}


% Trace vs soundness condition
% In the cyclic proof theory literature, the terms `trace condition' and `global
% trace condition' are often used interchangeably. The abstract notions of cyclic
% proof considered in this section clearly demonstrate that these are two separate
% notions. A trace condition simply specifies which infinite branches are sound,
% often in terms of objects `traced' along a branch which can `progress'. The
% soundness conditions of cyclic proof systems are almost always defined in terms
% of a trace condition but the global trace condition is only one among many.
